%===============================================================================
% LaTeX sjabloon voor de bachelorproef toegepaste informatica aan HOGENT
% Meer info op https://github.com/HoGentTIN/latex-hogent-report
%===============================================================================

\documentclass[dutch,dit,thesis]{hogentreport}

% TODO:
% - If necessary, replace the option `dit`' with your own department!
%   Valid entries are dbo, dbt, dgz, dit, dlo, dog, dsa, soa
% - If you write your thesis in English (remark: only possible after getting
%   explicit approval!), remove the option "dutch," or replace with "english".

\usepackage{lipsum} % For blind text, can be removed after adding actual content

%% Pictures to include in the text can be put in the graphics/ folder
\usepackage{adjustbox}
\usepackage{graphicx}
\graphicspath{{graphics/}}

%% For source code highlighting, requires pygments to be installed
%% Compile with the -shell-escape flag!
\usepackage[section]{minted}
% Info: https://www.overleaf.com/learn/latex/Code_Highlighting_with_minted
\renewcommand{\listoflistingscaption}{Lijst van broncode}
% Info: https://www.overleaf.com/learn/latex/Code_listing#Using_listings_to_highlight_code
\usepackage{comment}

\usepackage[printonlyused]{acronym}
%\usepackage[acronym]{glossaries}



\usemintedstyle{solarized-light}
\definecolor{bg}{RGB}{253,246,227} %% Set the background color of the codeframe

%% Change this line to edit the line numbering style:
\renewcommand{\theFancyVerbLine}{\ttfamily\scriptsize\arabic{FancyVerbLine}}

%% Macro definition to load external java source files with \javacode{filename}:
\newmintedfile[javacode]{java}{
    bgcolor=bg,
    fontfamily=tt,
    linenos=true,
    numberblanklines=true,
    numbersep=5pt,
    gobble=0,
    framesep=2mm,
    funcnamehighlighting=true,
    tabsize=4,
    obeytabs=false,
    breaklines=true,
    mathescape=false
    samepage=false,
    showspaces=false,
    showtabs =false,
    texcl=false,
}

% Other packages not already included can be imported here

%%---------- Document metadata -------------------------------------------------
% TODO: Replace this with your own information
\author{Dylan Pylyser}
\supervisor{Dhr. G. Bosteels}
\cosupervisor{Dhr. J. Creten}
\title{Microsoft administration met Microsoft Graph: van module naar API}
\academicyear{\advance\year by -1 \the\year--\advance\year by 1 \the\year}
\examperiod{2}
\degreesought{\IfLanguageName{dutch}{Professionele bachelor in de toegepaste informatica}{Bachelor of applied computer science}}
\partialthesis{false} %% To display 'in partial fulfilment'
%\institution{Internshipcompany BVBA.}

%% Add global exceptions to the hyphenation here
\hyphenation{back-slash}

%% The bibliography (style and settings are  found in hogentthesis.cls)
\addbibresource{bachproef.bib}            %% Bibliography file
\addbibresource{../voorstel/voorstel.bib} %% Bibliography research proposal
\defbibheading{bibempty}{}

%% Prevent empty pages for right-handed chapter starts in twoside mode
\renewcommand{\cleardoublepage}{\clearpage}

\renewcommand{\arraystretch}{1.2}

%% Content starts here.
\begin{document}

%---------- Front matter -------------------------------------------------------

\frontmatter

\hypersetup{pageanchor=false} %% Disable page numbering references
%% Render a Dutch outer title page if the main language is English
\IfLanguageName{english}{%
    %% If necessary, information can be changed here
    \degreesought{Professionele Bachelor toegepaste informatica}%
    \begin{otherlanguage}{dutch}%
       \maketitle%
    \end{otherlanguage}%
}{}

%% Generates title page content
\maketitle
\hypersetup{pageanchor=true}

%%=============================================================================
%% Voorwoord
%%=============================================================================

\chapter*{Woord vooraf}%
\label{ch:voorwoord}

% TODO: OK?
%% Het voorwoord is het enige deel van de bachelorproef waar je vanuit je
%% eigen standpunt (``ik-vorm'') mag schrijven. Je kan hier bv. motiveren
%% waarom jij het onderwerp wil bespreken.
%% Vergeet ook niet te bedanken wie je geholpen/gesteund/... heeft

Deze scriptie werd geschreven in het kader van het voltooien van de opleiding Toegepaste Informatica in de specialisatie System \& Network Administrator. \\

Het onderwerp van de bachelorproef werd me aangereikt via mijn stagebedrijf Easi. In het specifiek door de heer Jarne Creten, die ook het co-promotorschap heeft opgenomen. Ik heb van in het begin steeds een grote interesse gehad in Microsoft-technologieën. Microsoft Graph leek me een geschikte kandidaat wanneer dit onderwerp tot me werd geïntroduceerd. De complexiteit en mogelijkheden van Microsoft Graph zorgde voor extra motivatie. Daarnaast ben ik ook gefascineerd door de cloud, hierdoor wou ik me volledig verdiepen tot dit onderwerp. \\

Voor de volgende personen wil ik graag een oprechte dankuwel uitdrukken. Zonder hun hulp, moeite, tijd en zoveel meer zou deze scriptie niet tot stand zijn gekomen. \\

Eerst en vooral wil ik mijn co-promotor, Jarne Creten, bedanken voor zijn rol in dit onderzoek. Hij heeft mij tot dit onderwerp geïntroduceerd. Daarnaast kon ik dag en nacht bij hem terecht voor elk obstakel dat in mijn weg lag. Hij gaf me steeds de juiste feedback en wou dat dit onderzoek een succes zou worden. \\

Als tweede wil ik mijn promotor, Gertjan Bosteels, bedanken voor zijn sturing tijdens dit uitwerkingsproces. Hij gaf me de feedback die ik nodig had en stond steeds klaar om mijn vragen te beantwoorden. \\

Ook wil ik de mede-studenten van mijn academiejaar bedanken voor hun feedback. Bij elke vraag wouden ze mee naar een antwoord zoeken. \\

Als laatste wil ik mijn familie en vrienden bedanken. Sinds de start van mijn opleiding stonden ze steeds klaar voor mij. Het was een enorme mentale steun om te weten dat zij steeds in mij geloofden. \\

Ik wens u van harte een leuke en boeiende leeservaring toe!


% !TeX spellcheck = nl_NL-Dutch
%%=============================================================================
%% Samenvatting
%%=============================================================================

% TODO: De "abstract" of samenvatting is een kernachtige (~ 1 blz. voor een thesis) synthese van het document.
%
% Een goede abstract biedt een kernachtig antwoord op volgende vragen:
%
% 1. Waarover gaat de bachelorproef?
% 2. Waarom heb je er over geschreven?
% 3. Hoe heb je het onderzoek uitgevoerd?
% 4. Wat waren de resultaten? Wat blijkt uit je onderzoek?
% 5. Wat betekenen je resultaten? Wat is de relevantie voor het werkveld?
%
% Daarom bestaat een abstract uit volgende componenten:
%
% - inleiding + kaderen thema
% - probleemstelling
% - (centrale) onderzoeksvraag
% - onderzoeksdoelstelling
% - methodologie
% - resultaten (beperk tot de belangrijkste, relevant voor de onderzoeksvraag)
% - conclusies, aanbevelingen, beperkingen
% === SAMENVATTING ===
%\IfLanguageName{english}{%
\selectlanguage{dutch}
\chapter*{Samenvatting}

% Taalcheck OK

Deze scriptie onderzoekt de evolutie van Azure Active Directory Graph naar Microsoft Graph binnen het domein van Microsoft administration. Het bedrijf Easi maakt gebruik van de uitfaserende Azure Active Directory Graph en wil weten of Microsoft Graph klaar is om deze uitfaserende technologie te vervangen. Het doel van het onderzoek is om te bepalen of Microsoft Graph op dit moment in staat is om de beheertaken die eerder mogelijk waren met Azure Active Directory Graph over te nemen. Naast het beantwoorden van deze vraag, zijn de onderzoeksdoelstellingen het herwerken van een PowerShell-script met Microsoft Graph en het vergroten van de kennis over Microsoft Graph voor Microsoft-systeembeheerders. De methodologie van het onderzoek bestaat uit twee delen. In eerste instantie wordt een vergelijkende studie uitgevoerd op basis van vier criteria om Azure Active Directory Graph en Microsoft Graph te analyseren. Vervolgens wordt een Proof-of-Concept gerealiseerd door middel van het herschrijven van een PowerShell-script met behulp van Microsoft Graph PowerShell-modules. De resultaten van het onderzoek tonen aan dat elf van de twintig functies volledig kunnen worden herwerkt, terwijl twee functies deels zijn herwerkt. De resterende zeven functies zijn omtrent niet-persoonlijke mailboxen en kunnen niet worden omgezet, wat leidt tot limitaties van de technologie vandaag de dag. Daarnaast concludeert de vergelijkende studie dat beide technologieën gelijkenissen vertonen, maar toch verschillen op eindpunt, aanspreekbare data-objecten en dependencies. De resultaten van beide delen tonen aan dat Microsoft Graph in staat is om de beheertaken van Azure Active Directory Graph over te nemen. Op basis van de resultaten wordt volgende conclusie getrokken. Het blijkt dat Microsoft Graph verdergaat dan enkel het vervangen van Azure Active Directory Graph. Azure Active Directory Graph is gefocust op het beheren van Azure Active Directory, terwijl Microsoft Graph vandaag de dag ook verschillende Microsoft-entiteiten zoals Office 365 en bijhorende applicaties kan beheren. 

%  en kan verschillende Microsoft-entiteiten zoals Office 365 en bijhorende applicaties
%\selectlanguage{english}

%%---------- Samenvatting -----------------------------------------------------
% De samenvatting in de hoofdtaal van het document

% Azure \ac{AD} is gefocust op het beheren van Azure \ac{AD}, terwijl Microsoft Graph vandaag de dag ook Microsoft Teams, Outlook, To Do en andere entiteiten kan aanspreken naast Azure \ac{AD}. Azure \ac{AD} kan minder Microsoft-entiteiten aanspreken in vergelijking met Microsoft Graph. 


%---------- Inhoud, lijst figuren, ... -----------------------------------------

\tableofcontents

% In a list of figures, the complete caption will be included. To prevent this,
% ALWAYS add a short description in the caption!
%
%  \caption[short description]{elaborate description}
%
% If you do, only the short description will be used in the list of figures

\listoffigures

% If you included tables and/or source code listings, uncomment the appropriate
% lines.

\listoftables 


% \listoflistings
% \addcontentsline{toc}{chapter}{Lijst van broncode}
    


% Als je een lijst van afkortingen of termen wil toevoegen, dan hoort die
% hier thuis. Gebruik bijvoorbeeld de ``glossaries'' package.
% https://www.overleaf.com/learn/latex/Glossaries

\chapter{Lijst van afkortingen}
\begin{acronym}%\footnotesize
\acro{2FA}{Two-Factor Authentication}
\acro{AD}{Active Directory}
\acro{ADAL}{Azure Active Directory Authentication Library}
\acro{API}{Application Programming Interface}
\acro{CLI}{Command Line Interface}
\acro{DNS}{Domain Name System}
\acro{EU}{Europese Unie}
\acro{EC2}{Elastic Compute Cloud}
\acro{HTTP}{Hypertext Transfer Protocol}
\acro{IT}{Information Technology}
\acro{JSON}{JavaScript Object Notation}
\acro{MFA}{Multi-Factor Authentication}
\acro{MSAL}{Microsoft Authentication Library}
\acro{On-prem}{On-premises}
\acro{URL}{Uniform Resource Locators}
\acro{REST}{Representational State Transfer}
\acro{SDK}{Software Development Kit}
\acro{SMTP}{Simple Mail Transfer Protocol}
\acro{XML}{Extensible Markup Language}
\end{acronym}
%---------- Kern ---------------------------------------------------------------

\mainmatter{}

% De eerste hoofdstukken van een bachelorproef zijn meestal een inleiding op
% het onderwerp, literatuurstudie en verantwoording methodologie.
% Aarzel niet om een meer beschrijvende titel aan deze hoofdstukken te geven of
% om bijvoorbeeld de inleiding en/of stand van zaken over meerdere hoofdstukken
% te verspreiden!

%%=============================================================================
%% Inleiding
%%=============================================================================

\chapter{\IfLanguageName{dutch}{Inleiding}{Introduction}}%
\label{ch:inleiding}

\begin{comment}
De inleiding moet de lezer net genoeg informatie verschaffen om het onderwerp te begrijpen en in te zien waarom de onderzoeksvraag de moeite waard is om te onderzoeken. In de inleiding ga je literatuurverwijzingen beperken, zodat de tekst vlot leesbaar blijft. Je kan de inleiding verder onderverdelen in secties als dit de tekst verduidelijkt. Zaken die aan bod kunnen komen in de inleiding~\autocite{Pollefliet2011}:

\begin{itemize}
  \item context, achtergrond => OK?
  \item afbakenen van het onderwerp => OK?
  \item verantwoording van het onderwerp, methodologie !!!
  \item probleemstelling !!!
  \item onderzoeksdoelstelling: !!! TODO !!!
  \item onderzoeksvraag: Is Microsoft Graph op dit moment klaar om de beheertaken die mogelijk waren met Azure AD graph over te nemen?
  \item \ldots 
\end{itemize}
\end{comment}

% Taalcheck OK

Met de opkomst van digitale transformatie en de groeiende behoefte aan naadloze samenwerking en efficiënt gegevensbeheer, heeft Microsoft Graph zich gevestigd als een essentieel hulpmiddel van Microsoft binnen het moderne zakelijke landschap. Als een krachtige \Ac{API} biedt Microsoft Graph systeembeheerders de mogelijkheid om een geheel nieuwe dimensie van verbondenheid en intelligentie te creëren in systeembeheertaken. Met zijn diepgaande integratie met verschillende Microsoft-producten en -diensten ontsluit Microsoft Graph de waarde van gegevens, netwerken en activiteiten. Deze brede set aan mogelijkheden is ontstaan om zakelijke processen te verbeteren, het beheer van systemen en apparaten te stroomlijnen en gebruikers van de technologie een betere ervaring te bieden. \\ 

De evolutie van Microsoft administration heeft een aanzienlijke impact gehad op de manier waarop organisaties hun \ac{IT}-infrastructuur beheren. Microsoft administration heeft zich ontwikkeld van traditionele, handmatige beheerprocessen naar geavanceerde tools en technologieën die het beheer van \ac{AD} vereenvoudigen en optimaliseren. \ac{AD} staat dan ook in voor het verzamelen van allerlei informatie zoals gebruikers, computers, groepen, organisaties, beveiliging en certificaten. Met de opkomst van complexe netwerken, groeiende gebruikersbases en de behoefte aan strengere beveiligingsmaatregelen, hebben systeembeheerders steeds meer behoefte aan geautomatiseerde en schaalbare oplossingen om \ac{AD} effectief te beheren. \\

\section{Probleemstelling}
\label{sec:probleemstelling}

\begin{comment}
Uit je probleemstelling moet duidelijk zijn dat je onderzoek een meerwaarde heeft voor een concrete doelgroep. De doelgroep moet goed gedefinieerd en afgelijnd zijn. Doelgroepen als ``bedrijven,'' ``KMO's'', systeembeheerders, enz.~zijn nog te vaag. Als je een lijstje kan maken van de personen/organisaties die een meerwaarde zullen vinden in deze bachelorproef (dit is eigenlijk je steekproefkader), dan is dat een indicatie dat de doelgroep goed gedefinieerd is. Dit kan een enkel bedrijf zijn of zelfs één persoon (je co-promotor/opdrachtgever).
\end{comment}

% Taalcheck OK

De visie van Microsoft om een centraal eindpunt te ontwikkelen voor verschillende Microsoft-entiteiten brengt een verschuiving met zich mee. Allerlei producten en diensten van Microsoft, die vanuit een bijhorende \ac{API} worden gestuurd, bundelen zich in de nieuwe Microsoft Graph. Dit leidt tot de uitfasering van reeds bestaande Microsoft-technologieën waaronder \ac{ADAL} en Azure \Ac{AD} Graph \autocite{Sahay2022}. \\

Het \ac{IT}-bedrijf Easi, dat zich focust op \ac{IT}-oplossingen voor ondernemingen, biedt een Office 365-audit aan hun klanten. Met behulp van een PowerShell-script haalt de audit informatie uit een klantenomgeving om een advies naar een klant te onderbouwen. Dit script maakt gebruik van Azure \ac{AD} PowerShell-modules. Deze modules zitten mee in de aangekondigde uitfasering die op 30 juni 2023 zal plaatsvinden. Door dit scenario is Easi geïmpacteerd door deze verandering en wordt het gedwongen om gebruik te maken van de nieuwe Microsoft Graph. \\

Easi heeft sinds de aankondiging van Microsoft nog niet de mogelijkheid gehad om deze nieuwe kennis binnen te halen, waardoor dit PowerShell-script onveranderd is gebleven. Toch heeft Easi nood aan zekerheid en wilt het weten of dit script voor de uitfaseringsdatum kan omgezet worden met Microsoft Graph. Bovendien wilt het bedrijf weten wat de verschillen en gelijkenissen zijn tussen Azure \ac{AD} Graph en Microsoft Graph voor verder gebruik. \\

Door de casus biedt dit de mogelijkheid om onderzoek te doen naar Microsoft Graph. Om via dit onderzoek het probleem van Easi op te lossen, wordt er zowel een vergelijkende studie als Proof-of-Concept uitgevoerd. De vergelijkende studie vergelijkt de uitfaserende Azure \Ac{AD} Graph met de vervangende Microsoft Graph. De Proof-of-Concept bevat een praktische omvorming van het PowerShell-script van de Azure \Ac{AD} PowerShell naar Microsoft Graph PowerShell. \\

Naast Easi, waar de voornaamste focus op ligt als doelgroep, kan dit onderzoek ook een meerwaarde betekenen voor Microsoft-systeembeheerders die meer duiding willen over Microsoft Graph. Dit komt door een gebrek aan academische studies over Microsoft Graph, iets waar dit onderzoek verandering in brengt. 

\section{Onderzoeksvraag}
\label{sec:onderzoeksvraag}

\begin{comment}
Wees zo concreet mogelijk bij het formuleren van je onderzoeksvraag. Een onderzoeksvraag is trouwens iets waar nog niemand op dit moment een antwoord heeft (voor zover je kan nagaan). Het opzoeken van bestaande informatie (bv. ``welke tools bestaan er voor deze toepassing?'') is dus geen onderzoeksvraag. Je kan de onderzoeksvraag verder specifiëren in deelvragen. Bv.~als je onderzoek gaat over performantiemetingen, dan 
\end{comment}

\subsection{Hoofdonderzoeksvraag}

Uit de reeds besproken casus wordt volgende hoofdonderzoeksvraag gesteld: “Is Microsoft Graph op dit moment klaar om de beheertaken die mogelijk waren met Azure \ac{AD} Graph over te nemen?” De hoofdonderzoeksvraag zal dit onderzoek een onderbouwde conclusie geven dat te vinden is in Hoofdstuk \ref{ch:conclusie}.

\subsection{Deelonderzoeksvraag}

De hoofdonderzoeksvraag wordt ondersteund door volgende deelonderzoeksvragen. 

\begin{itemize}
   \item Wat zijn de mogelijke \Ac{HTTP}-verzoeken?
   \item Uit welke onderdelen bestaat het eindpunt?
   \item Zijn er verschillen aanwezig in requests en reponses tijdens het uitvoeren van een query?
   \item Welke PowerShell-versies zijn er?
   \item Welke data-objecten kunnen worden aangesproken?
   \item Bevat de technologie bepaalde afhankelijkheden?
   \item Hoe wordt er toegang verschaft tot de technologie?
   \item Zijn de toegangsmogelijkheden veilig?
   \item Wordt er gebruikgemaakt van rechten om misbruik tegen te gaan?
   \item Kan het PowerShell-script volledig omgezet worden van Azure \ac{AD} PowerShell naar Microsoft Graph PowerShell?
\end{itemize}

Door het gebruik van deze deelvragen kan de hoofdonderzoeksvraag beter benaderd worden en wordt het antwoord op de hoofdvraag grondiger onderbouwd. De deelvragen worden in de loop van de vergelijking en de praktische uitwerking beantwoord. Er volgt daarbij ook een samenvattend antwoord op de deelvragen in Hoofdstuk \ref{ch:conclusie}, waar er een conclusie wordt gevormd.

\section{Onderzoeksdoelstelling}%
\label{sec:onderzoeksdoelstelling}

\begin{comment}
Wat is het beoogde resultaat van je bachelorproef? Wat zijn de criteria voor succes? Beschrijf die zo concreet mogelijk. Gaat het bv.\ om een proof-of-concept, een prototype, een verslag met aanbevelingen, een vergelijkende studie, enz.
\end{comment}

% Taal OK

De hoofddoelstelling van dit onderzoek is om het audit-script van Easi zo volledig mogelijk om te vormen met het gebruik van Microsoft Graph PowerShell-modules. Deze doelstelling is volbracht als alle twintig, of zoveel mogelijk, onderdelen van het script zijn omgevormd. \\

Daarnaast heeft deze studie als doelstelling een concreet antwoord te geven op de hoofdonderzoeksvraag in de vorm van een aanbeveling. Deze aanbeveling is gunstig wanneer alle criteria uit de vergelijking en onderdelen van de Proof-of-Concept kunnen gebruikt worden om de aanbeveling te onderbouwen. \\

Het laatste doel is om Microsoft-systeembeheerders of geïnteresseerde bedrijven, die dit onderzoek lezen, bij te leren over hoe Microsoft Graph werkt. Met een nadruk op hoe de technologie moet en kan gebruikt worden met de huidige stand van zaken. 

\section{Opzet van deze bachelorproef}%
\label{sec:opzet-bachelorproef}

% Taal OK

De bachelorproef heeft volgende opbouw. \\

De stand van zaken betreft Microsoft administration, Azure \ac{AD} en Microsoft Graph worden weergegeven in Hoofdstuk \ref{ch:stand-van-zaken}. Dit hoofdstuk is een literaire verdieping van de drie onderwerpen. \\

Daarna volgt een toelichting van de methodologie in Hoofdstuk \ref{ch:methodologie}. In dit hoofdstuk worden de verdere fases van het onderzoek gekaderd en hoe deze worden uitgevoerd. \\

Vervolgens volgt de vergelijkende studie in Hoofdstuk \ref{ch:vergelijking} tussen Azure \Ac{AD} Graph en Microsoft Graph met hun bijhorende PowerShell-modules. In dit hoofdstuk worden beide technologieën tegenover elkaar gezet op basis van de vooropgestelde criteria. \\

Nadien volgt de Proof-of-Concept of praktische uitwerking in Hoofdstuk \ref{ch:poc}. Dit hoofdstuk is toegewijd aan het opzetten van de testomgeving en het omzetten van het audit-script met Microsoft Graph PowerShell-modules. \\

Achteraf volgt de conclusie in Hoofdstuk \ref{ch:conclusie}. De conclusie bevat een bespreking van de resultaten op basis van de drie vooropgestelde onderzoeksdoelstellingen met een antwoord op de hoofdonderzoeksvraag. Daarbij wordt ook een aanzet gegeven voor toekomstig onderzoek over dit onderwerp. \\

Ten slotte zijn de bijlagen in hoofdstuk \ref{ch:voorstel}, \ref{ch:HTTP-queries} en \ref{ch:PowerShell-scripts} te vinden die een ondersteunende functie hebben in dit onderzoek. Deze bijlagen bevatten het onderzoeksvoorstel, de uitgevoerde \Ac{HTTP}-queries en uitgewerkte Microsoft Graph PowerShell-scripts.

\begin{comment}
% Het is gebruikelijk aan het einde van de inleiding een overzicht te
% geven van de opbouw van de rest van de tekst. Deze sectie bevat al een aanzet
% die je kan aanvullen/aanpassen in functie van je eigen tekst.

De rest van deze bachelorproef is als volgt opgebouwd:

In Hoofdstuk~\ref{ch:stand-van-zaken} wordt een overzicht gegeven van de stand van zaken binnen het onderzoeksdomein, op basis van een literatuurstudie.

In Hoofdstuk~\ref{ch:methodologie} wordt de methodologie toegelicht en worden de gebruikte onderzoekstechnieken besproken om een antwoord te kunnen formuleren op de onderzoeksvragen.

% TODO: Vul hier aan voor je eigen hoofstukken, één of twee zinnen per hoofdstuk

In Hoofdstuk~\ref{ch:conclusie}, tenslotte, wordt de conclusie gegeven en een antwoord geformuleerd op de onderzoeksvragen. Daarbij wordt ook een aanzet gegeven voor toekomstig onderzoek binnen dit domein.
\end{comment}

\chapter{\IfLanguageName{dutch}{Stand van zaken}{State of the art}}%
\label{ch:stand-van-zaken}

% Tip: Begin elk hoofdstuk met een paragraaf inleiding die beschrijft hoe
% dit hoofdstuk past binnen het geheel van de bachelorproef. Geef in het
% bijzonder aan wat de link is met het vorige en volgende hoofdstuk.

% Pas na deze inleidende paragraaf komt de eerste sectiehoofding.

\begin{comment}

Dit hoofdstuk bevat je literatuurstudie. De inhoud gaat verder op de inleiding, maar zal het onderwerp van de bachelorproef *diepgaand* uitspitten. De bedoeling is dat de lezer na lezing van dit hoofdstuk helemaal op de hoogte is van de huidige stand van zaken (state-of-the-art) in het onderzoeksdomein. Iemand die niet vertrouwd is met het onderwerp, weet nu voldoende om de rest van het verhaal te kunnen volgen, zonder dat die er nog andere informatie moet over opzoeken \autocite{Pollefliet2011}.

Je verwijst bij elke bewering die je doet, vakterm die je introduceert, enz.\ naar je bronnen. In \LaTeX{} kan dat met het commando \texttt{$\backslash${textcite\{\}}} of \texttt{$\backslash${autocite\{\}}}. Als argument van het commando geef je de ``sleutel'' van een ``record'' in een bibliografische databank in het Bib\LaTeX{}-formaat (een tekstbestand). Als je expliciet naar de auteur verwijst in de zin, gebruik je \texttt{$\backslash${}textcite\{\}}.
Soms wil je de auteur niet expliciet vernoemen, dan gebruik je \texttt{$\backslash${}autocite\{\}}. In de volgende paragraaf een voorbeeld van elk.

\textcite{Knuth1998} schreef een van de standaardwerken over sorteer- en zoekalgoritmen. Experten zijn het erover eens dat cloud computing een interessante opportuniteit vormen, zowel voor gebruikers als voor dienstverleners op vlak van informatietechnologie~\autocite{Creeger2009}.

\end{comment}

In dit hoofdstuk wordt de state-of-the-art van dit onderzoek diepgaand besproken. Dit hoofdstuk biedt alle nodige informatie om de vergelijking tussen het oude en het nieuwe te vormen. Daarnaast is dit hoofdstuk de basislaag om dit onderzoek praktisch uit te werken. De titel bestaat uit drie onderdelen, namelijk: Microsoft administration, de module van Azure Active Directory en Microsoft Graph met \ac{API}. Als eerste wordt het begrip Microsoft administration besproken met bijhorende geschiedenis. Microsoft administration is de overkoepelende term waarin beide technologieën zich afspelen. Bij het tweede onderdeel wordt er dieper gekeken in dit domein. Er wordt meer duiding gegeven over Azure Active Directory met bijhorende componenten waaronder de PowerShell-module. Als laatste wordt Microsoft Graph aangeraakt met een focus op de \ac{API}. De interne werking en mogelijkheden worden hierbij gedetailleerd besproken.


\section{Wat is Microsoft administration?}

% Taalcheck OK

Microsoft administration bestaat uit twee kernwoorden die bekend zijn binnen de \ac{IT}-wereld. Microsoft staat voor het Amerikaanse technologiebedrijf en de producten die het aanbiedt \autocite{Warner2019}. Het woord administration, oftewel administratie in het Nederlands, staat voor het beheren van iets \autocite{Burgess2003}. Vanuit het woord administration volgde het woord administrator, dat duidt op een persoon die een of meerdere instanties beheert. Kortom, Microsoft administration staat voor het beheren van Microsoft-instanties en -producten (bv. Windows en Azure). 

\subsection{Administration in de jaren tachtig en negentig}

% Taalcheck OK

Het beheren van systemen kan omvat worden in enkele taken, hieronder volgt een lijst van frequente taken tussen het jaar 1980 en 2000 \autocite{Frisch2002}.

\begin{itemize}
    \item Toevoegen van nieuwe gebruikers en toestellen
    \item Maken en beheren van back-ups
    \item Bestanden en andere data recupereren
    \item Gebruikers assisteren in dagelijkse taken en problemen
    \item Monitoren van systemen
    \item De veiligheid van de systemen garanderen
    \item Het beheren en installeren van updates
    \item Het automatiseren van taken
\end{itemize}

Deze taken zijn vandaag de dag nog steeds herkenbaar als dagdagelijkse taken van een systeembeheerder. 

\subsection{Microsoft administration via Windows}

% Taalcheck OK

Sinds de opkomst van Windows-systemen zoals Windows 2000 server, zijn er mogelijkheden om administratieve taken uit te voeren binnen netwerkinfrastructuren \autocite{Tulloch2001}. \\

Binnen Windows 2000 server zijn er administratieve tools zoals Microsoft Management Console, Event Viewer en Active Directory Domains and Trusts-instellingen beschikbaar. Deze tools dienen om de taken van een administrator te vergemakkelijken, door een overzicht te brengen van alle data in dat specifieke onderdeel \autocite{Sibisi2022}. Door het gebruik van deze administratieve tools kan een administrator Microsoft-entiteiten waaronder Windows-toestellen beheren om zijn taken mee uit te voeren. 

\subsection{De evolutie van systemen en zijn administratie}

% Taalcheck OK

In het begin van de eenentwintigste eeuw werden systemen en servers zoals Windows 2000 server met een focus op \ac{On-prem} onderhouden \autocite{Microsoft2022a}. \ac{On-prem} betekent dat software en hardware, zoals computers en servers, op locaties staan dat eigendom is van het bedrijf en lokaal worden toegepast \autocite{Gastermann2015}. Hierbij wordt de systeemadministratie door systeembeheerders lokaal aangepakt. Ter illustratie, in dit scenario focust de automatisatie via PowerShell zich op de lokale entiteiten binnen het bedrijf. Databanken, mailservers, \ac{DNS}-servers en andere instanties worden lokaal aangesproken en geautomatiseerd indien nodig. \\

Rond het jaar 2006 evolueerde de On-premises-aanpak naar een cloud-aanpak \autocite{Hayes2008}. \ac{EC2} van Amazon is een van de grondleggers binnen de cloudservices \autocite{Qian2009}. Deze inventie van Amazon heeft sindsdien invloed op de huidige marktleiderspositie van AWS in cloud computing \autocite{Vailshery2022}. Op de tweede plaats bevindt zich de cloudservice van Microsoft, genaamd Azure.



\subsection{De impact van cloud computing op systeem administratie}

% Taalcheck OK

Cloud computing heeft een brede betekenis. Het is in feite een technologiemodel dat gebruikt wordt om middelen en diensten beschikbaar te stellen op het internet, of de cloud. \autocite{Haag2009} \\

De migratie van \ac{On-prem} naar de cloud is geen toeval. Eenenveertig procent van de bedrijven uit de \ac{EU} maakt gebruik van cloud computing in 2021 \autocite{EU2021}. \\

Het gebruik van cloud computing en cloudservices heeft vele voordelen. De volgende lijst bevat enkele voordelen van cloud computing. De lijst is samengesteld uit onderzoek van \textcite{Aljabre2012}, \textcite{Rittinghouse2016}.

\begin{itemize}
    \item Minder onderhouds-, implementatie- en infrastructuurkosten.
    \item Goedkopere computers per gebruiker (via virtuele machines).
    \item Verhoogde mobiliteitskansen voor de werkkrachten.
    \item Nieuwe en flexibelere infrastructuren met verhoogde schaalbaarheid.
    \item Vergroening van data centers.
    \item Verhoogde beschikbaarheid van applicaties.
    \item Mogelijkheid om gebruikers te doen samenwerken in documenten en projecten.
\end{itemize}



\subsection{Verschuiving van On-premises naar de cloud}

% Taalcheck

De evolutie van \ac{On-prem} naar de cloud heeft invloed op de huidige stand van zaken binnen Microsoft administration. Microsoft administratie staat in voor het beheren van Microsoft-entiteiten. Door een verschuiving van \ac{On-prem} naar cloudomgevingen, wordt de nadruk stilaan gelegd op het beheren van bepaalde entiteiten via cloudtoepassingen. \\

Deze nadruk valt op wanneer de productenlijst van Microsoft Azure wordt geraadpleegd \autocite{Microsoft2023b}. Dit is een lijst die steeds verder wordt uitgebreid met nieuwe technologieën. \\

Een praktisch voorbeeld van een entiteit, of een groep van entiteiten, dat via de cloud kan beheerd worden is Azure Active Directory \autocite{Microsoft2023c}. Azure Active Directory kan gebruikt worden als alternatief voor Active Directory voor bepaalde onderdelen. Active Directory is kortweg een opslagplaats die een focus heeft op On-premises-instanties. Active Directory en Azure Active Directory worden in het volgende onderdeel verder besproken.

% ---
% Volgende onderdeel
% ---

\section{Wat is Azure Active Directory?}

\ac{AD} is een centrale en gemeenschappelijke opslagplaats voor informatie geïntroduceerd door Microsoft \autocite{Allen2003}. De eerste versie van \ac{AD} is gemaakt voor de Windows 2000 server-editie. Deze opslagplaats bevat allerlei informatie binnen een netwerk, zoals gebruikers, groepen, computers, applicaties, bestanden en printers. Deze informatie kan worden opgevraagd en beheerd. \\

Azure \ac{AD} is een modernere aanpak van Active Directory binnen de cloud dat ontstaan is in 2008 \autocite{Chappell2008}. Azure \ac{AD} is een gecentraliseerd beheerplatform van Microsoft voor gebruikers en apparaten in netwerken die een verbinding hebben met de Azure-clouddienst \autocite{Mayank2019}. Middelen, waaronder gebruikers en apparaten, kunnen vanuit Azure \ac{AD} beheerd worden met bijhorende netwerkauthenticatie. Het dient als een centraal punt van informatie, waarbij details over alle middelen in het netwerk worden opgeslagen.

\subsection{Azure Active Directory PowerShell-modules} 

Azure \ac{AD} PowerShell voor Graph, kortweg Azure \ac{AD} PowerShell, is een module binnen PowerShell die gebruikt kan worden om Azure \ac{AD} te beheren \autocite{Microsoft2023}. PowerShell is een oplossing van Microsoft voor taakautomatisering via een \ac{CLI} \autocite{Microsoft2022}. \\

PowerShell staat gekend voor zijn breed scala aan informatie dat het kan verkrijgen. Dit breed scala gaat van systemen, servers, randapparatuur, mobiele apparaten tot gegevensgestuurde toepassingen zoals Active Directory \autocite{Hosmer2019}. Vandaag de dag ondersteunt PowerShell meer dan 11.350 unieke modules en scripts via de PowerShell Gallary, waaronder de Azure \ac{AD} PowerShell module \autocite{Microsoft2023a}. \\

Door het gebruik van Azure \ac{AD} in combinatie met PowerShell, wordt er gebruikgemaakt van het beste van de twee werelden. Een gecentraliseerd beheerplatform automatisch doen werken brengt bijkomende voordelen voor de maker. Uit onderzoek van \textcite{Breton2003} zijn enkele voordelen van automatisatie minder stress, tijdbesparing en een lagere kans op menselijke fouten. \\

Naast de Azure \ac{AD} PowerShell-module, is er de MSOnline PowerShell-module \autocite{Prigent2019}. MSOnline is de voorganger van de AzureAD-module. MSOnline is de eerste versie, in vergelijking met AzureAD dat de tweede versie is. MSOnline heeft dezelfde functie als AzureAD en biedt de mogelijkheid aan om via PowerShell Azure \ac{AD} te kunnen aanspreken.

\subsubsection{Ondersteunende data-objecten}

PowerShell heeft verschillende soorten commando's voor het toevoegen, verwijderen of lezen van iets. Wanneer een commando de functie heeft om iets op te vragen, dan kan dit bijvoorbeeld een “Get”-commando zijn zoals “Get-MgUser”. \\

Binnen Azure AD PowerShell zijn er twaalf soorten commando's, namelijk:

\begin{itemize}
    \item Add: Toevoegen.
    \item Confirm: Bevestigen.
    \item Connect: Verbinden.
    \item Disconnect: Ontkoppelen.
    \item Get: Informatie ophalen.
    \item New: Aanmaken.
    \item Remove: Verwijderen.
    \item Restore: Herstellen.
    \item Revoke: Intrekken.
    \item Select: Informatie ophalen.
    \item Set: Wijzigen.
    \item Update: Bijwerken.
\end{itemize}

Azure AD PowerShell ondersteunt volgende data-objecten met een specifiek aantal uitvoerbare commando's \autocite{Microsoft2023i}. Per data-object zijn er een of meerdere soorten PowerShell-commando's, in de tabel wordt alleen het aantal commando's weergegeven. Een overzicht van alle ondersteunde data-objecten met het aantal commando's wordt weergeven in Tabel \ref{AADT}. 

\begin{table}
    \small
    \centering
    \begin{tabular}{ |c|c| } 
        \hline
        \textbf{Data-object} & \textbf{Aantal commando's} \\
        \hline
        Administrative Units & 9 \\ 
        Application Proxy Application Management & 8 \\ 
        Application Proxy Connector Management & 9 \\
        Applications & 20 \\ 
        AzureAD & 49 \\ 
        Certificate Authorities & 4 \\ 
        Connect to directory & 2 \\ 
        Contacts & 8 \\ 
        Contracts & 1 \\ 
        Deleted Objects & 1 \\ 
        Devices & 11 \\    
        Directory & 3 \\
        Directory Objects & 1 \\ 
        Directory Roles & 13 \\ 
        Domains & 8 \\ 
        Extension Properties & 1 \\ 
        Groups & 26 \\ 
        MSOnline & 97 \\
        OAuth2 & 2 \\ 
        Policies & 2 \\ 
        Service Principals & 22 \\ 
        Users & 30 \\ 
        \hline
    \end{tabular}
    \caption[Tabel Azure AD en MSonline data-objecten]{Tabel met overzicht van alle ondersteunende data-objecten van Azure \ac{AD} en MSOnline met bijhorende commando's binnen Azure \ac{AD} PowerShell.}
    \label{AADT}
\end{table}

\subsection{Wat is Azure Active Directory Graph?}

% TODO: uitleggen wat MSOnline is ??? (niet hier maar ergens anders)

De communicatie tussen Azure \ac{AD} en de Azure \ac{AD} PowerShell-module gebeurd via een \ac{API}. Deze \ac{API} is een \ac{REST} \ac{API} die Graph wordt genoemd. \\ 

De naam Graph is afgeleid uit het wiskundige figuur van een graaf. Een graaf is een verzameling van punten die wel of niet met elkaar zijn verbonden \autocite{Denaux2022}. Een voorbeeld van een wiskundige graaf is te zien op Figuur \ref{mga}. \\

\begin{figure}[!b]
    \includegraphics[width=\textwidth]{MathGraphExample.jpg}
    \caption[Voorbeeld wiskundige graaf]{Voorbeeld van een graaf in de wiskunde uit het boek Graph Theory van \textcite{Diestel2010}.}
    \label{mga}
\end{figure}

Graph van Microsoft heeft een soortgelijke betekenis als dat van een wiskundige graaf. Graph staat in voor de verbindingen tussen de entiteiten dat het ondersteunt, in dit geval Microsoft-entiteiten \autocite{Kokkarinen2022}. Een logische interpretatie van Graph wordt weergegeven op Figuur \ref{gms}. \\

\begin{figure}[!b]
    \includegraphics[width=\textwidth]{GraphMicrosoft.png}
    \caption[Voorbeeld Azure AD Graph]{Voorstelling van Azure \Ac{AD} Graph door \textcite{Microsoft2017}.}
    \label{gms}
\end{figure}

\subsubsection{Achterliggende werking van Azure Active Directory Graph API}

% TODO: Bronnen (zie word onder Azure AD Graph API)

% TODO: Schrijf verder vanuit de bron van weareminky, dit is de makkelijkste manier

De Azure \ac{AD} Graph \ac{API} is een OData 3.0-compatibele dienst om objecten zoals gebruikers, groepen en contactpersonen in een tenant te lezen en te wijzigen \autocite{Microsoft2016}. \\

Een tenant kan vergeleken worden met een appartementencomplex \autocite{Saxton2015}. Binnen dit complex zijn meerdere appartementen. In elk appartement heb je bijvoorbeeld een badkamer, slaapkamer, living en balkon. Deze uitleg wordt gevisualiseerd op Figuur \ref{rlt}. \\

\begin{figure}[!h]
    \includegraphics[width=50mm, scale=0.5]{RealTenant.png}
    \caption[Voorbeeld werkelijke tenant]{Voorbeeld van een tenant in het echte leven uit de blog van \textcite{Saxton2015}.}
    \label{rlt}
\end{figure}

In dit scenario met Microsoft is (in \ac{IT}-termen) het complex de Microsoft Office 365-datacenters. Het appartement stelt de tenant of de organisatie in het algemeen voor. Deze tenant-omgeving bevat bijvoorbeeld abonnementen, gebruikers, domeinen en groepen. De tenant wordt gezien als een huurder van de omgeving. Dit concept wordt gevisualiseerd op Figuur \ref{mst}. \\

\begin{figure}[!h]
    \includegraphics[width=50mm, scale=0.5]{MSTenant.png}
    \caption[Voorbeeld Microsoft tenant]{Voorbeeld van een tenant binnen Microsoft uit de blog van \textcite{Saxton2015}.}
    \label{mst}
\end{figure}

Het OData-protocol dient voor interacties met gegevens via \ac{REST}-webdiensten. Dit protocol biedt een uniforme manier om gegevens en gegevensmodellen te beschrijven \autocite{OData2023}. \\

Door het gebruik van \ac{REST}-eindpunten kunnen \ac{HTTP}-verzoeken verstuurd worden om bewerkingen uit te voeren met de service. De Azure \ac{AD} Graph \ac{API} heeft een eigen demo-omgeving genaamd Azure \ac{AD} Graph Explorer, waarin bepaalde functies en bewerkingen worden getest \autocite{Microsoft}. \\

Het proces van Graph gaat van start wanneer een applicatie een verzoek doet. Dit verzoek moet een token bevatten. Deze token wordt gegeven door Azure \ac{AD} en bewijst dat de gebruiker de juiste machtigingen heeft om de gevraagde data te raadplegen.



\subsubsection{Authenticatie via libraries}

Vooraleer de autorisatie kan plaatsvinden, moet er eerst worden geauthenticeerd. Dit gebeurd via een tool genaamd \ac{ADAL} of de modernere \ac{MSAL} van \textcite{Microsoft2022d}. \\

\ac{ADAL} en \ac{MSAL} worden gebruikt voor verificatie- en autorisatiefunctionaliteiten \autocite{Ooms2022}. Beide technologieën dienen om te kunnen authenticeren met de Graph \ac{API}. \\ 

Er zijn twee methodes om te verbinden met de Graph \ac{API}. \\

De eerste is door middel van een gedelegeerde of interactieve manier. Er wordt gebruikgemaakt van een prompt waarbij de gebruikersnaam (of mailadres) en het wachtwoord wordt gevraagd. \\

Bovendien kan er gevraagd worden om een \ac{2FA}- of \ac{MFA}-methode toe te passen. Dit gaat gepaard met het account waarop de gebruiker wilt inloggen, indien dit geactiveerd is. \\

Het gebruik van \ac{2FA} of \ac{MFA} is veiliger dan alleen het gebruik van een naam en wachtwoord, bevestigt het onderzoek van \textcite{Gunson2011} en \textcite{Banyal2013}. Het wordt sterk aangeraden om minstens \ac{2FA} toe te passen bij het gebruik van een gedelegeerde methode. \\

De tweede methode is door een niet-interactieve of geprogrammeerde manier. Deze methode maakt gebruik van secrets die aan de registratie van een applicatie is gekoppeld. Deze registratie gebeurd in Azure. 



\subsubsection{Autorisatie door middel van tokens}

Azure Active Directory-tokens bevatten informatie over de applicatie, de gebruiker, authenticatie en de rechten die een applicatie mag uitvoeren op de bijhorende directory \autocite{Microsoft2015}. \\

Deze tokens bevinden zich in de autorisatieheader van een verzoek. Een afgekort voorbeeld hiervan is te zien op Figuur \ref{ahtoken}. \\

\begin{figure}[h]
    \begin{verbatim}Authorization: Bearer eyJ0eX ... FWSXfwtQ
    \end{verbatim}    
    \caption[Voorbeeld Azure AD-token]{Afgekort voorbeeld van een Azure \Ac{AD}-token binnen een autorisatieheader.}
    \label{ahtoken}
\end{figure}

Vervolgens voert de \ac{API} de nodige autorisaties uit via permissie scopes. Dit zijn OAuth 2.0 permissie scopes die aanwezig zijn in het Azure \Ac{AD}-token. OAuth 2.0 is een standaardprotocol voor autorisatie \autocite{OAuth}. \\

De permissie scopes worden gebruikt om te controleren of een gebruiker toegang heeft tot een bepaalde map of locatie. Een ontwikkelaar moet in dit scenario de juiste permissie scopes gebruiken om over de vereiste rechten te beschikken voor een actie. \\

Wanneer er wordt aangemeld, krijgt de gebruiker de mogelijkheid om toestemming te geven of de applicatie in kwestie de mapgegevens van de gebruiker mag benutten. Tijdens het toestaan worden de permissie scopes meegegeven die de ontwikkelaar heeft ingesteld. Een opsomming van de mogelijk permissie scopes binnen de Azure \ac{AD} Graph \ac{API} bevindt zich in Tabel \ref{psaad}. 

\begin{table}[ht]
    \centering
    \begin{adjustbox}{width=1\textwidth}
    \begin{tabular}{ |c|c|c| }
        \hline
        \textbf{Scope} & \textbf{Beschrijving} & \textbf{Type} \\
        \hline
        User.Read & Rechten om aan te melden en het gebruikersprofiel te lezen. & Delegated \\
        User.ReadBasic.All & Rechten om de basisprofielen van alle gebruikers te lezen. & Delegated \\
        User.Read.All & Rechten om het volledige profiel van alle gebruikers te lezen. & Delegated \\
        Group.Read.All & Rechten om alle groepen te lezen. & Delegated \\
        Group.ReadWrite.All & Rechten om alle groepen te lezen en te bewerken. & Delegated \\
        Device.ReadWrite.All & Rechten om alle apparaten te lezen en te bewerken. & App-only \\
        Directory.Read.All & Rechten om alle mapgegevens te lezen. & App-only, Delegated \\
        Directory.ReadWrite.All & Rechten om alle mapgegevens te lezen en te bewerken. & App-only, Delegated \\
        Directory.AccessAsUser.All & Rechten om in de map toe treden als de aangemelde gebruiker. & Delegated \\
        \hline
    \end{tabular}
    \end{adjustbox}
    \caption[Tabel Azure \ac{AD} Graph Permission scopes]{Tabel met de beschikbare permissie scopes, bijhorende beschrijving en scope-type door \textcite{Microsoft2016a}.}
    \label{psaad}
\end{table}

Het gebruik van permissie scopes komt overeen met een gekend veiligheidsprincipe binnen de \ac{IT}. “Principle of Least Privilege”, of “Beginsel van de minste voorrechten” in het Nederlands, staat voor het toekennen van een minimum aan rechten \autocite{Saltzer1975}. 



\subsubsection{Eindpunt adressering}

Om taken of functies uit te voeren met de Graph \ac{API}, wordt er gebruikgemaakt van \ac{HTTP}-verzoeken. Deze verzoeken maken gebruik van een bepaalde methode. Een opsomming van de mogelijk methodes wordt hieronder weergegeven met bijhorende betekenis, gebaseerd op het onderzoek van \textcite{Fielding1999}, \textcite{Dusseault2010}.

\begin{itemize}
    \item GET: Geeft data weer van de server.
    \item POST: Verstuurt data naar de server om nieuwe data aan te maken.
    \item PATCH: Verstuurt data naar de server om gedeeltelijk de bron bij te werken.
    \item PUT: Verstuurt data naar de server om de volledige bron bij te werken.
    \item DELETE: Verwijdert data van de server.
\end{itemize}

Deze verzoeken zijn gericht op een eindpunt van een dienst, een resource, een verzameling van resources of andere entiteiten die de \ac{API} ondersteunt. De eindpunten worden genoteerd als een \ac{URL}. Het gebruikelijk formaat van zo'n eindpunt wordt voorgesteld op Figuur \ref{bfe}. \\

\begin{figure}[h]
    \footnotesize\begin{verbatim}https://graph.windows.net/{tenant_id}/{resource}?{version}&query-parameters
    \end{verbatim}    
    \caption[Basis formaat Graph API-eindpunt]{Het basis formaat van een Azure \ac{AD} Graph \ac{API}-eindpunt uit documentatie van \textcite{Microsoft2023o}.}
    \label{bfe}
\end{figure}

De reeds voorgestelde \ac{URL} bestaat uit vier componenten.

\begin{itemize}
    \item Service root: Het aanspreekpunt voor alle Graph \ac{API}-verzoeken. Voor Azure \ac{AD} Graph is dit “https://graph.windows.net”.
    \item Tenant Identifier: De identiteit van de tenant waar het verzoek naar gericht is.
    \item Resource path: Het pad van de bron waar het verzoek naar gericht is.
    \item Graph \ac{API} version: De versie van de \ac{API} waar het verzoek naar gericht is.
\end{itemize}

Een praktisch voorbeeld met betrekking tot het oproepen van gebruikersgegevens, wordt weergegeven op Figuur \ref{pfe}. \\

\begin{figure}[h]
    \footnotesize\begin{verbatim}https://graph.windows.net/contoso.com/users/john@contoso.com/
$links/manager?api-version=1.6
    \end{verbatim}    
    \caption[Voorbeeld Graph API-eindpunt]{Praktisch voorbeeld van een Graph \ac{API}-eindpunt, gericht op de manager eigenschap van “john@contoso.com”.}
    \label{pfe}
\end{figure}



\subsubsection{OData Query Parameters}

Zoals reeds vermeld maakt de Graph \ac{API} gebruik van OData. Het gebruik van OData-queryparameters zorgt ervoor dat een ingelezen verzameling van bronnen kunnen gefilterd, gesorteerd en gepagineerd worden. \\

De Graph \ac{API} biedt ondersteuning aan voor de volgende parameters met bijhorende betekenis. De betekenissen staan gedefinieerd in studies van \textcite{Liang2016}, \textcite{Wojcieszyn2014}. 

\begin{itemize}
    \item \$batch: Indienen van \ac{HTTP} POST-verzoeken.
    \item \$expand: Opnemen van een of meerdere bronnen in het antwoord.
    \item \$filter: Filteren van beschikbare bronnen.
    \item \$orderby: Opgeven van ordening door de opgevraagde collectie.
    \item \$previous-page: Ophalen van vorige pagina met resultaten.
    \item \$top: Beperken van een teruggezonden opgevraagde verzameling.
    \item \$skiptoken: Overslaan van opgegeven aantal items.
\end{itemize}

Een toegepast voorbeeld van een OData-queryparameter is te vinden in Figuur \ref{odqp}. \\

\begin{figure}[h]
\footnotesize\begin{verbatim}GET https://graph.windows.net/contoso.com/directoryObjects?api-version=2013-04-05&
$filter=isof('Microsoft.WindowsAzure.ActiveDirectory.User')%20or%20isof
('Microsoft.WindowsAzure.ActiveDirectory.Group')%20or%20isof
('Microsoft.WindowsAzure.ActiveDirectory.Contact')&deltaLink=HTTP/1.1
\end{verbatim}    
\caption[Voorbeeld OData-queryparamter]{Toegepast voorbeeld van een OData-queryparameter op een \ac{HTTP} GET-request.}
\label{odqp}
\end{figure}

\subsubsection{Request en Response Headers}

% TODO: Bron zoeken voor onderstaande uitleg?

De Graph \ac{API} werkt met verzoeken en antwoorden. Een verzoek werkt met een aantal soorten headers en bodies. \\

Drie voorbeelden van voorkomende verzoekheaders zijn de volgende:

\begin{itemize}
    \item Authorization: Uitgegeven Azure \ac{AD}-token.
    \item Content-Type: Mediatype van de inhoud.
    \item Content-Length: Lengte van het verzoek (in bytes).
\end{itemize} 

Daarnaast bestaan er ook een aantal antwoordheaders. Hieronder worden een aantal mogelijke antwoordheaders meegegeven met hun betekenis.

\begin{itemize}
    \item Content-Type: Mediatype van de inhoud.
    \item Location: Antwoord op POST-verzoeken wanneer een nieuwe bron in de directory wordt aangemaakt.
    \item ocp-aad-diagnostics-server-name: Identifier voor de server die een bewerking uitvoert.
    \item ocp-aad-session-key: Sleutel die een sessie met de directorydienst identificeert.
\end{itemize}

Een voorbeeld van antwoordheaders binnen Azure \ac{AD} Graph wordt weergegeven bij Listing \ref{rhaad}. \\

\begin{listing}[h]
\begin{minted}
[
frame=lines,
framesep=2mm,
baselinestretch=1.2,
fontsize=\footnotesize,
linenos
]  
{json}
{
    "cache-control": "no-cache",
    "client-request-id": "140987df-c416-44b8-bf96-16d550256bad",
    "content-length": "12478",
    "content-type": 
        "application/json; 
        odata=minimalmetadata; 
        streaming=true; 
        charset=utf-8",
    "expires": "-1",
    "ocp-aad-session-key": "MzsMU-KCo5fDEUHgzYfj ... hf8ZctaauwL-EZo",
    "pragma": "no-cache",
    "request-id": "c7f02989-6a03-444a-abb2-e3c7993c1ded"
}
    \end{minted}
    \caption[Voorbeeld Response Headers Azure AD Graph]{Voorbeeld van Response Headers binnen Azure \ac{AD} Graph via Azure \Ac{AD} Graph Explorer.}
    \label{rhaad}
\end{listing}

\subsubsection{Request en Response Bodies}

Request bodies kunnen via \Ac{JSON}- of \ac{XML}-payloads verzonden worden voor POST-, PATCH- en PUT-verzoeken. Bovendien kunnen antwoorden (van een server) worden teruggestuurd via \ac{JSON} of \ac{XML}. \\

Het woord “payload” staat voor data die via een pakket of transmissie worden gedragen \autocite{Comer2006}. \\

Deze payloads kunnen in de request bodies gespecifieerd worden via de Content-Type verzoekheader en in antwoorden van Accept-verzoekheaders. \\

Een voorbeeld van een request, met bijhorende request en response body, wordt voorgesteld bij Listing \ref{hpr}, \ref{hreqb} en \ref{hresb}. \\

\begin{listing}[h]
\begin{verbatim}
POST https://graph.windows.net/myorganization/users?api-version
\end{verbatim}
\caption[Voorbeeld HTTP POST-request]{Voorbeeld van een \ac{HTTP} POST-request binnen Azure \ac{AD} Graph.}
\label{hpr}
\end{listing}

\begin{listing}[!b]
\begin{minted}
[
frame=lines,
framesep=2mm,
baselinestretch=1.2,
fontsize=\footnotesize,
linenos
]  
{json}
{
    "accountEnabled": true,
    "displayName": "Alex Wu",
    "mailNickname": "AlexW",
    "passwordProfile": {
        "password": "Test1234",
        "forceChangePasswordNextLogin": false
    },
    "userPrincipalName": "Alex@a830edad9050849NDA1.onmicrosoft.com"
}
\end{minted}
\caption[Voorbeeld Request Body Azure AD Graph]{Voorbeeld van een Request Body binnen Azure \ac{AD} Graph.}
\label{hreqb}
\end{listing}

\begin{listing}[!t]
\begin{minted}
[
frame=lines,
framesep=2mm,
baselinestretch=1.2,
fontsize=\footnotesize,
linenos
]  
{json}
{
    "odata.metadata": "https://graph.windows.net/myorganization\n
    /$metadata#directoryObjects/Microsoft.DirectoryServices.User/@Element",
    "odata.type": "Microsoft.DirectoryServices.User",
    "objectType": "User",
    "objectId": "84fba1e8-b942-47c9-a10e-a4bee353ce60",
    "deletionTimestamp": null,
    "accountEnabled": true,
    "assignedLicenses": [],
    "assignedPlans": [],
    "city": null,
    "country": null,
    "department": null,
    "dirSyncEnabled": null,
    "displayName": "Alex Wu",
    "facsimileTelephoneNumber": null,
    "givenName": null,
    "immutableId": null,
    "jobTitle": null,
    "lastDirSyncTime": null,
    "mail": null,
    "mailNickname": "AlexW",
    "mobile": null,
    "onPremisesSecurityIdentifier": null,
    "otherMails": [],
    "passwordPolicies": null,
    "passwordProfile": null,
    "physicalDeliveryOfficeName": null,
    "postalCode": null,
    "preferredLanguage": null,
    "provisionedPlans": [],
    "provisioningErrors": [],
    "proxyAddresses": [],
    "sipProxyAddress": null,
    "state": null,
    "streetAddress": null,
    "surname": null,
    "telephoneNumber": null,
    "usageLocation": null,
    "userPrincipalName": "alex@a830edad9050849NDA1.com",
    "userType": "Member"
}
\end{minted}
\caption[Voorbeeld Response Body Azure AD Graph]{Voorbeeld van een Response Body binnen Azure \ac{AD} Graph.}
\label{hresb}
\end{listing}

% ---
% Volgend onderdeel
% ---

\section{Wat is Microsoft Graph?}

Microsoft Graph is, net zoals de reeds besproken Azure \Ac{AD} Graph, een centrale \ac{REST} \ac{API} voor Microsoft-entiteiten. Deze technologie is verschillend, maar volgt de uitfaserende Azure \ac{AD} Graph op \autocite{Microsoft2023n}. Microsoft Graph werd door \textcite{Microsoft2015a} in de schijnwerper gezet in 2015. \\

De technologie is gebasseerd op de wiskundige graaf, zoals ook reeds besproken werd bij Azure \Ac{AD} Graph. Microsoft Graph integreert applicaties met verschillende programmeertalen en platformen. Het logisch concept van Graph wordt weergegeven op Figuur \ref{msg}.

\begin{figure}[h]
    \includegraphics[width=\textwidth]{MicrosoftGraph.png}
    \caption[Voorbeeld Microsoft Graph]{Voorstelling van Microsoft Graph door \textcite{Microsoft2023d}.}
    \label{msg}
\end{figure}

\subsection{Uitfasering van Azure AD Graph}
 
Microsoft Graph bepaalde PowerShell-modules die de Azure \ac{AD} en MSOnline PowerShell-modules kunnen vervangen. Op 30 september 2022 maakte Microsoft bekend dat de uitfasering van de Azure \ac{AD} PowerShell-modules en \ac{ADAL} van start zou gaan op 30 juni 2023 \autocite{Sahay2022}. \\

De reden waarom Microsoft deze overgang in gang zet, ontstaat uit volgende redenen volgens \textcite{Microsoft2023e}:

\begin{itemize}
    \item Het gebruik van Microsoft Graph ligt dubbel zo hoog dan dat van Azure \ac{AD} Graph.
    \item Microsoft Graph bevat meer dan 150 nieuwe functies.
    \item Microsoft Graph biedt meer veiligheid en is veerkrachtiger.
    \item Microsoft Graph client libraries bevatten een ingebouwde ondersteuning voor bepaalde functies, waaronder
    \begin{itemize}
        \item herhaalde verwerkingshandelingen,
        \item veilige doorverwijzing,
        \item transparante verificatie,
        \item en payloadcompressie.
    \end{itemize}
    \item Verbeterde mogelijkheden zoals Microsoft 365-groepsbeheer, uitnodiging voor externe gebruikers en anderen.
\end{itemize} 

\subsubsection{Uitfasering van ADAL}

Naast de uitfasering van de Azure \ac{AD} PowerShell-modules, start ook de uitval van \ac{ADAL} op 30 juni 2023. Zoals reeds besproken in Azure \ac{AD} wordt er gebruikgemaakt van \ac{ADAL} of \ac{MSAL} om met de oude als nieuwe Graph te kunnen verbinden. \\

De migratie naar \ac{MSAL} brengt voordelen met zich mee, dat \ac{ADAL} niet kan aanbieden, verklaard \autocite{Microsoft2023m}. Daarnaast is \ac{MSAL} gebouwd om met het Microsoft identity platform te werken, dat de werking met Microsoft Graph vereenvoudigd.



\subsection{Microsoft Graph toepassingen}

Microsoft Graph bestaat uit drie toepassingen. \\

De eerste toepassing is de Microsoft Graph \Ac{API}. Dit is een eindpunt dat kan aangesproken worden via “https://graph.microsoft.com” dat Microsoft-entiteiten binnen de cloud kan aanspreken. Dit onderdeel wordt later dieper besproken. \\

De tweede toepassing is Microsoft Graph connectors. Deze Graph-connectoren werken in de ingaande richting. Door het gebruik van deze connectoren, kunnen gegevens buiten de Microsoft-cloud worden gestuurd naar Graph en bijhorende toepassingen. \\

Door het gebruik van deze connectors vermindert de overhead, doordat de informatie bereikbaar is. Deze bereikbaarheid is te danken aan het gebruik van eindpunten zoals SharePoint, Office.com of Bing.com dat de connectors ondersteunen. Op Figuur \ref{MSGC} wordt de werking van Connectors geïllustreerd. \\

\begin{figure}[!h]
    \includegraphics[width=\textwidth]{GraphConnectors.png}
    \caption[Voorbeeld Microsoft Graph Connectors]{Voorstelling van Connectors binnen Microsoft Graph door \textcite{Hay2021}.}
    \label{MSGC}
\end{figure}

Als laatste worden er een set van tools aangeboden via Microsoft Graph Data Connect. Deze tools kunnen gebruikt worden om applicaties te ontwikkelen voor analyse, intelligentie en bedrijfsprocesoptimalisatie met Microsoft 365-data. Deze data wordt geïntegreerd binnen Azure, zodat de reken- en opslagcapaciteiten van het cloudplatform worden gebruikt. Een voorstelling van Data Connect is te zien op Figuur \ref{MSGDC}. \\

\begin{figure}[!h]
    \includegraphics[width=\textwidth]{GraphDataConnect.png}
    \caption[Voorbeeld Microsoft Graph Data Connect]{Voorstelling van Data Connect binnen Microsoft Graph door \textcite{Microsoft2022c}.}
    \label{MSGDC}
\end{figure}

\subsection{Authenticatie en Autorisatie}

Vooraleer er een token wordt uitgegeven, moet een applicatie geregistreerd zijn in het Microsoft identity platform \autocite{Microsoft2022b}. Bovendien moet de applicatie beschikken over de juiste machtigingen om Microsoft Graph te mogen gebruiken. \\

\subsubsection{Toegangsmogelijkheden}

Binnen Microsoft Graph zijn er twee mogelijkheden om toegang te verschaffen tot de gegevens. Dit kan via de gedelegeerde- of App-only-methode. Een simpele weergave van wat deze methodes betekenen wordt weergegeven op Figuur \ref{MIPM}. \\

\begin{figure}[h]
    \includegraphics[width=\textwidth]{MIPmethods.png}
    \caption[Voorbeeld toegangsmogelijkheden]{Voorstelling van de twee mogelijke methodes om toegang te krijgen tot het Microsoft identity platform door \autocite{Microsoft2022b}.}
    \label{MIPM}
\end{figure}

De eerste methode, delegated access, staat voor het toegang hebben namens de gebruiker. Dit is mogelijk wanneer een gebruiker zich aanmeldt bij een applicatie. Hierdoor geeft de gebruiker toestemming. Vervolgens kan Microsoft Graph in naam van de gebruiker een actie uitvoeren. Zowel de applicatie als de gebruiker moet over de juiste machtigingen beschikken om het verzoek te laten doorgaan. \\

Een ander woord voor gedelegeerde machtigingen zijn scopes. Deze scopes maken gebruik van OAuth2-permissies. Microsoft Graph biedt voor bepaalde onderdelen permission scopes aan die de rechten van een gebruiker of applicatie bepalen. Door \textcite{Microsoft2023p} wordt er een overzicht gegeven van alle permissie scope onderdelen of resources die Microsoft Graph op dit moment aanbiedt. \\


App-only access is de tweede methode. Dit principe werkt zonder een gebruiker om met gegevens te interageren. Dit principe komt eerder voor bij automatische taken (bv. een back-up) of achtergronddiensten (bv. daemons). Dit principe is aan te raden wanneer een gebruiker niet mag inloggen of de vereiste gegevens via meerdere gebruikers worden toegewezen. \\

Bij de tweede methode moet de applicatie over de juiste privileges beschikken. Dit kan via permissies of app-rollen. Een andere manier is via toekennen van eigendomsrechten aan de applicatie. 

\subsubsection{Access tokens}

Access tokens worden verkregen wanneer een toepassing of applicatie een authenticatieverzoek doet. Deze tokens worden gebruikt om de \ac{API} aan te spreken. \\

Als extra veiligheidsprincipe worden toegangstokens van het Microsoft-\newline
identiteitsplatform uitgerust met claims. Claims bevatten extra informatie dat kan dienen als extra validatiemethode. Deze validatie wordt gebruikt om na te kijken of de instantie wel degelijk over de juiste rechten beschikken om bepaalde acties uit te voeren. \\

Deze tokens worden behandeld als ondoorzichtige strings dat alleen voor de \ac{API} bedoeld zijn. Een voorbeeld van een access token is te vinden op Figuur \ref{MSGAT}. \\

\begin{figure}[h]
    \footnotesize\begin{verbatim}eyJ0eXAiOiJKV1QiLCJhb ... lciIUs9DrBLfpCt
\end{verbatim}    
    \caption[Afgekort voorbeeld Microsoft Graph access token]{Afgekort voorbeeld van een access token binnen Microsoft Graph.}
    \label{MSGAT}
\end{figure}

Microsoft Graph wordt aangeroepen door een autorisatieverzoek van een applicatie. De toegangstoken wordt als een Bearer-token gekoppeld aan de Autorisatieheader binnen een \ac{HTTP}-verzoek. Dit concept is gelijkaardig aan dat van een Azure \ac{AD}-token dat al reeds besproken werd. Een voorbeeld van een \ac{HTTP}-verzoek met deze onderdelen wordt weergegeven op Figuur \ref{MSGA}. \\

\begin{figure}[h]
    \footnotesize\begin{verbatim}GET https://graph.microsoft.com/v1.0/me/ HTTP/1.1
Host: graph.microsoft.com
Authorization: Bearer EwAoA8l6BAAU ... 7PqHGsykYj7A0XqHCjbKKgWSkcAg==
    \end{verbatim}    
    \caption[Voorbeeld Microsoft Graph Autorisatieverzoek]{Een voorbeeld van een autorisatieverzoek met een access token binnen Microsoft Graph.}
    \label{MSGA}
\end{figure}

Om aan access tokens te geraken, bestaan er oplossingen zoals authenticatiebibliotheken. Een oplossing van Microsoft is \ac{MSAL} dat voor dit scenario toegankelijk is. Bovendien bestaan er alternatieven zoals Server middleware en authenticatiebibliotheken van derde partijen. \\

Hoe dan ook, deze bibliotheken zijn niet verplicht. Access tokens kunnen ook rechtstreeks verkregen worden als dit gewenst is. 

\subsubsection{Toegang hebben op naam van een gebruiker}

Zoals reeds vermeld, regelen access tokens de toegang tot het gebruik van Microsoft Graph. Om een access token te verkrijgen via een gebruiker, worden de volgende vijf stappen gehanteerd. \\

Als eerste stap moet de applicatie worden geregisteerd bij Azure \ac{AD}. De registratie gebeurd via het onderdeel App Registrations binnen Azure. Wanneer de applicatie geregistreerd is, gaat het Microsoft-identiteitsplatform instaan voor volgende informatie: 

\begin{itemize}
    \item Application ID: Unieke identificatie.
    \item Redirect URI/URL: Eindpunt waar de applicatie wordt op aangesproken.
    \item Client secret: Wachtwoord of sleutelpaar dat gebruikt wordt voor authenticatie. Dit wordt gebruikt bij webapplicaties.
\end{itemize}

Als tweede volgt het autoriseren. Dit gebeurd via het Microsoft identity platform. Via dit platform kan een gebruiker aanmelden en toestemming geven. Bij het verkrijgen van toestemming wordt er een code teruggestuurd. Deze code zorgt voor het verlenen van een access token. Een voorbeeld van deze terugkerende code wordt weergegeven op Figuur \ref{MSGAR}. \\

\begin{figure}[h]
    \footnotesize
    \begin{verbatim}
GET https://localhost/myapp/?
code=M0ab92efe-b6fd-df08-87dc-2c6500a7f84d
&state=12345
    \end{verbatim}    
    \caption[Voorbeeld Microsoft Graph Authorization response]{Een voorbeeld van een autorisatie antwoord met de nodige code om een access token te verkrijgen binnen Microsoft Graph.}
    \label{MSGAR}
\end{figure}

Daarna moet er een token worden verkregen. Dit gebeurd via een tokenverzoek dat op Figuur \ref{HTR} wordt weergegeven. Na het versturen van dit verzoek volgt er een antwoord met een access token in \ac{JSON}-formaat. Dit antwoord wordt voorgesteld op Figuur \ref{HTRES}. \\ 

\begin{figure}[!h]
    \footnotesize\begin{verbatim}
POST /common/oauth2/v2.0/token HTTP/1.1
Host: https://login.microsoftonline.com
Content-Type: application/x-www-form-urlencoded
        
client_id=6731de76-14a6-49ae-97bc-6eba6914391e
&scope=user.read%20mail.read
&code=OAAABAAAAiL9Kn2Z27UubvWFPbm0gLWQJVzCTE9UkP3pSx1aXxUjq3n8b2JRLk4OxVXr...
&redirect_uri=http%3A%2F%2Flocalhost%2Fmyapp%2F
&grant_type=authorization_code
&client_secret=JqQX2PNo9bpM0uEihUPzyrh
    \end{verbatim}    
    \caption[Voorbeeld User Token Request Microsoft Graph]{Een voorbeeld van een tokenverzoek via \ac{HTTP} binnen Microsoft Graph.}
    \label{HTR}
\end{figure}

\begin{figure}[!h]
    \footnotesize\begin{verbatim}
{
    "token_type": "Bearer",
    "scope": "user.read%20Fmail.read",
    "expires_in": 3600,
    "access_token": "eyJ0eXAiOiJKV1QiLCJhbGciOiJSUzI1NiIsIng1dCQ...",
    "refresh_token": "AwABAAAAvPM1KaPlrEqdFSBzjqfTGAMxZGUTdM0t4B4..."
}        
    \end{verbatim}    
    \caption[Voorbeeld User Token Response Microsoft Graph]{Een voorbeeld van een tokenantwoord via \ac{JSON} binnen Microsoft Graph.}
    \label{HTRES}
\end{figure}

Vervolgens wordt de verkregen token gebruikt om Microsoft Graph op te roepen. Het toegankstoken wordt in de autorisatieheader van een verzoek geplaatst. Dit werd al reeds voorgesteld op Figuur \ref{MSGA}. \\

Als laatste volgt het gebruik van een refresh token. Een refresh token ververst de duur van een toegangstoken, doordat een toegangstoken kan vervallen. Een verzoek en antwoord, met het gebruik van refresh tokens, is te vinden op Figuur \ref{MSGRTR} en \ref{MSGRTRES}.

\begin{figure}[!h]
    \footnotesize\begin{verbatim}
POST /{tenant}/oauth2/v2.0/token HTTP/1.1
Host: https://login.microsoftonline.com
Content-Type: application/x-www-form-urlencoded

client_id=11111111-1111-1111-1111-111111111111
&scope=user.read%20mail.read
&refresh_token=OAAABAAAAiL9Kn2Z27UubvWFPbm0gLWQJVzCTE9UkP3pSx1aXxUjq...
&grant_type=refresh_token
&client_secret=jXoM3iz...   
    \end{verbatim}    
    \caption[Voorbeeld Refresh Token request Microsoft Graph]{Een voorbeeld van een verzoek met een refresh token via \ac{HTTP} binnen Microsoft Graph.}
    \label{MSGRTR}
\end{figure}

\begin{figure}[!h]
    \footnotesize\begin{verbatim}
{
    "access_token": "eyJ0eXAiOiJKV1QiLCJhbGciOiJSUzI1NiIsIng1dCI6Ik5HVEZ2ZEstZnl0aEV1Q...",
    "token_type": "Bearer",
    "expires_in": 3599,
    "scope": "user.read%20mail.read",
    "refresh_token": "AwABAAAAvPM1KaPlrEqdFSBzjqfTGAMxZGUTdM0t4B4...",
}    
    \end{verbatim}    
    \caption[Voorbeeld Refresh Token response Microsoft Graph]{Een voorbeeld van een antwoord met een refresh token via \ac{JSON} binnen Microsoft Graph.}
    \label{MSGRTRES}
\end{figure}



\subsubsection{Toegang hebben zonder een gebruiker}

De tweede manier om een access token te verkrijgen, is zonder het gebruik van een user. Dit zijn de volgende vijf stappen gehanteerd. \\

De eerste stap is het registreren van een applicatie. Dit komt overeen met de uitleg die in de eerste manier werd beschreven. \\

Vervolgens moeten de permissies van Microsoft Graph worden geconfigureerd. Dit is nodig zodat een applicatie zonder toestemming een token mag aanvragen. Dit kan geconfigureerd worden onder “Request \ac{API} permission” bij het registratieportaal voor applicaties binnen Azure. \\

De derde stap omvat het verkrijgen van toegang door de beheerder. Het verkijgen van toegang door een beheerder kan gebeuren op twee manieren. De eerste manier is via Azure, door het gebruik van het portaal. De andere optie is via het Microsoft identity platform door middel van een verzoek naar het adminconsent-eindpunt. Een bijhorend voorbeeld van een verzoek en antwoord wordt weergegeven op \ref{MSGRAR} en \ref{MSGRARES}. \\

\begin{figure}[!h]
    \footnotesize\begin{verbatim}
GET https://login.microsoftonline.com/{tenant}/adminconsent
?client_id=6731de76-14a6-49ae-97bc-6eba6914391e
&state=12345
&redirect_uri=https://localhost/myapp/permissions
    \end{verbatim}    
    \caption[Voorbeeld Adminconsent request Microsoft Graph]{Een voorbeeld van een verzoek voor toestemming van de beheerder via \ac{HTTP} binnen Microsoft Graph.}
    \label{MSGRAR}
\end{figure}

\begin{figure}[!h]
    \footnotesize\begin{verbatim}
GET https://localhost/myapp/permissions
?tenant=a8990e1f-ff32-408a-9f8e-78d3b9139b95&state=12345
&admin_consent=True
    \end{verbatim}    
    \caption[Voorbeeld Adminconsent respons Microsoft Graph]{Een voorbeeld van een antwoord voor toestemming van de beheerder via \ac{HTTP} binnen Microsoft Graph.}
    \label{MSGRARES}
\end{figure}

Bij de vierde stap moet er een access token aanwezig zijn. Dit principe komt overeen met de derde stap van de vorige methode. Een voorbeeld van een verzoek en antwoord binnen deze stap wordt weergegeven op Figuur \ref{MSGATRR} en \ref{MSGATRRES}. \\

\begin{figure}[!h]
    \footnotesize\begin{verbatim}
POST https://login.microsoftonline.com/{tenant}/oauth2/v2.0/token HTTP/1.1
Host: login.microsoftonline.com
Content-Type: application/x-www-form-urlencoded

client_id=535fb089-9ff3-47b6-9bfb-4f1264799865
&scope=https%3A%2F%2Fgraph.microsoft.com%2F.default
&client_secret=qWgdYAmab0YSkuL1qKv5bPX
&grant_type=client_credentials 
    \end{verbatim}    
    \caption[Voorbeeld Application Token Request Microsoft Graph]{Een voorbeeld van een verzoek met een applicatie access token via \ac{HTTP} binnen Microsoft Graph.}
    \label{MSGATRR}
\end{figure}

\begin{figure}[!h]
    \footnotesize\begin{verbatim}
{
    "token_type": "Bearer",
    "expires_in": 3599,
    "ext_expires_in":3599,
    "access_token": "eyJ0eXAiOiJKV1QiLCJhbGciOiJSUzI1NiIsIng1dCI6Ik1uQ19WWmNBVGZNNXBP..."
} 
    \end{verbatim}    
    \caption[Voorbeeld Application Token Response Microsoft Graph]{Een voorbeeld van een antwoord met een applicatie access token via \ac{JSON} binnen Microsoft Graph.}
    \label{MSGATRRES}
\end{figure}

De vijfde stap is het gebruiken van de access token om Microsoft Graph op te roepen. Dit komt overeen met de vierde stap van de eerste manier. Er wordt een verzoek gedaan met een access token dat verkregen werd. Een voorbeeld van dit verzoek is te vinden op Figuur \ref{MSGAAT}.

\begin{figure}[!h]
    \footnotesize\begin{verbatim}
GET https://graph.microsoft.com/v1.0/users/12345678-73a6-4952-a53a-e9916737ff7f
Authorization: Bearer eyJ0eXAiO ... 0X2tnSQLEANnSPHY0gKcgw
Host: graph.microsoft.com
    \end{verbatim}    
    \caption[Voorbeeld Application autorisatieverzoek Microsoft Graph]{Een voorbeeld van een autorisatieverzoek met een application access token via \ac{HTTP} binnen Microsoft Graph.}
    \label{MSGAAT}
\end{figure}



\subsection{Microsoft Graph API}

Zoals vermeld is de Microsoft Graph \Ac{API} gebasseerd op \Ac{REST}. Door het gebruik van de \ac{API} zijn de cloud-servicebronnen van Microsoft beschikbaar. \\

Om tot deze bronnen toegang te hebben en om de \Ac{API} aan te spreken, moeten er twee acties voldaan zijn:

\begin{itemize}
    \item De applicatie is geregistreerd.
    \item Er zijn authenticatietokens verkregen voor een gebruiker of instantie.
\end{itemize}

\subsubsection{OData}

De \ac{API} maakt gebruik van OData. OData werd al reeds besproken in het vorige onderdeel, dit wordt dus niet nog eens aangekaart. Door het gebruik van OData kunnen bronnen, methodes en opsommingen in de metadata van Microsoft Graph worden gedefinieerd.

\subsubsection{REST API}

Om naar een bron te schrijven of te lezen, wordt er gebruikgemaakt van een verzoek. De opbouw van een verzoek wordt weergegeven op Figuur \ref{RAM}. \\

\begin{figure}[h]
    \footnotesize\begin{verbatim}https://graph.microsoft.com/{version}/{resource}?query-parameters
    \end{verbatim}    
    \caption[Basis formaat Microsoft Graph API-eindpunt]{Het basis formaat van een Microsoft Graph \Ac{API}-eindpunt uit documentatie van \textcite{Microsoft2023o}.}
    \label{RAM}
\end{figure}

Een verzoek bestaat uit vier componenten, namelijk het volgende: 

% TODO: FOUT? HTTP method nakijken!!!
\begin{itemize}
    \item \ac{HTTP} method: Een \ac{HTTP}-methode dat gebruik wordt voor het verzoek.
    \item version: De versie van de \ac{API} dat de bijhorende applicatie ondersteunt.
    \item resource: De bron of resource dat gebruikt wordt.
    \item query-parameters: Optionele OData-queryparameters of \Ac{REST}-methodes dat het antwoord aanpassen.
\end{itemize}

% TODO: nakijken!
\begin{comment}
Wanneer er een verzoek wordt verstuurd, krijgt dit ook een antwoord terug. Een antwoord bestaat uit minstens volgende onderdelen: 

\begin{itemize}
    \item Status code:
    \item Response message:
    \item @odata.nextLink:
\end{itemize}
\end{comment}

\subsubsection{HTTP methods}

Microsoft Graph werkt met \Ac{HTTP}-methodes bij een verzoek. Microsoft Graph maakt gebruik van vijf soorten methodes, namelijk:

\begin{itemize}
    \item GET
    \item POST
    \item PATCH
    \item PUT
    \item DELETE
\end{itemize}

De betekenis van deze vijf methodes werd al reeds aangehaald bij Azure \Ac{AD} Graph. \\

De GET- en DELETE-methode vallen onder CRUD-methodes. CRUD staat voor CREATE, READ, UPDATE en DELETE \autocite{Truica2015}. De CRUD-operaties komen voor in het \ac{IT}-component rond databanken. In Microsoft Graph staan GET en DELETE gelijk aan de CRUD-operaties READ en DELETE. Bovendien maken deze twee methodes geen gebruik van een request body. \\

De andere drie methodes (POST, PATCH en PUT) gebruiken wel een request body. Meestal wordt in dit deel van het verzoek gebruikgemaakt van het \ac{JSON}-formaat. In dit formaat wordt er extra informatie meegeven over welke waarden moeten gebruikt worden voor een operatie of methode uit te voeren. Een voorbeeld van een request met bijhorend antwoord is te vinden op Figuur \ref{MSPR} en \ref{MSPRES}. \\

\begin{figure}[h]
    \footnotesize\begin{verbatim}POST
https://graph.microsoft.com/v1.0/tenantRelationships/delegatedAdminRelationships/
5d027261- ... -a3205431b836/requests
Content-Type: application/json

{
    "action": "lockForApproval"
}
    \end{verbatim}    
    \caption[Voorbeeld Microsoft Graph POST-verzoek]{Een voorbeeld van een \ac{HTTP} POST-verzoek binnen Microsoft Graph.}
    \label{MSPR}
\end{figure}

\begin{figure}[h]
    \footnotesize\begin{verbatim}
HTTP/1.1 201 Created
Content-Type: application/json
Location: https://graph.microsoft.com/v1.0/tenantRelationships/
delegatedAdminRelationships/c45e5ffb- ... -25a5a3fbf339

{
    "@odata.type": "#microsoft.graph.delegatedAdminRelationshipRequest",
    "@odata.context": "https://graph.microsoft.com/v1.0\n
    /tenantRelationships/$metadata#requests",
    "id": "5a6666c9-7282-0a41-67aa-25a5a3fbf339",
    "action": "lockForApproval",
    "status": "created",
    "createdDateTime": "2022-02-10T10:55:47.1180588Z",
    "lastModifiedDateTime": "2022-02-10T10:55:47.1180588Z"
}
    \end{verbatim}    
    \caption[Voorbeeld Microsoft Graph POST-antwoord]{Een voorbeeld van een \ac{HTTP} POST-antwoord binnen Microsoft Graph.}
    \label{MSPRES}
\end{figure}

\subsubsection{Beschikbare versies}

Microsoft Graph is volop in ontwikkeling en ondersteunt, op dit moment van het schrijven, twee versies \Autocite{Microsoft2023f}. \\

De eerste versie, onder de naam v1.0, is een stabiele versie dat toegepast kan worden in productieomgevingen en -applicaties. \\

De tweede versie, onder de naam beta, is een versie waar actief aan (nieuwe) concepten gesleuteld wordt. Deze versie wordt toegepast in testomgevingen, doordat het nog in ontwikkeling is. Het wordt niet aangeraden voor in productie te gebruiken. 

\subsubsection{Wat zijn resources?}

Bronnen, of beter bekend als resources, zijn entiteiten of types die gebruikt worden om de communicatie met een verzoek mogelijk te maken. Zoals in de \ac{URL} weergegeven, komt een resource hierin voor. \\

Mogelijke bronnen zijn bijvoorbeeld “me”, een user, een groep, etc. Deze bronnen ondersteunen ook relaties, waardoor andere onderdelen aangesproken kunnen worden. Ter illustratie, wanneer er een gebruiker een mail wilt sturen, is dit mogelijk via “me/sendMail”. Een voorbeeld van het gebruik van een resource is te vinden op Figuur \ref{MSGR}. \\

\begin{figure}[h]
    \footnotesize\begin{verbatim}GET https://graph.microsoft.com/v1.0/me/onenote/resources/{id}/content
    \end{verbatim}    
    \caption[Voorbeeld Microsoft Graph resource]{Een voorbeeld van een “me/onenote” resource binnen Microsoft Graph.}
    \label{MSGR}
\end{figure}

Bronnen vereisen machtigingen om bepaalde acties uit te voeren. Hier moet rekening mee gehouden worden voor er een actie wordt uitgevoerd.

\subsubsection{Responses aanpassen via queryparameters}

Antwoorden of responses worden weergegeven na het gebruik van een bepaald HTTP-verzoek. Het antwoord kan aangepast worden via twee soorten queryparameters. \\

Als eerste de OData-queryparameters. Deze queryparamaters zijn afkomstig van het OData-protocol waarvan bepaalde parameters worden ondersteund door \textcite{Microsoft2023g}. Een toegepast voorbeeld van een verzoek met een OData-queryparameter wordt weergegeven op Figuur \ref{TRAM}. \\

\begin{figure}[h]
    \footnotesize\begin{verbatim}GET https://graph.microsoft.com/v1.0/me?$select=displayName,jobTitle
    \end{verbatim}    
    \caption[Voorbeeld OData HTTP-verzoek]{Praktisch OData voorbeeld van een Microsoft Graph \Ac{API}-eindpunt, waarbij er op “displayName” en “jobTitle” wordt gefilterd.}
    \label{TRAM}
\end{figure}

Het tweede soort dat antwoorden kan aanpassen zijn de normale queryparameters. Met het woord normaal worden niet-OData-gerelateerde parameters bedoelt. Een voorbeeld van dit soort queryparameters is te vinden op Figuur \ref{NRAM}. \\

\begin{figure}[h]
    \footnotesize\begin{verbatim}GET https://graph.microsoft.com/me/calendarView?startDateTime=
2019-09-01T09:00:00.0000000&endDateTime=2019-09-01T17:00:00.0000000
    \end{verbatim}    
    \caption[Voorbeeld niet-OData HTTP-verzoek]{Praktisch niet-OData voorbeeld van een Microsoft Graph \Ac{API}-eindpunt, waarbij een bepaalde tijdsperiode wordt opgevraagd.}
    \label{NRAM}
\end{figure}

\subsubsection{Tools om de Microsoft Graph API te testen}

De werking van Microsoft Graph API kan worden getest in een demo-omgeving. Deze omgeving wordt Graph Explorer genoemd en wordt voorzien door \textcite{Microsoft2023h}. Hierin kunnen verzoeken worden getest met of zonder Microsoft-tenant. \\

Daarnaast kan de API ook worden getest via Postman. Postman is een collaboratief \ac{API}-platform dat los van Microsoft staat \autocite{Postman2023}.

% TODO: \subsubsection{MSAL ...}

\subsection{Microsoft Graph met PowerShell}

Microsoft Graph kan worden aangestuurd met PowerShell \autocite{Microsoft2023j}. Deze PowerShell-integratie zit in de Microsoft Graph PowerShell \ac{SDK}. Een \ac{SDK} bestaat uit een verzameling van tools dat instaan voor het maken van bepaalde software of hardware \autocite{RedHat2020}. Deze PowerShell \ac{SDK} stelt de \ac{API} van Microsoft Graph bloot. Door deze blootstelling kan PowerShell de \ac{API} gebruiken, beheren en automatiseren.

\subsubsection{Ondersteunende dependencies}

Microsoft Graph PowerShell ondersteunt een lijst van dependencies \autocite{Microsoft2023k}. Een overzicht van alle ondersteunde dependencies is te vinden in Tabel \ref{MSGDT}. 

\begin{table}
    \small
    \centering
    \begin{tabular}{ |c| } 
        \hline
        \textbf{Dependencies (tot aan versie 1.23)} \\
        \hline
        Applications \\
        Authentication \\
        Bookings \\
        Calender \\
        Changing Notifications \\
        Cloud Communications \\
        Compliance \\
        Cross Device Experiences \\
        Device Management \\
        Device Management Actions \\
        Device Management Administration \\
        Device Management Enrolment \\
        Device Management Functions \\
        Devices Cloud Print \\
        Devices Corporate Management \\
        Devices Service Announcement
        Directory Objects \\
        Education \\
        Files \\
        Financials \\
        Groups \\
        Identity Directory Management \\
        Identity Governance \\
        Identity Sign-ins \\
        Mail \\
        Managed Tenants \\
        Notes \\
        People \\
        Personal Contacts \\
        Planner \\
        Reports \\
        Schema Extensions \\
        Search \\
        Sites \\
        Teams \\
        Users \\
        Users Actions \\
        Users Funtions \\
        Windows Updates \\
        \hline
    \end{tabular}
    \caption[Tabel Microsoft Graph dependencies]{Tabel met overzicht van alle ondersteunende dependencies tot aan versie 1.23 binnen Microsoft Graph PowerShell.}
    \label{MSGDT}
\end{table}

\subsubsection{Geïntegreerde data-objecten van Azure AD}

Door de uitfasering van Azure \ac{AD} PowerShell, worden de data-objecten van de AzureAD-module geïntegreerd binnen Microsoft Graph \autocite{Microsoft2023l}. Een overzicht van deze integratie wordt weergegeven in Tabel \ref{MSGDOT}.

\begin{table}
    \small
    \centering
    \begin{tabular}{ |c|c| } 
        \hline
        \textbf{Geïntegreerd data-object} & \textbf{Aantal commando's} \\
        \hline
        Administrative Units & 9 \\ 
        Applications & 20 \\ 
        AzureAD & 48 \\ 
        Certificate Authorities & 1 \\ 
        Connect to directory & 2 \\ 
        Contacts & 7 \\ 
        Contracts & 1 \\ 
        Deleted Objects & 1 \\ 
        Devices & 9 \\    
        Directory & 3 \\
        Directory Objects & 1 \\ 
        Directory Roles & 13 \\ 
        Domains & 8 \\ 
        Extension Properties & 1 \\ 
        Groups & 26 \\
        MSOnline & 76 \\ 
        OAuth2 & 2 \\ 
        Policies & 2 \\ 
        Service Principals & 22 \\ 
        Users & 30 \\ 
        \hline
    \end{tabular}
    \caption[Tabel geïntegreerde data-objecten]{Tabel met overzicht van alle geïntegreerde data-objecten met bijhorende commando's binnen Microsoft Graph PowerShell.}
    \label{MSGDOT}
\end{table}

% === END ===
%%=============================================================================
%% Methodologie
%%=============================================================================

\chapter{\IfLanguageName{dutch}{Methodologie}{Methodology}}%
\label{ch:methodologie}

%% TODO: Hoe ben je te werk gegaan? Verdeel je onderzoek in grote fasen, en
%% licht in elke fase toe welke stappen je gevolgd hebt. Verantwoord waarom je
%% op deze manier te werk gegaan bent. Je moet kunnen aantonen dat je de best
%% mogelijke manier toegepast hebt om een antwoord te vinden op de
%% onderzoeksvraag.

Om een antwoord te vinden op deze onderzoeksvraag, worden de volgende stappen gehanteerd. \\

Als eerste wordt het onderzoeksdomein volledig gekaderd aan de hand van een literaire stand van zaken. Na het verzamelen en verwerken van alle literaire studies, volgt de vergelijking tussen het oude en het nieuwe. Zoals reeds vermeld wordt de PowerShell-module en \ac{API} van Azure \ac{AD} uitgefaseerd. Azure \ac{AD} wordt gezien als het “oude”. De nieuwe Microsoft Graph, de vervanger van Azure \ac{AD}, wordt gezien in de vergelijking als het “nieuwe”. \\

De vergelijking tussen de twee bevat de eerste helft van het antwoord op de onderzoeksvraag. In deze vergelijking wordt er gekeken naar vier criteria of onderdelen. 

\begin{itemize}
    \item Werking van de \ac{API}: Wat zijn de mogelijke \Ac{HTTP}-verzoeken? Uit welke onderdelen bestaat het eindpunt? Zijn er verschillen aanwezig in requests en reponses tijdens het uitvoeren van een query? Welke PowerShell-versies zijn er?
    \item Aanspreekbare data-objecten en dependencies: Welke data-objecten kunnen worden aangesproken? Bevat de technologie bepaalde afhankelijkheden?
    \item Toegangsmogelijkheden: Hoe wordt er toegang verschaft tot de technologie? 
    \item Security: Zijn de toegangsmogelijkheden veilig? Wordt er gebruikgemaakt van rechten om misbruik tegen te gaan?
\end{itemize}

Met deze vier onderdelen wordt de vergelijking tussen het oude en het nieuwe uitgevoerd. Deze vergelijking wordt aan de hand van de verwerkte literatuur toegepast. \\

Om de tweede helft van het antwoord te geven op de onderzoeksvraag wordt de casus praktisch uitgewerkt in een Proof-of-Concept. Het verouderde auditing-script van Easi, dat geschreven is in PowerShell, maakt gebruik van de verouderde modules. Dit script wordt vernieuwd door middel van de modernere Microsoft Graph PowerShell-modules. Door dit script op een praktisch manier te herwerken, kan dit aantonen wat vandaag de dag wel en niet mogelijk is met Microsoft Graph. \\

Tijdens de praktische uitwerking wordt er gewerkt met een vijfstappenplan. Als eerste, het bekijken en beschrijven van de startende dataset na het activeren van de licentie. Als tweede, wordt de dataset aangepast naar een nagebootste klantenomgeving. Deze nabootsing wordt op basis van een intern document binnen Easi uitgevoerd. Daarna volgt het aanmaken van een applicatie voor Microsoft Graph. Vervolgens, het maken van een PowerShell-script om de testomgeving aan te spreken. Als laatste, het omvormen van het verouderde PowerShell-script dat Easi gebruikt. \\

Om dit script zo efficiënt mogelijk te herwerken, worden twintig onderdelen van dit script behandeld. Deze twintig onderdelen of functies maken gebruik van de uitfaserende Azure \ac{AD} PowerShell-module. Hieronder volgt een opsomming van de twintig functies met een korte beschrijving. 

\begin{enumerate}
    \item Aantal aanwezige domeinen binnen de Office 365-omgeving.
    \item Aantal gebruikers binnen de omgeving, onderverdeeld in interne
    en extene gebruikers.
    \item Aantal gebruikers die geblokkeerd zijn.
    \item Aantal geblokkeerde gebruikers met actieve licenties.
    \item Overzicht en telling van administrators met hun account status.
    \item Overzicht en telling van de interne gebruikers met \ac{MFA}.
    \item Aantal gelicenseerde, actieve accounts waarbij \ac{MFA} niet aanwezig is.
    \item Aantal externe accounts waarbij \ac{MFA} uitstaat.
    \item Overzicht van accounts met administrator-priviliges,
    onderverdeeld in gebruik van \ac{MFA}.
    \item Groottes van elk type mailbox.
    \item Groottes van elke mailbox, gerangschikt van groot naar klein.
    \item Onderverdeling van gebruikersmailboxen op basis van grootte.
    \item Onderverdeling van gedeelde mailboxen op basis van grootte.
    \item Onderverdeling van hoeveelheid mailboxen per domein.
    \item Overzicht van accounts met als primair \Ac{SMTP}-adresdomein “.onmicrosoft.com”.
    \item Overzicht gedeelde mailboxen, onderverdeeld in aanwezigheid van een licentie.
    \item Overzicht van verborgen mailboxen.
    \item Overzicht van \ac{SMTP}-forwarding.
    \item Overzicht verschillende types van licenties en het aantal personen met dit licentietype.
    \item Overzicht verschillende methodes waarop \ac{MFA} wordt gebruikt met hoeveelheid gebruikers, onderverdeeld op basis van methode.
\end{enumerate}

Binnen deze twintig onderdelen zijn er vijf groepen van data. Dit zijn de vijf soorten data die worden nagekeken wanneer klanten een Office 365-audit uitvoeren door Easi. De vijf groepen worden hieronder weergegeven. 

\begin{itemize}
    \item Domeinen
    \item Gebruikers
    \item Licenties
    \item Mailboxen
    \item \ac{MFA}
\end{itemize}

Tijdens de praktische uitwerking wordt er gebruikgemaakt van de Microsoft Graph PowerShell-modules. In het bijzonder, de meest recente versie van de eerste stabiele uitrol dat beschikbaar is tijdens de uitvoering van dit onderzoek, namelijk versie 1.25 \autocite{Microsoft2023k}. De mogelijkheid is reëel dat er tijdens of na de uitwerking een recentere versie zal worden uitgegeven door Microsoft. Er wordt niet afgeweken van versie 1.25 met bijhorende modules in dit onderzoek. \\  

Wanneer het onderdeel kan worden uitgevoerd via de PowerShell-modules van Microsoft Graph, dan wordt er een PowerShell-script uitgewerkt voor dat onderdeel. Dit bewijst dat Microsoft Graph het onderdeel uit het audit-script kan vervangen. \\

Als een onderdeel niet kan worden uitgevoerd via de Microsoft Graph PowerShell-modules, dan wordt er dieper gekeken naar het bijhorende domein. Er wordt uitgezocht dat het domein wel of niet ondersteund wordt door de PowerShell-modules van Microsoft Graph. Een mogelijk alternatief wordt besproken, maar de uitwerking hiervan valt buiten de scope van dit onderzoek. \\

Uiteindelijk wordt een conclusie gevormd op basis van beide helften uit dit onderzoek. 

% === END ====

% Voeg hier je eigen hoofdstukken toe die de ``corpus'' van je bachelorproef
% vormen. De structuur en titels hangen af van je eigen onderzoek. Je kan bv.
% elke fase in je onderzoek in een apart hoofdstuk bespreken.

%%=============================================================================
%% Vergelijkende studie
%%=============================================================================

\chapter{Vergelijking oud en nieuw}%
\label{ch:vergelijking}

De eerste helft van de conclusie wordt gestaafd aan de hand van een vergelijkende studie. Azure \ac{AD} Graph en PowerShell wordt vergeleken met Microsoft Graph met bijhorende PowerShell-module. In de vergelijking worden beide technologieën aangesproken als het “oude” en het “nieuwe”. 

\section{Werking van de API}

% TO DO: 4. Ideeën nakijken => Normaal OK
% Idee: Vergelijking in hoe wordt de data verzonden, verwerkt en teruggestuurd?
% Vb. Begint met ... en eindigt met JSON. Is dit hetzelfde of...
% Note: ook nog mogelijk om meerdere subsecties of subsubsecties te maken om de fases goed in detail te bespreken en te vergelijken.

  

\subsection{Soorten HTTP-verzoek}

Zowel Azure \ac{AD} Graph als Microsoft Graph ondersteunen vijf soorten \ac{HTTP}-verzoeken. Deze vijf zijn GET, POST, PATCH, PUT en DELETE. 

\subsection{Eindpunt}

Op Figuur \ref{bfe} bij Azure \ac{AD} Graph, wordt er gebruikgemaakt van “windows” in de \ac{URL}. In vergelijking met Microsoft Graph, dat op Figuur \ref{RAM} te vinden is, komt het woord “microsoft” voor in de \ac{URL}. \\

Op vlak van onderdelen, wordt het volgende opgemerkt bij het vergelijken. Azure \ac{AD} Graph werkt met vier onderdelen: Tenant ID, Resource, (\ac{API}) Version en optionele OData query-parameters. Terwijl Microsoft Graph maar drie onderdelen gebruikt. de Tenant ID wordt uit het eindpunt geschrapt. Daarnaast worden de onderdelen “version” en “resource” van plaats gewisseld. De optionele query-parameters zijn in beide aanwezig. 


% TODO: Invullen wanneer er tijd over is
% !!! Idee: Alle HTTP-verzoeken uitvoeren, zien dat dit hetzelfde is en de resultaten opschrijven (zie hieronder voor voorbeeld)

% https://graphexplorer.azurewebsites.net/#
% GET: https://graph.windows.net/ffa43659-6d7d-4f83-a517-838af35d1353/domains
% Resultaat... Helemaal meegeven

% Dit hetzelfde voor MS Graph: https://developer.microsoft.com/en-us/graph/graph-explorer
% GET: https://graph.microsoft.com/v1.0/domains
% Resultaat... Helemaal meegeven

% VERGLIJKING !!!

% \subsection{Request en Response}

%Voor dit onderdeel wordt er gebruikgemaakt van de Graph Explorer die zowel het oude %als het nieuwe aanbiedt.



\subsection{PowerShell-versies}

% TODO: Laten nakijken door Jarne

Azure \ac{AD} Graph en Microsoft Graph maken gebruik van PowerShell-modules. Deze PowerShell-modules worden vernieuwd aan de hand van versies die nieuwe of stabiele elementen toevoegen (of verwijderen) aan de PowerShell-modules. \\

Azure \ac{AD} Graph werkt met twee soorten PowerShell-modules, namelijk MSOnline en AzureAD. MSOnline is de eerste versie, waarbij AzureAD de verbeterde versie is van MSOnline voor het beheren van Azure \ac{AD}. \\

Bij Microsoft Graph zijn er op dit moment twee versies beschikbaar. Versie 1.0 dat de stabiele versie voorstelt. Daarnaast bestaat ook de beta-versie, vandaag de dag voorgesteld als versie 2.0, dat de nadruk legt op nieuwe elementen die nog in ontwikkeling zijn. \\ 

% ---
% Section
% ---

\section{Aanspreekbare data-objecten en dependencies}

% Idee: uitleggen hoe de API werkt en hoe het aan een data-object kan aanspreken
% Vb. Waar zit het verschil? Is dit hetzelfde gebleven?

%Idee: een tabel of mooi overzicht over hoeveel data-objecten er kunnen worden aangesproken door beide.
%Vb. Zijn er veel nieuwe bijgekomen? Valt er iets op? Is er iets weggegaan...
% Hier wordt het data-object 


Met de migratie van Azure \ac{AD} Graph naar Microsoft Graph, wordt er verondersteld dat alle aanspreekbare data-objecten worden overgenomen. Wanneer het overzicht van data-objecten bij Azure \ac{AD} PowerShell naast dat van Microsoft Graph wordt gezet, ontstaat volgend overzicht dat te vinden is in Tabel \ref{AADMSG}. \\

\begin{table}
    \tiny
    \centering
    \begin{tabular}{ |c|c||c|c| } 
        \hline
        \textbf{Azure AD data-object} & \textbf{Aantal commando's} & \textbf{Microsoft Graph data-object} & \textbf{Aantal commando's} \\
        \hline
        Administrative Units & 9 & Administrative Units & 9 \\ 
        Application Proxy Application Management & 8 & / & / \\
        Application Proxy Connector Management & 9 & / & / \\
        Applications & 20 & Applications & 20 \\ 
        AzureAD & 49 & AzureAD & 48 \\ 
        Certificate Authorities & 4 & Certificate Authorities & 1 \\ 
        Connect to directory & 2 & Connect to directory & 2 \\ 
        Contacts & 8 & Contacts & 7 \\ 
        Contracts & 1 & Contracts & 1 \\ 
        Deleted Objects & 1 & Deleted Objects & 1 \\ 
        Devices & 11 & Devices & 9 \\    
        Directory & 3 & Directory & 3 \\
        Directory Objects & 1 & Directory Objects & 1 \\ 
        Directory Roles & 13 & Directory Roles & 13 \\ 
        Domains & 8 & Domains & 8 \\ 
        Extension Properties & 1 & Extension Properties & 1 \\ 
        Groups & 26 & Groups & 26 \\ 
        MSOnline & 97 & MSOnline & 76 \\
        OAuth2 & 2 & OAuth2 & 2 \\ 
        Policies & 2 & Policies & 2 \\ 
        Service Principals & 22 & Service Principals & 22 \\ 
        Users & 30 & Users & 30 \\ 
        \hline
    \end{tabular}
    \caption[Tabel migratie Azure AD data-objecten naar Microsoft Graph]{Tabel met overzicht van alle ondersteunende data-objecten bij de migratie van Azure \ac{AD} PowerShell naar Microsoft Graph uit documentatie van \textcite{Microsoft2023l}}
    \label{AADMSG}
\end{table}

Uit \ref{AADMSG} blijkt dat Microsoft Graph de meerderheid van de Azure \ac{AD} data-objecten overneemt. Toch worden er enkele verschillen opgemerkt uit deze migratie. Zowel de Application Proxy Application Management en Application Proxy Connector Management vallen weg. Daarnaast is er een commando weggehaald uit het AzureAD en Contacts data-object. Vervolgens zijn er twee commando's minder bij het Devices data-object. Finaal zijn er 21 commando's weggehaald uit het MSOnline data-object in Microsoft Graph. \\

Naast de data-objecten bevat Microsoft Graph meer dependencies, dit wordt weergegeven in Tabel \ref{AADT}. Dit wilt zeggen dat Microsoft Graph meer Microsoft-entiteiten kan aanspreken dan de Azure \ac{AD} PowerShell-module. Azure \ac{AD} is dan ook gefocust op het beheren van Azure \ac{AD}, terwijl Microsoft Graph vandaag de dag ook Microsoft Teams, Outlook, To Do en andere entiteiten kan aanspreken naast Azure \ac{AD}. 

% ---
% Section
% ---

\section{Gebruik}

\subsection{Toegangsmogelijkheden}

%Idee: Via welke methodes kan men zich inloggen? 

%Vb. Is er een verschil? Wordt er MFA ondersteund? ... (Gaat nog niet in detail over veiligheid -> zie Security)

Voor beide technologieën bestaan de twee dezelfde toegangsmogelijkheden, namelijk delegated en app-only. Er zijn geen verschillen aan te merken voor de toegangsmogelijkheden. \\

De enige opmerking dat er kan gegeven worden, maar buiten de scope van de vergelijking valt, is dat \ac{ADAL} niet meer ondersteunt wordt op 30 juni 2023. Wanneer er wordt gebruikgemaakt van een library, bijvoorbeeld voor app-only access met secrets, dan kan dit via \ac{MSAL}.

\subsection{Ondersteunende programmeertalen}

% TODO: Is dit nog relevant? Want veel VERSCHILLENDE info over gevonden, niet zeker...

% Idee: Waar of hoe kan men de technologie aanspreken of gebruiken?

% Vb. Moet die manueel gebeuren? Is dit ook mogelijk via scripts? Kan dit via PowerShell of moet je in Azure zitten...

% manueel werk, script werk... => verschil van werking, eventueel PowerShell aantoetsen, Java? ...

Naast de toegangsmogelijkheden is het ook nodig om te weten met welke programmeertalen of methodes beide technologieën kunnen aangesproken worden. In dit onderdeel worden \Ac{HTTP}, \Ac{JSON} en \Ac{XML} niet beschouwd als programmeertalen. \\

Bij het oude kan er worden gebruikgemaakt van de programmeertalen C\#, Java, JavaScript, ObjC, PHP, Python, Ruby. Als tweede de MSOnline en AzureAD PowerShell-modules om Azure \ac{AD} Graph aan te sturen met PowerShell. \\

Bij het nieuwe gaat dit verder dan alleen \ac{HTTP} en PowerShell. Aan de hand van de Microsoft Graph \ac{SDK} kunnen ook de programmeertalen C\#, Go, Java, JavaScript en PHP gebruikt worden. \\

% ---
% Section
% ---

\section{Security}

\subsection{Veiligheid van de toegangsmogelijkheden}

De eerste methode is de gedelegeerde of interactieve manier. Deze methode maakt gebruik van een interactie van een gebruiker om de data te kunnen raadplegen. Hier ligt dus de nadruk op de gebruiker die wilt inloggen. \\

De gebruiker die de actie uitvoert heeft bepaalde rechten verkregen. Dit kan gaan van leesrechten, maar ook om schrijfrechten. De veiligheid van de data wordt bepaald door twee factoren. \\

Enerzijds de rechten van de gebruiker. Een beheerder heeft normalerwijze veel rechten om beheertaken te kunnen uitvoeren, zoals het lezen of schrijven van data. Wanneer er wordt ingelogd met een beheerder zijn de mogelijke consequenties ook groter, ten opzichte van een gebruiker die alleen maar data kan opvragen. \\ 

Anderzijds de getroffen methodes om het aanmelden mogelijk te maken. Als de gebruiker alleen maar een naam en wachtwoord moet meegeven, is de kans op een inbreuk groter dan bij een gebruiker die gebruikmaakt van \ac{2FA} of \Ac{MFA}. \\

De tweede methode is de geprogrammeerde manier of niet-interactieve manier. Deze methode maakt geen of minder gebruik van interacties in tegenstelling tot de eerste manier. De nadruk ligt op de applicatie die gebruikt wordt om toegang te verschaffen. \\

De applicatie heeft net zoals een gebruiker rechten die kunnen worden ingesteld. Deze applicatie staat los van de gebruikers en kan alleen worden aangepast door een gebruiker die toegang heeft tot het domein. \\

Een applicatie kan niet gebruikmaken van \ac{MFA}, doordat er geen interacties plaatsvinden. In plaats daarvan is het mogelijk om gebruik te maken van secrets. Het principe van een secret is dat de waarde alleen bij het aanmaken wordt weergegeven, daarna kan het niet meer worden opgevraagd.

\subsection{Gebruik van rechten}

% Idee: Het beginsel van de minste voorrechten (PoLP) wordt toegepast. Is dit bij beide zo en hoe sterk is dit principe?

% Vb. Leg PoLP dieper uit, wat kan het tegenhouden, waarom wordt het toegepast...

Beide technologieën maken gebruik van rechten tot groepen van data. Standaard krijgt een gebruiker geen rechten en moeten deze worden verkregen via een gemachtigde gebruiker. Door deze aanpak wordt het “Principle of Least Privilege” toegepast. \\

Azure \ac{AD} kan minder Microsoft-entiteiten aanspreken in vergelijking met Microsoft Graph. 
%%=============================================================================
%% Proof-of-Concept
%%=============================================================================

\chapter{Proof-of-Concept}%
\label{ch:poc}



\lipsum[78-80]

\section{Testomgeving}

Voor de Proof-of-Concept wordt er gebruiktgemaakt van een testomgeving. Deze testomgeving werd opgesteld via een gratis Azure Active Directory developer tenant. 

\subsection{Wat is een Azure Active Directory developer tenant?}

Een Azure \ac{AD} developer tenant is een gratis Sandbox-omgeving dat voor 90 dagen beschikbaar wordt gesteld. Een Sandbox is in IT-termen een omgeving waarin je kan spelen met een of meerdere technologieën zonder dat hier consequenties aan vast hangen.  

% TODO: Bron zoeken over Sandboxes en over De Azure AD tenant

% TODO: Stappen oplijsten + figuren/images toevoegen van elke stap?

% TODO: kijk best naar een andere BP hiervoor


\section{Uitwerking}

\lipsum[76-80]
%%=============================================================================
%% Conclusie
%%=============================================================================

\chapter{Conclusie}%
\label{ch:conclusie}

% TODO: Trek een duidelijke conclusie, in de vorm van een antwoord op de
% onderzoeksvra(a)g(en). Wat was jouw bijdrage aan het onderzoeksdomein en
% hoe biedt dit meerwaarde aan het vakgebied/doelgroep? 
% Reflecteer kritisch over het resultaat. In Engelse teksten wordt deze sectie
% ``Discussion'' genoemd. Had je deze uitkomst verwacht? Zijn er zaken die nog
% niet duidelijk zijn?
% Heeft het onderzoek geleid tot nieuwe vragen die uitnodigen tot verder 
%onderzoek?


% ONDERDEEL: Azure \ac{AD} is dan ook gefocust op het beheren van Azure \ac{AD}, terwijl Microsoft Graph vandaag de dag ook Microsoft Teams, Outlook, To Do en andere entiteiten kan aanspreken naast Azure \ac{AD}. 

% Azure \ac{AD} kan minder Microsoft-entiteiten aanspreken in vergelijking met Microsoft Graph. Dit leidt tot minder beschikbare permissie scopes, zoals te zien is in Tabel \ref{psaad}. \\

% Microsoft Graph heeft meer dan 100 verschillende permissie scope onderdelen of domeinen. 



%---------- Bijlagen -----------------------------------------------------------

\appendix

\chapter{Onderzoeksvoorstel}

Het onderwerp van deze bachelorproef is gebaseerd op een onderzoeksvoorstel dat vooraf werd beoordeeld door de promotor. Dat voorstel is opgenomen in deze bijlage.

% TODO: 
\section*{Samenvatting}

De visie van Microsoft voor een centraal ecosysteem leidt tot het vervangen van verouderde technologie met nieuwe state of the art technologie. De bestaande Azure Active Directory Graph API moet plaatsmaken voor de opkomende Microsoft Graph. Deze aanpassing leidt tot verouderde scripts bij het IT-bedrijf Easi, die dagelijks met de uitfaserende Azure AD modules werkt. Microsoft Graph is een universele REST API, die als eindpunt dient voor Microsoft 365 tenants. Voor productieomgevingen is Microsoft Graph niet geschikt. Met dit onderzoek worden de mogelijkheden van Graph geïllustreerd. Deze casus wordt onderzocht aan de hand van twee fases. In de eerste fase wordt een vergelijkende studie tussen de oude module en de nieuwe API opgesteld. Ten slotte wordt een Proof-of-Concept uitgewerkt. Deze fase gaat na of een Microsoft Graph audit-script de verouderde modules kan vervangen. Er wordt verwacht dat Microsoft Graph, vandaag de dag, niet stabiel is voor productieomgevingen met Microsoft 365 tenants. Er wordt aangenomen dat Microsoft Graph verbeteringen en meerwaarde met zich meebrengt aan de hand van een nieuw audit-script. Daarnaast heeft Graph ook meer voordelen op vlak van logica, ondersteuning, gebruik en security in vergelijking met de huidige Azure AD modules.

% Verwijzing naar het bestand met de inhoud van het onderzoeksvoorstel
%---------- Inleiding ---------------------------------------------------------

\section{Introductie}%
\label{sec:introductie}

\begin{comment}
Waarover zal je bachelorproef gaan? Introduceer het thema en zorg dat volgende zaken zeker duidelijk aanwezig zijn:

\begin{itemize}
  \item kaderen thema
  \item de doelgroep
  \item de probleemstelling en (centrale) onderzoeksvraag
  \item de onderzoeksdoelstelling
\end{itemize}

Denk er aan: een typische bachelorproef is \textit{toegepast onderzoek}, wat betekent dat je start vanuit een concrete probleemsituatie in bedrijfscontext, een \textbf{casus}. Het is belangrijk om je onderwerp goed af te bakenen: je gaat voor die \textit{ene specifieke probleemsituatie} op zoek naar een goede oplossing, op basis van de huidige kennis in het vakgebied.

De doelgroep moet ook concreet en duidelijk zijn, dus geen algemene of vaag gedefinieerde groepen zoals \emph{bedrijven}, \emph{developers}, \emph{Vlamingen}, enz. Je richt je in elk geval op it-professionals, een bachelorproef is geen populariserende tekst. Eén specifiek bedrijf (die te maken hebben met een concrete probleemsituatie) is dus beter dan \emph{bedrijven} in het algemeen.

Formuleer duidelijk de onderzoeksvraag! De begeleiders lezen nog steeds te veel voorstellen waarin we geen onderzoeksvraag terugvinden.

Schrijf ook iets over de doelstelling. Wat zie je als het concrete eindresultaat van je onderzoek, naast de uitgeschreven scriptie? Is het een proof-of-concept, een rapport met aanbevelingen, \ldots Met welk eindresultaat kan je je bachelorproef als een succes beschouwen?

\end{comment}

Naar aanleiding van de snel evoluerende cloudtechnologie, focust Microsoft zich steeds meer op het bouwen van een ecosysteem dat bestuurd kan worden vanuit zoveel mogelijk platformen \autocite{Parker2021}.

Deze gedachtegang van Microsoft brengt volgende wijziging met zich mee, namelijk het uitfaseren van de bestaande Azure AD (Active Directory) Graph API (Application Programming Interface) en ADAL (Azure AD Authentication Library) \autocite{Sahay2022}. Deze wijziging staat gepland om op 30 juni 2023 op Microsoft 365 tenants doorgevoerd te worden.

Bij het IT-bedrijf Easi worden er courante taken, automatisaties en wederkerende acties uitgevoerd bij klanten met behulp van de vooraf vernoemde Azure AD PowerShell modules. Hierdoor is het bedrijf genoodzaakt om een oplossing te vinden voor bovenstaand probleem.

Een mogelijke oplossing is het gebruik van MSAL (Microsoft Authentication Library) en Microsoft Graph API \autocite{Microsoft2023}. Microsoft Graph is nog volop in ontwikkeling, waardoor er nog geen zekerheid bestaat om dit in te zetten bij klanten.

Dit onderzoek zal zich focussen op wat er vandaag de dag mogelijk is binnen Microsoft Graph. Het doel is om een antwoord te geven op de vraag of Microsoft Graph klaar is om gebruikt te worden als vervanger van de Azure AD PowerShell modules. Het antwoord op de vraag wordt onderbouwd door een bijhorende vergelijkende studie en Proof-of-Concept die de mogelijkheden van de technologie illustreert.


%---------- Stand van zaken ---------------------------------------------------

\section{State-of-the-art}%
\label{sec:state-of-the-art}

\begin{comment}

Hier beschrijf je de \emph{state-of-the-art} rondom je gekozen onderzoeksdomein, d.w.z.\ een inleidende, doorlopende tekst over het onderzoeksdomein van je bachelorproef. Je steunt daarbij heel sterk op de professionele \emph{vakliteratuur}, en niet zozeer op populariserende teksten voor een breed publiek. Wat is de huidige stand van zaken in dit domein, en wat zijn nog eventuele open vragen (die misschien de aanleiding waren tot je onderzoeksvraag!)?

Je mag de titel van deze sectie ook aanpassen (literatuurstudie, stand van zaken, enz.). Zijn er al gelijkaardige onderzoeken gevoerd? Wat concluderen ze? Wat is het verschil met jouw onderzoek?

Verwijs bij elke introductie van een term of bewering over het domein naar de vakliteratuur, bijvoorbeeld~\autocite{Hykes2013}! Denk zeker goed na welke werken je refereert en waarom.

Draag zorg voor correcte literatuurverwijzingen! Een bronvermelding hoort thuis \emph{binnen} de zin waar je je op die bron baseert, dus niet er buiten! Maak meteen een verwijzing als je gebruik maakt van een bron. Doe dit dus \emph{niet} aan het einde van een lange paragraaf. Baseer nooit teveel aansluitende tekst op eenzelfde bron.

Als je informatie over bronnen verzamelt in JabRef, zorg er dan voor dat alle nodige info aanwezig is om de bron terug te vinden (zoals uitvoerig besproken in de lessen Research Methods).

% Voor literatuurverwijzingen zijn er twee belangrijke commando's:
% \autocite{KEY} => (Auteur, jaartal) Gebruik dit als de naam van de auteur
%   geen onderdeel is van de zin.
% \textcite{KEY} => Auteur (jaartal)  Gebruik dit als de auteursnaam wel een
%   functie heeft in de zin (bv. ``Uit onderzoek door Doll & Hill (1954) bleek
%   ...'')

Je mag deze sectie nog verder onderverdelen in subsecties als dit de structuur van de tekst kan verduidelijken.



AWS \autocite{AWS2022}, de grootste Cloud Provider volgens \textcite{Vailshery2022} en \textcite{SRG2022}, geeft de mogelijkheid tot automatisatie van Cloud Computing Services op hun platform. Hierdoor zijn er veel onderzoeken binnen AWS uitgevoerd omtrent automatisatie van virtuele-omgevingen. Dit onderzoek focust zich op Azure, de grootste concurrent van AWS op dit moment.

Azure biedt ondersteuning aan voor verschillende niet Microsoft-gerelateerde Infrastructure Automation tools. Configuration Management tools zoals Ansible, Chef en Puppet werden al onderzocht en onderling vergeleken in andere studies \autocite{Microsoft2022a}. 

Ansible biedt goede ondersteuning aan op \newline lange termijn en is simpel in gebruik met goede documentatie. Daarnaast is Ansible een krachtige tool met veel potentieel \autocite{Masek2018}. Het wordt eerder aangeraden voor kleinere projecten door zijn onvolledigheid. Waarbij Chef en Puppet meer compleet zijn en grote projecten beter kan verwerken \autocite{Bertram2016}. 

Naast bovenstaande tools ondersteunt Azure Orchestration tools waaronder Terraform. Deze tool is onveranderlijk ten opzichte van Configuration Management tools. Terraform wordt gebruikt om machines te orkestreren wat eenvoudiger is om meerdere instanties te voorzien. Terwijl Configuration Management tools zich focussen op het configureren van instanties \autocite{Brikman2016}.

Binnen Hogeschool Gent zijn er reeds een aantal studies uitgevoerd naar aspecten die voorkomen in dit onderwerp. Kelvin \textcite{Vermeulen2021} heeft een onderzoek uitgevoerd naar het automatiseren van een Public Cloud-omgeving binnen AWS. Daarnaast heeft Joachim \textcite{VandeKeere2021} een studie uitgevoerd naar het gebruik van Ansible binnen lokale omgevingen. 

In het eerste onderzoek van Vermeulen werd de configuratie van AWS manueel uitgevoerd. Vervolgens werd in het tweede onderzoek van Van der Keere Ansible gebruikt voor een lokale \newline Windows- en Linux-omgeving. In deze studie worden de configuraties van Azure geautomatiseerd. Verder wordt er een Linux-testomgeving opgezet in de Cloud via Ansible.

\end{comment}

Microsoft Azure is de voorbije jaren de grootste concurrent van AWS binnen Cloud Computing Services, blijkt uit onderzoek van \textcite{Vailshery2022} en \textcite{SRG2022}. Het platform biedt verschillende services aan waaronder Azure Active Directory. Azure AD kan beheerd worden vanuit bijhorende PowerShell modules, maar de ondersteuning voor deze modules wordt stopgezet, zoals vermeld in de introductie. 

Microsoft 365 applicaties en services hebben een eigen unieke API om data te raadplegen. In projecten of applicaties waarbij meerdere Microsoft 365 services moeten aangesproken worden, kan dit voor problemen zorgen. Bijvoorbeeld, de API van Azure AD heeft andere functies en werkwijzen in vergelijking met een SharePoint API \autocite{VanRousselt2021}. Deze verschillen leiden tot inefficiëntie. Een mogelijke oplossing is een universele API die alle services kan aanspreken.

Microsoft Graph is een universele API dat toegepast kan worden voor bovenstaande problemen. Graph is een toegangspoort om data te kunnen raadplegen in Microsoft 365 applicaties en services. Deze toegangspoort is een universele API dat gebruikmaakt van REST (Representational State Transfer). Het idee hierachter is om een eindpunt aan te bieden, waardoor de verschillende applicaties en services gebundeld worden tot een samenhang. 

Uit onderzoek bleek dat Microsoft Graph stabiel genoeg is om specifieke taken binnen Microsoft 365 applicaties en services uit te voeren, ondanks nog steeds in de ontwerpfase te zitten. \textcite{Hoefling2022} maakte gebruik van Graph om OneDrive data te kunnen raadplegen. \textcite{Jenkins2021} illustreerde het gebruik van Graph in combinatie met Teams. Daarnaast bewezen \textcite{Parsa2019} de eenvoud van de Graph API om Outlook data te verwerken in een kamerbeheersysteem. 

Deze studie legt de focus op Azure Active Dirctory, een Microsoft Cloud-service dat nog niet in onderzoek werd verwerkt met Microsoft Graph. 

%---------- Methodologie ------------------------------------------------------
\section{Methodologie}%
\label{sec:methodologie}

\begin{comment}

Hier beschrijf je hoe je van plan bent het onderzoek te voeren. Welke onderzoekstechniek ga je toepassen om elk van je onderzoeksvragen te beantwoorden? Gebruik je hiervoor literatuurstudie, interviews met belanghebbenden (bv.~voor requirements-analyse), experimenten, simulaties, vergelijkende studie, risico-analyse, PoC, \ldots?

Valt je onderwerp onder één van de typische soorten bachelorproeven die besproken zijn in de lessen Research Methods (bv.\ vergelijkende studie of risico-analyse)? Zorg er dan ook voor dat we duidelijk de verschillende stappen terug vinden die we verwachten in dit soort onderzoek!

Vermijd onderzoekstechnieken die geen objectieve, meetbare resultaten kunnen opleveren. Enquêtes, bijvoorbeeld, zijn voor een bachelorproef informatica meestal \textbf{niet geschikt}. De antwoorden zijn eerder meningen dan feiten en in de praktijk blijkt het ook bijzonder moeilijk om voldoende respondenten te vinden. Studenten die een enquête willen voeren, hebben meestal ook geen goede definitie van de populatie, waardoor ook niet kan aangetoond worden dat eventuele resultaten representatief zijn.

Uit dit onderdeel moet duidelijk naar voor komen dat je bachelorproef ook technisch voldoen\-de diepgang zal bevatten. Het zou niet kloppen als een bachelorproef informatica ook door bv.\ een student marketing zou kunnen uitgevoerd worden.

Je beschrijft ook al welke tools (hardware, software, diensten, \ldots) je denkt hiervoor te gebruiken of te ontwikkelen.

Probeer ook een tijdschatting te maken. Hoe lang zal je met elke fase van je onderzoek bezig zijn en wat zijn de concrete \emph{deliverables} in elke fase?



Het onderzoek omvat vier fases. In de eerste fase wordt er een vergelijkende studie uitgevoerd. Deze tools worden onder de loep genomen en vergeleken op vlak van logica, leesbaarheid, verwerkingssnelheid, documentatie, community en eventuele limieten. Er wordt zowel gebruikgemaakt van vakliteratuur als simulaties binnen Azure. Achteraf volgt er een verklarende tekst die alle resultaten en bevindingen op papier zet. 

De tweede fase bestaat uit een Proof-of-\newline{}Concept die de werking van de Infrastructure Automation tools illustreert. De tools worden gebruikt om de configuraties van Azure te automatiseren om vervolgens te kunnen gebruiken voor een virtuele omgeving. Nadien worden deze uitgewerkt tot scripts via de Azure CLI (Command Line Interface). Deze scripts worden vervolgens toegevoegd in de scriptie met eventuele bevindingen. Deze fase is doorslaggevend voor de casus en de onderzoeksvraag.

In de derde fase volgt een nabootsing van een realistische Linux-omgeving. Aan de hand van geautomatiseerde configuraties ontstaat de mogelijkheid om een omgeving op te zetten. Hierbij wordt Ansible als Configuration Management tool ingezet voor het opzetten van de infrastructuur. In de testomgeving voorzien we vier virtuele servers en een virtuele client. Waarbij alle virtuele machines AlmaLinux \autocite{AOF2022} als distributie gebruiken. De eerste server maakt gebruik van BIND \autocite{ISC2022} om de DNS (Domain Name System) te voorzien. De tweede server voorziet DHCP (Dynamic Host Configuration Protocol) met dynamische IP-adressen voor de virtuele client. De derde server fungeert als webserver en databank met een \textcite{WordPress2022} installatie. De laatste server maakt gebruik van Prometheus \autocite{PrometheusAuthors2022} en Grafana \autocite{GrafanaLabs2022} om de servers te kunnen monitoren. De client is een gebruiker van het domein en test de beschikbare functionaliteiten.

De laatste fase omvat een Disaster Recovery-scenario waarbij de omgeving wordt beschadigd. Door deze beschadiging wordt de omgeving nogmaals opgezet aan de hand van automatisatie uit de tweede fase. Er wordt gekeken naar welke impact de automatisatie heeft in vergelijking met de manuele manier van configureren. Deze impact wordt geanalyseerd en getest, nadien worden alle bevinden verwerkt in de scriptie.

\end{comment}

In dit onderzoek wordt er in twee fases gewerkt.
 
In de eerste fase wordt er een vergelijkende studie uitgevoerd tussen de uitfaserende Azure AD PowerShell modules en Microsoft Graph. Deze twee technologieën worden vergeleken op vlak van achterliggende logica, ondersteunende tenants, gebruik en security. Deze vergelijking verduidelijkt de evolutie van de PowerShell modules (Azure AD) naar de API (Graph).

Eerst wordt er gekeken naar de achterliggende logica van beide technologieën. De focus ligt op de werking en communicatie met Microsoft 365 tenants. 

Ten tweede worden de ondersteunde Microsoft 365 tenants onder de loep genomen. Bij dit onderdeel ligt de nadruk op welke Microsoft 365 applicaties en services kunnen aangesproken worden met beide technologieën.

Vervolgens wordt het gebruik besproken. Dit wilt zeggen, hoe zien beide technologieën eruit op vlak van code en notatie. Dit is belangrijk om te begrijpen hoe beiden geautomatiseerd worden en hoe de complexiteit van een script er kan uitzien.

Als laatste worden de PowerShell modules en Graph vergeleken op security. De focus wordt gelegd op welke data beide technologieën verwerken en nodig hebben. Bovendien wordt er gekeken naar mogelijke veiligheidsrisico's. Dit zal bepalen hoe veilig het gebruik van beide technologieën zijn.

De tweede fase bestaat uit een Proof-of-Concept die de werking van Microsoft Graph illustreert. Voor deze uitwerking wordt er gebruikgemaakt van een reeds bestaand PowerShell-script. Dit script omvat het genereren van Office 365 Security Audit documenten voor klanten van Easi. Het genereren van deze Audit-data wordt via Azure AD PowerShell modules voorzien, waardoor deze kwetsbaar is voor de uitfasering. 

Als Proof-of-Concept wordt bovenstaand script omgevormd en uitgewerkt met Microsoft Graph. Het doel van deze uitwerking is om te bewijzen wat Microsoft Graph vandaag de dag al kan en wat nog niet. Bovendien verduidelijkt deze fase de evolutie van Azure AD Graph modules naar de Graph API op een praktische manier. Daarnaast kan deze uitwerking gebruikt worden als basis voor andere toepassingen met Graph zoals Microsoft Purview \autocite{Microsoft2023a}. 

%---------- Verwachte resultaten ----------------------------------------------
\section{Verwacht resultaat, conclusie}%
\label{sec:verwachte_resultaten}

\begin{comment}

Hier beschrijf je welke resultaten je verwacht. Als je metingen en simulaties uitvoert, kan je hier al mock-ups maken van de grafieken samen met de verwachte conclusies. Benoem zeker al je assen en de onderdelen van de grafiek die je gaat gebruiken. Dit zorgt ervoor dat je concreet weet welk soort data je moet verzamelen en hoe je die moet meten.

Wat heeft de doelgroep van je onderzoek aan het resultaat? Op welke manier zorgt jouw bachelorproef voor een meerwaarde?

Hier beschrijf je wat je verwacht uit je onderzoek, met de motivatie waarom. Het is \textbf{niet} erg indien uit je onderzoek andere resultaten en conclusies vloeien dan dat je hier beschrijft: het is dan juist interessant om te onderzoeken waarom jouw hypothesen niet overeenkomen met de resultaten.



Er wordt verwacht dat alle niet Microsoft-\newline{}gerelateerde Infrastructure Automation tools de nodige Azure configuraties kunnen opzetten voor een testomgeving. Hoewel Ansible en Terraform een opmerkelijk beter potentieel hebben ten opzichte van de andere tools. Dit gaat gepaard met meer mogelijkheden binnen Azure. In vergelijking met de andere tools, zitten Ansible en Terraform in een continue ontwikkelingscyclus door het jonge toetreden tot de IT-sector. Beide tools scoren goed op vlak van logica, leesbaarheid, documentatie, community en limieten. In tegenstelling tot de andere tools die eerder op vlak van verwerkingssnelheid domineren. 

Vervolgens wordt er uit het onderzoek verwacht dat het automatiseren van Azure configuraties alleen maar voordelen heeft. Het enige nadeel in verband met de Infrastructure Automation tools is het leerproces om de tools onder de knie te hebben. Toch wordt dit ook als een voordeel gezien, omwille de tools buiten Azure populair zijn en ondersteund worden. Door het gebruik van deze automatisatie wordt de efficiëntie aanzienlijk verhoogd. Vervolgens biedt het de mogelijkheid tot Disaster Recovery aan. Daarbovenop is de kans tot menselijke fouten opmerkelijk verkleind.

\end{comment}

Er wordt verwacht dat Microsoft Graph in de vergelijking met Azure AD Graph verbeterd is op de vier onderdelen. De logica van Graph is anders, maar wordt langer ondersteunend door het gebruik van REST. Graph ondersteunend meer tenants, dus meer Microsoft 365 applicaties en services. Het gebruik van Graph is schaalbaarder en efficiënter, doormiddel van universele functies en minder verschillende soorten API’s. De security van Graph is veiliger door een zero-trust beleid dat wordt toegepast bij het gebruiken van de Microsoft Graph API.

Vervolgens wordt er verwacht dat Microsoft Graph, op dit moment, niet stabiel genoeg is om alle Azure AD PowerShell modules te vervangen tegen de uitfaseringsdeadline van 30 juni 2023. Bovendien is Microsoft Graph, op dit moment, niet aan te raden in cruciale productieomgevingen van klanten. Echter wordt er wel verwacht dat Microsoft Graph de Proof-of-Concept in verband met auditing kan uitwerken. Hierdoor is het stabiel genoeg om niet-bedrijfskritische toepassingen uit te werken.



\chapter{Request- en Response-queries}%
\label{ch:HTTP-queries}

\clearpage

\section{Query 1: Opvragen domeinen}

\subsection{Azure AD Graph}

\subsubsection{Request}

\begin{itemize}
    \item \Ac{HTTP}-method: GET
    \item \ac{URL}: https://graph.windows.net/ffa43659-...-838af35d1353/domains
    \item \Ac{API}-version: 1.6
\end{itemize}

\subsubsection{Response body}

\begin{listing}[!h]
    \begin{minted}
        [
        frame=lines,
        framesep=2mm,
        baselinestretch=1.2,
        fontsize=\scriptsize,
        linenos
        ]  
        {json}
{
    "odata.metadata": "https://graph.windows.net/ffa43659-...-838af35d1353/$metadata#domains",
    "value": [
    {
        "authenticationType": "Managed",
        "availabilityStatus": null,
        "isAdminManaged": true,
        "isDefault": true,
        "isDefaultForCloudRedirections": false,
        "isInitial": true,
        "isRoot": true,
        "isVerified": true,
        "name": "25ky3d.onmicrosoft.com",
        "supportedServices": ["Email", "OfficeCommunicationsOnline"],
        "forceDeleteState": null,
        "state": null,
        "passwordValidityPeriodInDays": 2147483647,
        "passwordNotificationWindowInDays": 14
    }
    ]
}
    \end{minted}
    \caption[Query 1: Response body Azure AD Graph]{Response body van de eerste query via Azure \ac{AD} Graph Explorer.}
    \label{Q1AADRB}
\end{listing}

%\clearpage

\subsubsection{Response header}

\begin{listing}[ht]
    \begin{minted}
        [
        frame=lines,
        framesep=2mm,
        baselinestretch=1.2,
        fontsize=\scriptsize,
        linenos
        ]  
        {json}
{
    "cache-control": "no-cache",
    "client-request-id": "82bc7ed4-db42-414f-8260-73754bca7ce1",
    "content-length": "504",
    "content-type": "application/json; odata=minimalmetadata; streaming=true; charset=utf-8",
    "expires": "-1",
    "ocp-aad-session-key": "fdBtvCAl-..._n1UQvQe9reWesQHDTlUo",
    "pragma": "no-cache",
    "request-id": "dfd33cc1-ca0e-4964-ae48-0d3f6f222514"
}
    \end{minted}
    \caption[Query 1: Response header Azure AD Graph]{Response header van de eerste query via Azure \ac{AD} Graph Explorer.}
    \label{Q1AADRH}
\end{listing}

\subsection{Microsoft Graph}

\subsubsection{Request}

\begin{itemize}
    \item \Ac{HTTP}-method: GET
    \item \Ac{API}-version: 1.0
    \item \ac{URL}: https://graph.microsoft.com/v1.0/domains
\end{itemize}

\subsubsection{Response body}

\begin{listing}[!h]
    \begin{minted}
        [
        frame=lines,
        framesep=2mm,
        baselinestretch=1.2,
        fontsize=\scriptsize,
        linenos
        ]  
        {json}
{
    "@odata.context": "https://graph.microsoft.com/v1.0/$metadata#domains",
    "value": [
    {
        "authenticationType": "Managed",
        "availabilityStatus": null,
        "id": "25ky3d.onmicrosoft.com",
        "isAdminManaged": true,
        "isDefault": true,
        "isInitial": true,
        "isRoot": true,
        "isVerified": true,
        "supportedServices": ["Email", "OfficeCommunicationsOnline"],
        "passwordValidityPeriodInDays": 2147483647,
        "passwordNotificationWindowInDays": 14,
        "state": null
    }
    ]
}
    \end{minted}
    \caption[Query 1: Response body Microsoft Graph]{Response body van de eerste query via Microsoft Graph Explorer.}
    \label{Q1MSGRB}
\end{listing}

\subsubsection{Response header}

\begin{listing}[!h]
    \begin{minted}
        [
        frame=lines,
        framesep=2mm,
        baselinestretch=1.2,
        fontsize=\scriptsize,
        linenos
        ]  
        {json}
{
    "cache-control": "no-cache",
    "client-request-id": "45bd0201-69d2-f78c-cee8-aaedfce8b3d8",
    "content-type": "application/json;odata.metadata=minimal;odata.streaming=true;
        IEEE754Compatible=false;charset=utf-8",
    "request-id": "0f814559-7945-48c7-bb2c-a5929a768b32"
}
    \end{minted}
    \caption[Query 1: Response header Microsoft Graph]{Response van de eerste query via Microsoft Graph Explorer.}
    \label{Q1MSGRH}
\end{listing}

\clearpage

\section{Query 2: Gebruiker aanmaken}

\subsection{Azure AD Graph}

\subsubsection{Request}

\begin{itemize}
    \item \Ac{HTTP}-method: POST
    \item \ac{URL}: https://graph.windows.net/ffa43659-...-838af35d1353/users
    \item \Ac{API}-version: 1.6
\end{itemize}

\subsubsection{Request body}

\begin{listing}[!h]
    \begin{minted}
        [
        frame=lines,
        framesep=2mm,
        baselinestretch=1.2,
        fontsize=\scriptsize,
        linenos
        ]  
        {json}
{
    "accountEnabled": true,
    "displayName": "Alex Wu",
    "mailNickname": "AlexW",
    "passwordProfile": {
        "password": "Test1234",
        "forceChangePasswordNextLogin": false
    },
    "userPrincipalName": "Alex@25ky3d.onmicrosoft.com"
}
    \end{minted}
    \caption[Query 2: Request body Microsoft Graph]{Request body van de tweede query via Azure \Ac{AD} Graph Explorer.}
    \label{Q2AADRQB}
\end{listing}

\clearpage

\subsubsection{Response body}

\begin{listing}[!h]
    \begin{minted}
        [
        frame=lines,
        framesep=2mm,
        baselinestretch=1.2,
        fontsize=\tiny,
        linenos
        ]  
        {json}
{
    "odata.metadata":  "https://graph.windows.net/ffa43659-...-838af35d1353/$metadata#directoryObjects/@Element",
    "odata.type": "Microsoft.DirectoryServices.User",
    "objectType": "User",
    "objectId": "5c9ab483-c9b4-4ab3-a857-4763c055ef53",
    "deletionTimestamp": null,
    "accountEnabled": true,
    "ageGroup": null,
    "assignedLicenses": [],
    "assignedPlans": [],
    "city": null,
    "companyName": null,
    "consentProvidedForMinor": null,
    "country": null,
    "createdDateTime": null,
    "creationType": null,
    "department": null,
    "dirSyncEnabled": null,
    "displayName": "Alex Wu",
    "employeeId": null,
    "facsimileTelephoneNumber": null,
    "givenName": null,
    "immutableId": null,
    "isCompromised": null,
    "jobTitle": null,
    "lastDirSyncTime": null,
    "legalAgeGroupClassification": null,
    "mail": null,
    "mailNickname": "AlexW",
    "mobile": null,
    "onPremisesDistinguishedName": null,
    "onPremisesSecurityIdentifier": null,
    "otherMails": [],
    "passwordPolicies": null,
    "passwordProfile": null,
    "physicalDeliveryOfficeName": null,
    "postalCode": null,
    "preferredLanguage": null,
    "provisionedPlans": [],
    "provisioningErrors": [],
    "proxyAddresses": [],
    "refreshTokensValidFromDateTime": "2023-05-06T10:53:45.6717827Z",
    "showInAddressList": null,
    "signInNames": [],
    "sipProxyAddress": null,
    "state": null,
    "streetAddress": null,
    "surname": null,
    "telephoneNumber": null,
    "usageLocation": null,
    "userIdentities": [],
    "userPrincipalName": "Alex@25ky3d.onmicrosoft.com",
    "userState": null,
    "userStateChangedOn": null,
    "userType": "Member"
}
    \end{minted}
    \caption[Query 2: Response body Microsoft Graph]{Response body van de tweede query via Azure \Ac{AD} Graph Explorer.}
    \label{Q2AADRB}
\end{listing}

\clearpage

\subsubsection{Response header}

\begin{listing}[!h]
    \begin{minted}
        [
        frame=lines,
        framesep=2mm,
        baselinestretch=1.2,
        fontsize=\scriptsize,
        linenos
        ]  
        {json}
{
    "cache-control": "no-cache",
    "client-request-id": "b47a729f-c944-44a1-a6a3-5f9d84b0b73f",
    "content-length": "1387",
    "content-type": "application/json; odata=minimalmetadata; streaming=true; charset=utf-8",
    "expires": "-1",
    "ocp-aad-session-key": "EYQrEhc2N...S1buL6cO8",
    "pragma": "no-cache",
    "request-id": "40d52611-5733-411b-aa59-1e76643a6ff4"
}
    \end{minted}
    \caption[Query 2: Response body Microsoft Graph]{Response body van de tweede query via Azure \Ac{AD} Graph Explorer.}
    \label{Q2AADRH}
\end{listing}

\subsection{Microsoft Graph}

\subsubsection{Request}

\begin{itemize}
    \item \Ac{HTTP}-method: POST
    \item \ac{API}-version: 1.0
    \item \Ac{URL}: https://graph.microsoft.com/v1.0/users
\end{itemize}

\subsubsection{Request body}

\begin{listing}[!h]
    \begin{minted}
        [
        frame=lines,
        framesep=2mm,
        baselinestretch=1.2,
        fontsize=\scriptsize,
        linenos
        ]  
        {json}
{
    "accountEnabled": true,
    "displayName": "Alex Wu",
    "mailNickname": "AlexW",
    "passwordProfile": {
        "password": "Test1234",
        "forceChangePasswordNextSignin": false
    },
    "userPrincipalName": "Alex@25ky3d.onmicrosoft.com"
}
    \end{minted}
    \caption[Query 2: Request body Microsoft Graph]{Request body van de tweede query via Microsoft Graph Explorer.}
    \label{Q2MSGRQB}
\end{listing}

\clearpage

\subsubsection{Response body}

\begin{listing}[!h]
    \begin{minted}
        [
        frame=lines,
        framesep=2mm,
        baselinestretch=1.2,
        fontsize=\scriptsize,
        linenos
        ]  
        {json}
{
    "@odata.context": "https://graph.microsoft.com/v1.0/$metadata#users/$entity",
    "id": "bc521ac5-ab90-4427-880d-c1c52ffe3ee7",
    "businessPhones": [],
    "displayName": "Alex Wu",
    "givenName": null,
    "jobTitle": null,
    "mail": null,
    "mobilePhone": null,
    "officeLocation": null,
    "preferredLanguage": null,
    "surname": null,
    "userPrincipalName": "Alex@25ky3d.onmicrosoft.com"
}
    \end{minted}
    \caption[Query 2: Response body Microsoft Graph]{Response body van de tweede query via Microsoft Graph Explorer.}
    \label{Q2MSGRB}
\end{listing}

\subsubsection{Response header}

\begin{listing}[!h]
    \begin{minted}
        [
        frame=lines,
        framesep=2mm,
        baselinestretch=1.2,
        fontsize=\scriptsize,
        linenos
        ]  
        {json}
{
    "cache-control": "no-cache",
    "client-request-id": "dce37693-434c-e47b-0cf5-b08bc11752fb",
    "content-type": "application/json;odata.metadata=minimal;
        odata.streaming=true;IEEE754Compatible=false;charset=utf-8",
    "location": "https://graph.microsoft.com/v2/ffa43659-...-838af35d1353/directoryObjects/
        bc521ac5-...-c1c52ffe3ee7/Microsoft.DirectoryServices.User",
    "request-id": "fab6e8dc-c90c-4377-a54f-b67034f3a07e"
}
    \end{minted}
    \caption[Query 2: Response header Microsoft Graph]{Response header van de tweede query via Microsoft Graph Explorer.}
    \label{Q2MSGRH}
\end{listing}

\clearpage

\section{Query 3: Gebruiker aanpassen}

\subsection{Azure AD Graph}

\subsubsection{Request}

\begin{itemize}
    \item \Ac{HTTP}-method: PATCH
    \item \ac{URL}: https://graph.windows.net/ffa...353/users/Alex@25ky3d.onmicrosoft.com
    \item \Ac{API}-version: 1.6
\end{itemize}

\subsubsection{Request body}

\begin{listing}[!h]
    \begin{minted}
        [
        frame=lines,
        framesep=2mm,
        baselinestretch=1.2,
        fontsize=\scriptsize,
        linenos
        ]  
        {json}
{
    "displayName": "Alex A. Wu"
}
    \end{minted}
    \caption[Query 3: Request body Azure AD Graph]{Request body van de derde query via Azure \Ac{AD} Graph Explorer.}
    \label{Q3AADRQB}
\end{listing}

\subsubsection{Response}

Geen response beschikbaar. Bij een PATCH-request wordt er bij een succesvolle verwerking geen response meegegeven. Als bewijs dat de query werd uitgevoerd wordt volgende screenshot meegegeven. 

% TODO: instert photo

\subsection{Microsoft Graph}

\subsubsection{Request}

\begin{itemize}
    \item \Ac{HTTP}-method: PATCH
    \item \ac{API}-version: 1.0
    \item \Ac{URL}: https://graph.microsoft.com/v1.0/users/\{407074b1-...-41b76fbd7b67\}
\end{itemize}

\subsubsection{Request body}

\begin{listing}[!h]
    \begin{minted}
        [
        frame=lines,
        framesep=2mm,
        baselinestretch=1.2,
        fontsize=\scriptsize,
        linenos
        ]  
        {json}
{
    "displayName": "Alex A. Wu"
}
    \end{minted}
    \caption[Query 3: Request body Microsoft Graph]{Request body van de derde query via Microsoft Graph Explorer.}
    \label{Q3MSGRQB}
\end{listing}

\subsubsection{Response}

Geen response beschikbaar. Bij een PATCH-request wordt er bij een succesvolle verwerking geen response meegegeven. Als bewijs dat de query werd uitgevoerd kan ... geraadpleegd worden. % TODO ... vervangen met verwijzing naar vorige onderdeel 

\clearpage

\section{Query 4: Gebruiker verwijderen}

\subsection{Azure AD Graph}

\subsubsection{Request}

\begin{itemize}
    \item \Ac{HTTP}-method: DELETE
    \item \ac{URL}: https://graph.windows.net/ffa...353/users/Alex@25ky3d.onmicrosoft.com
    \item \Ac{API}-version: 1.6
\end{itemize}

\subsubsection{Response}

Geen response beschikbaar. Bij een DELETE-request wordt er bij een succesvolle verwerking geen response meegegeven. Als bewijs dat de query werd uitgevoerd wordt volgende screenshot meegegeven. 

% TODO: instert photo

\subsection{Microsoft Graph}

\subsubsection{Request}

\begin{itemize}
    \item \Ac{HTTP}-method: DELETE
    \item \ac{API}-version: 1.0
    \item \Ac{URL}: https://graph.microsoft.com/v1.0/users/\{407074b1-...-41b76fbd7b67\}
\end{itemize}

\subsubsection{Response}

Geen response beschikbaar. Bij een DELETE-request wordt er bij een succesvolle verwerking geen response meegegeven. Als bewijs dat de query werd uitgevoerd kan ... geraadpleegd worden. % TODO ... vervangen met verwijzing naar vorige onderdeel

% ---
% CHAPTER
% ---

\chapter{Audit-script PowerShell-scripts}%
\label{ch:PowerShell-scripts}

Hier komen de 20 onderdelen + connect MS Graph scripts

%%---------- Andere bijlagen --------------------------------------------------
% TODO: Voeg hier eventuele andere bijlagen toe. Bv. als je deze BP voor de
% tweede keer indient, een overzicht van de verbeteringen t.o.v. het origineel.
%\input{...}

%%---------- Backmatter, referentielijst ---------------------------------------

\backmatter{}

\emergencystretch=2em
\setlength\bibitemsep{2pt} %% Add Some space between the bibliograpy entries
\printbibliography[heading=bibintoc]

\end{document}
