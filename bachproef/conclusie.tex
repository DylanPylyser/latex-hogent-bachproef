%%=============================================================================
%% Conclusie
%%=============================================================================

\chapter{Conclusie}%
\label{ch:conclusie}

% TODO: Trek een duidelijke conclusie, in de vorm van een antwoord op de
% onderzoeksvra(a)g(en). Wat was jouw bijdrage aan het onderzoeksdomein en
% hoe biedt dit meerwaarde aan het vakgebied/doelgroep? 
% Reflecteer kritisch over het resultaat. In Engelse teksten wordt deze sectie
% ``Discussion'' genoemd. Had je deze uitkomst verwacht? Zijn er zaken die nog
% niet duidelijk zijn?
% Heeft het onderzoek geleid tot nieuwe vragen die uitnodigen tot verder 
%onderzoek?

In dit onderzoek wordt er een antwoord gegeven op de onderzoeksvraag: “Is Microsoft Graph klaar om de beheertaken die mogelijk waren met Azure AD Graph over te nemen?” Om een antwoord te geven op deze vraag, werden beide technologieën tegenover elkaar gezet op vier onderdelen. Bovendien is er een praktische uitwerking uitgevoerd om de praktische mogelijkheden van Microsoft Graph ten opzichte van Azure \Ac{AD} Graph te bekijken. \\

Uit de vergelijkende studie over de werking van de \Ac{API} blijkt het volgende. Om te beginnen behoudt Microsoft Graph dezelfde basis als Azure \Ac{AD} Graph voor de \Ac{API}. De mogelijke \Ac{HTTP}-verzoeken binnen de \Ac{API} zijn hetzelfde. Vervolgens verschillen de eindpunten niet significant, maar het eindpunt van Microsoft Graph wordt wel anders aangesproken. Bovendien bevatten de request en responses tussen beide lichte verschillen. Microsoft Graph verwerkt opvallend minder data voor dezelfde queries. Microsoft Graph ondersteunt niet altijd dezelfde properties die in Azure \Ac{AD} Graph voorkomen, maar in het merendeel van de gevallen wel. Als laatste heeft Azure \Ac{AD} Graph twee stabiele versies voor PowerShell, terwijl Microsoft Graph nog steeds in ontwikkeling is en maar een stabiele versie bevat voor PowerShell. \\

Voor de resterende drie onderdelen uit de vergelijkende studie kan het volgende besloten worden. Microsoft Graph heeft niet alle Azure \Ac{AD} Graph data-objecten overgenomen, wat leidt tot een voorlopige afsterving van bepaalde data-objecten binnen Azure \Ac{AD}. Microsoft Graph gaat wel verder dan alleen Azure \Ac{AD} en kan verschillende Microsoft-entiteiten aanspreken zoals Outlook, Teams en meer. Beide technologieën ondersteunen dezelfde toegangsmogelijkheden, namelijk een gedelegeerde en geautomatiseerde manier. Op vlak van Security bevat Microsoft Graph meer permissies dan Azure \Ac{AD} Graph. Enerzijds is dit een voordeel, doordat een gebruiker nauwkeuriger rechten kan krijgen. Anderzijds is dit een nadeel, meer permissies en entiteiten zorgen voor een complexer beheer van de rechten. Dit kan leiden tot fouten of misbruik van de rechten of permissies. \\ 

Uit de Proof-of-Concept blijkt dat het auditing-script niet volledig kan worden omgezet met behulp van Microsoft Graph PowerShell-modules. 13 van de 20 functies uit het PowerShell-script zijn omgezet van Azure \Ac{AD} PowerShell-modules naar Microsoft Graph PowerShell-modules. De zeven resterende functies gaan over niet-persoonlijke mailboxen. Deze functies kunnen worden opgelost aan de hand van gecodeerde \ac{HTTP}-verzoeken, maar niet via de Microsoft Graph PowerShell-modules. Functies in verband met domeinen, gebruikers, licenties, persoonlijke mailboxen en \Ac{MFA} worden wel ondersteund. Dit leidt tot de conclusie dat de eerste stabiele versie nog niet alles bevat dat wel met de Azure \Ac{AD} PowerShell-modules mogelijk waren. \\ 

Is Microsoft Graph klaar om de beheertaken die mogelijk waren met Azure \Ac{AD} Graph over te nemen? Het antwoord op deze vraag is zowel ja als neen. Enerzijds gaat Microsoft Graph breder en heeft het een ruimer assortiment van Microsoft-entiteiten dat het kan aanspreken en beheren. Anderzijds biedt Microsoft Graph vandaag de dag minder beheermogelijkheden aan op vlak van Azure \Ac{AD} in vergelijking met Azure \Ac{AD} Graph en bijhorende PowerShell-modules. \\

Bij de start van dit onderzoek waren de persoonlijke verwachtingen in de lijn van de conclusie. Bepaalde verwachtingen op vlak van de vergelijking, die in het onderzoeksvoorstel te vinden zijn, komen overeen met de eindresultaten. Naar de praktische kant toe, werd er verwacht dat de Microsoft Graph PowerShell-modules nog niet alles konden aanbieden in de eerste stabiele versie. Bovendien overstijgt Microsoft Graph de persoonlijke verwachtingen, door een stabiele versie aan te bieden voor productieomgevingen. Dit werd niet verwacht tijdens de start van dit onderzoek. \\

Dit onderzoek biedt eerst en vooral een meerwaarde voor het bedrijf Easi. Easi was vragende partij in dit onderzoek, doordat het auditing-script moest herwerkt worden met Microsoft Graph. Bovendien is deze studie een van de voorlopers op vlak van onderzoek naar Microsoft Graph. Voor Microsoft administrators is deze studie een goed vertrekpunt voor realisaties en onderzoeken naar de toekomst toe. \\

Microsoft Graph zit nog steeds in de ontwikkelingsfase. Naar de toekomst kan de technologie veel veranderen. Dit kan leiden tot een merkwaardige evolutie tussen dit onderzoek en toekomstige studies over het onderwerp. 
 

