%%=============================================================================
%% Inleiding
%%=============================================================================

\chapter{\IfLanguageName{dutch}{Inleiding}{Introduction}}%
\label{ch:inleiding}

\begin{comment}
De inleiding moet de lezer net genoeg informatie verschaffen om het onderwerp te begrijpen en in te zien waarom de onderzoeksvraag de moeite waard is om te onderzoeken. In de inleiding ga je literatuurverwijzingen beperken, zodat de tekst vlot leesbaar blijft. Je kan de inleiding verder onderverdelen in secties als dit de tekst verduidelijkt. Zaken die aan bod kunnen komen in de inleiding~\autocite{Pollefliet2011}:

\begin{itemize}
  \item context, achtergrond => OK?
  \item afbakenen van het onderwerp => OK?
  \item verantwoording van het onderwerp, methodologie !!!
  \item probleemstelling !!!
  \item onderzoeksdoelstelling: !!! TODO !!!
  \item onderzoeksvraag: Is Microsoft Graph op dit moment klaar om de beheertaken die mogelijk waren met Azure AD graph over te nemen?
  \item \ldots 
\end{itemize}
\end{comment}

% Taalcheck OK

Met de opkomst van digitale transformatie en de groeiende behoefte aan naadloze samenwerking en efficiënt gegevensbeheer, heeft Microsoft Graph zich gevestigd als een essentieel hulpmiddel van Microsoft binnen het moderne zakelijke landschap. Als een krachtige \Ac{API} biedt Microsoft Graph systeembeheerders de mogelijkheid om een geheel nieuwe dimensie van verbondenheid en intelligentie te creëren in systeembeheertaken. Met zijn diepgaande integratie met verschillende Microsoft-producten en -diensten ontsluit Microsoft Graph de waarde van gegevens, netwerken en activiteiten. Deze brede set aan mogelijkheden is ontstaan om zakelijke processen te verbeteren, het beheer van systemen en apparaten te stroomlijnen en gebruikers van de technologie een betere ervaring te bieden. \\ 

De evolutie van Microsoft administration heeft een aanzienlijke impact gehad op de manier waarop organisaties hun \ac{IT}-infrastructuur beheren. Microsoft administration heeft zich ontwikkeld van traditionele, handmatige beheerprocessen naar geavanceerde tools en technologieën die het beheer van \ac{AD} vereenvoudigen en optimaliseren. \ac{AD} staat dan ook in voor het verzamelen van allerlei informatie zoals gebruikers, computers, groepen, organisaties, beveiliging en certificaten. Met de opkomst van complexe netwerken, groeiende gebruikersbases en de behoefte aan strengere beveiligingsmaatregelen, hebben systeembeheerders steeds meer behoefte aan geautomatiseerde en schaalbare oplossingen om \ac{AD} effectief te beheren. \\

\section{Probleemstelling}
\label{sec:probleemstelling}

\begin{comment}
Uit je probleemstelling moet duidelijk zijn dat je onderzoek een meerwaarde heeft voor een concrete doelgroep. De doelgroep moet goed gedefinieerd en afgelijnd zijn. Doelgroepen als ``bedrijven,'' ``KMO's'', systeembeheerders, enz.~zijn nog te vaag. Als je een lijstje kan maken van de personen/organisaties die een meerwaarde zullen vinden in deze bachelorproef (dit is eigenlijk je steekproefkader), dan is dat een indicatie dat de doelgroep goed gedefinieerd is. Dit kan een enkel bedrijf zijn of zelfs één persoon (je co-promotor/opdrachtgever).
\end{comment}

% Taalcheck OK

De visie van Microsoft om een centraal eindpunt te ontwikkelen voor verschillende Microsoft-entiteiten brengt een verschuiving met zich mee. Allerlei producten en diensten van Microsoft, die vanuit een bijhorende \ac{API} worden gestuurd, bundelen zich in de nieuwe Microsoft Graph. Dit leidt tot de uitfasering van reeds bestaande Microsoft-technologieën waaronder \ac{ADAL} en Azure \Ac{AD} Graph \autocite{Sahay2022}. \\

Het \ac{IT}-bedrijf Easi, dat zich focust op \ac{IT}-oplossingen voor ondernemingen, biedt een Office 365-audit aan hun klanten. Met behulp van een PowerShell-script haalt de audit informatie uit een klantenomgeving om een advies naar een klant te onderbouwen. Dit script maakt gebruik van Azure \ac{AD} PowerShell-modules. Deze modules zitten mee in de aangekondigde uitfasering die op 30 juni 2023 zal plaatsvinden. Door dit scenario is Easi geïmpacteerd door deze verandering en wordt het gedwongen om gebruik te maken van de nieuwe Microsoft Graph. \\

Easi heeft sinds de aankondiging van Microsoft nog niet de mogelijkheid gehad om deze nieuwe kennis binnen te halen, waardoor dit PowerShell-script onveranderd is gebleven. Toch heeft Easi nood aan zekerheid en wilt het weten of dit script voor de uitfaseringsdatum kan omgezet worden met Microsoft Graph. Bovendien wilt het bedrijf weten wat de verschillen en gelijkenissen zijn tussen Azure \ac{AD} Graph en Microsoft Graph voor verder gebruik. \\

Door de casus biedt dit de mogelijkheid om onderzoek te doen naar Microsoft Graph. Om via dit onderzoek het probleem van Easi op te lossen, wordt er zowel een vergelijkende studie als Proof-of-Concept uitgevoerd. De vergelijkende studie vergelijkt de uitfaserende Azure \Ac{AD} Graph met de vervangende Microsoft Graph. De Proof-of-Concept bevat een praktische omvorming van het PowerShell-script van de Azure \Ac{AD} PowerShell naar Microsoft Graph PowerShell. \\

Naast Easi, waar de voornaamste focus op ligt als doelgroep, kan dit onderzoek ook een meerwaarde betekenen voor Microsoft-systeembeheerders die meer duiding willen over Microsoft Graph. Dit komt door een gebrek aan academische studies over Microsoft Graph, iets waar dit onderzoek verandering in brengt. 

\section{Onderzoeksvraag}
\label{sec:onderzoeksvraag}

\begin{comment}
Wees zo concreet mogelijk bij het formuleren van je onderzoeksvraag. Een onderzoeksvraag is trouwens iets waar nog niemand op dit moment een antwoord heeft (voor zover je kan nagaan). Het opzoeken van bestaande informatie (bv. ``welke tools bestaan er voor deze toepassing?'') is dus geen onderzoeksvraag. Je kan de onderzoeksvraag verder specifiëren in deelvragen. Bv.~als je onderzoek gaat over performantiemetingen, dan 
\end{comment}

\subsection{Hoofdonderzoeksvraag}

Uit de reeds besproken casus wordt volgende hoofdonderzoeksvraag gesteld: “Is Microsoft Graph op dit moment klaar om de beheertaken die mogelijk waren met Azure \ac{AD} Graph over te nemen?” De hoofdonderzoeksvraag zal dit onderzoek een onderbouwde conclusie geven dat te vinden is in Hoofdstuk \ref{ch:conclusie}.

\subsection{Deelonderzoeksvraag}

De hoofdonderzoeksvraag wordt ondersteund door volgende deelonderzoeksvragen. 

\begin{itemize}
   \item Wat zijn de mogelijke \Ac{HTTP}-verzoeken?
   \item Uit welke onderdelen bestaat het eindpunt?
   \item Zijn er verschillen aanwezig in requests en reponses tijdens het uitvoeren van een query?
   \item Welke PowerShell-versies zijn er?
   \item Welke data-objecten kunnen worden aangesproken?
   \item Bevat de technologie bepaalde afhankelijkheden?
   \item Hoe wordt er toegang verschaft tot de technologie?
   \item Zijn de toegangsmogelijkheden veilig?
   \item Wordt er gebruikgemaakt van rechten om misbruik tegen te gaan?
   \item Kan het PowerShell-script volledig omgezet worden van Azure \ac{AD} PowerShell naar Microsoft Graph PowerShell?
\end{itemize}

Door het gebruik van deze deelvragen kan de hoofdonderzoeksvraag beter benaderd worden en wordt het antwoord op de hoofdvraag grondiger onderbouwd. De deelvragen worden in de loop van de vergelijking en de praktische uitwerking beantwoord. Er volgt daarbij ook een samenvattend antwoord op de deelvragen in Hoofdstuk \ref{ch:conclusie}, waar er een conclusie wordt gevormd.

\section{Onderzoeksdoelstelling}%
\label{sec:onderzoeksdoelstelling}

\begin{comment}
Wat is het beoogde resultaat van je bachelorproef? Wat zijn de criteria voor succes? Beschrijf die zo concreet mogelijk. Gaat het bv.\ om een proof-of-concept, een prototype, een verslag met aanbevelingen, een vergelijkende studie, enz.
\end{comment}

% Taal OK

De hoofddoelstelling van dit onderzoek is om het audit-script van Easi zo volledig mogelijk om te vormen met het gebruik van Microsoft Graph PowerShell-modules. Deze doelstelling is volbracht als alle twintig, of zoveel mogelijk, onderdelen van het script zijn omgevormd. \\

Daarnaast heeft deze studie als doelstelling een concreet antwoord te geven op de hoofdonderzoeksvraag in de vorm van een aanbeveling. Deze aanbeveling is gunstig wanneer alle criteria uit de vergelijking en onderdelen van de Proof-of-Concept kunnen gebruikt worden om de aanbeveling te onderbouwen. \\

Het laatste doel is om Microsoft-systeembeheerders of geïnteresseerde bedrijven, die dit onderzoek lezen, bij te leren over hoe Microsoft Graph werkt. Met een nadruk op hoe de technologie moet en kan gebruikt worden met de huidige stand van zaken. 

\section{Opzet van deze bachelorproef}%
\label{sec:opzet-bachelorproef}

% Taal OK

De bachelorproef heeft volgende opbouw. \\

De stand van zaken betreft Microsoft administration, Azure \ac{AD} en Microsoft Graph worden weergegeven in Hoofdstuk \ref{ch:stand-van-zaken}. Dit hoofdstuk is een literaire verdieping van de drie onderwerpen. \\

Daarna volgt een toelichting van de methodologie in Hoofdstuk \ref{ch:methodologie}. In dit hoofdstuk worden de verdere fases van het onderzoek gekaderd en hoe deze worden uitgevoerd. \\

Vervolgens volgt de vergelijkende studie in Hoofdstuk \ref{ch:vergelijking} tussen Azure \Ac{AD} Graph en Microsoft Graph met hun bijhorende PowerShell-modules. In dit hoofdstuk worden beide technologieën tegenover elkaar gezet op basis van de vooropgestelde criteria. \\

Nadien volgt de Proof-of-Concept of praktische uitwerking in Hoofdstuk \ref{ch:poc}. Dit hoofdstuk is toegewijd aan het opzetten van de testomgeving en het omzetten van het audit-script met Microsoft Graph PowerShell-modules. \\

Achteraf volgt de conclusie in Hoofdstuk \ref{ch:conclusie}. De conclusie bevat een bespreking van de resultaten op basis van de drie vooropgestelde onderzoeksdoelstellingen met een antwoord op de hoofdonderzoeksvraag. Daarbij wordt ook een aanzet gegeven voor toekomstig onderzoek over dit onderwerp. \\

Ten slotte zijn de bijlagen in hoofdstuk \ref{ch:voorstel}, \ref{ch:HTTP-queries} en \ref{ch:PowerShell-scripts} te vinden die een ondersteunende functie hebben in dit onderzoek. Deze bijlagen bevatten het onderzoeksvoorstel, de uitgevoerde \Ac{HTTP}-queries en uitgewerkte Microsoft Graph PowerShell-scripts.

\begin{comment}
% Het is gebruikelijk aan het einde van de inleiding een overzicht te
% geven van de opbouw van de rest van de tekst. Deze sectie bevat al een aanzet
% die je kan aanvullen/aanpassen in functie van je eigen tekst.

De rest van deze bachelorproef is als volgt opgebouwd:

In Hoofdstuk~\ref{ch:stand-van-zaken} wordt een overzicht gegeven van de stand van zaken binnen het onderzoeksdomein, op basis van een literatuurstudie.

In Hoofdstuk~\ref{ch:methodologie} wordt de methodologie toegelicht en worden de gebruikte onderzoekstechnieken besproken om een antwoord te kunnen formuleren op de onderzoeksvragen.

% TODO: Vul hier aan voor je eigen hoofstukken, één of twee zinnen per hoofdstuk

In Hoofdstuk~\ref{ch:conclusie}, tenslotte, wordt de conclusie gegeven en een antwoord geformuleerd op de onderzoeksvragen. Daarbij wordt ook een aanzet gegeven voor toekomstig onderzoek binnen dit domein.
\end{comment}
