%%=============================================================================
%% Methodologie
%%=============================================================================

\chapter{\IfLanguageName{dutch}{Methodologie}{Methodology}}%
\label{ch:methodologie}

%% TODO: Hoe ben je te werk gegaan? Verdeel je onderzoek in grote fasen, en
%% licht in elke fase toe welke stappen je gevolgd hebt. Verantwoord waarom je
%% op deze manier te werk gegaan bent. Je moet kunnen aantonen dat je de best
%% mogelijke manier toegepast hebt om een antwoord te vinden op de
%% onderzoeksvraag.

Om een antwoord te vinden op deze onderzoeksvraag, worden de volgende stappen gehanteerd. \\

Als eerste wordt het onderzoeksdomein volledig gekaderd aan de hand van een literaire stand van zaken. Binnen deze literatuurstudie wordt er stap voor stap verder verdiept om volledig up-to-date te zijn met wat de termen en technologieën inhouden. Dit hoofdstuk wordt in drie grote delen opgesplitst om duidelijk de drie verschillende domeinen op te splitsen. Bij Azure Active Direcoty en Microsoft Graph wordt er breed gestart, om daarna de technologie op zich technisch te behandelen. \\

Na het verzamelen en verwerken van alle literaire studies, volgt de vergelijking tussen het oude en het nieuwe. Zoals reeds vermeld wordt de PowerShell-module en \ac{API} van Azure \ac{AD} uitgefaseerd. Azure \ac{AD} wordt dus gezien als het “oude”. De nieuwe Microsoft Graph, de vervanger van Azure \ac{AD}, is nog vandaag de dag volop in ontwikkeling. \\

De vergelijking tussen de twee bevat de eerste helft van het antwoord op de onderzoeksvraag. In deze vergelijking wordt er gekeken naar vier onderdelen. \\

\begin{itemize}
    \item Logica: Hoe worden de gegevens van start tot eind verwerkt?
    \item Data-objecten: Welke data-objecten kunnen worden aangesproken?
    \item Gebruik: Hoe wordt er geauthenticeerd en geautoriseerd? Waar en hoe maken we gebruik van deze technologie?
    \item Security: Hoe wordt er geauthenticeerd, en is deze authenticatiemethode veilig? Wordt er gebruikgemaakt van rechten om misbruik tegen te gaan?
\end{itemize}

Met deze vier onderdelen wordt de vergelijking tussen het oude en het nieuwe uitgevoerd. Deze vergelijking wordt aan de hand van de verwerkte literatuur toegepast.  \\

Om de tweede helft van het antwoord te geven op de onderzoeksvraag wordt de casus praktisch uitgewerkt in een Proof-of-Concept. Het verouderde Audit script, dat geschreven is in PowerShell, maakt gebruik van de verouderde modules. Dit script wordt vernieuwd door middel van Microsoft Graph. Door dit script te herwerken wordt er op een praktisch manier aangetoond wat wel en niet mogelijk is, vandaag de dag, met Microsoft Graph. \\

Om dit script zo efficiënt mogelijk te herwerken, worden twintig kleinere onderdelen van dit script behandeld. \\

% TODO Verder schrijven