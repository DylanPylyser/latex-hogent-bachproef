%%=============================================================================
%% Methodologie
%%=============================================================================

\chapter{\IfLanguageName{dutch}{Methodologie}{Methodology}}%
\label{ch:methodologie}

%% TODO: Hoe ben je te werk gegaan? Verdeel je onderzoek in grote fasen, en
%% licht in elke fase toe welke stappen je gevolgd hebt. Verantwoord waarom je
%% op deze manier te werk gegaan bent. Je moet kunnen aantonen dat je de best
%% mogelijke manier toegepast hebt om een antwoord te vinden op de
%% onderzoeksvraag.

Om een antwoord te vinden op deze onderzoeksvraag, zijn volgende stappen gehanteerd. \\

Als eerste kadert de literaire stand van zaken het onderzoeksdomein. Na het verzamelen en verwerken van alle literaire studies, volgt de vergelijking tussen het oude en het nieuwe. Zoals reeds vermeld komt er een uifasering van de Azure \ac{AD} PowerShell-modules en \ac{API}. Azure \ac{AD} wordt gezien als het “oude”. De nieuwe Microsoft Graph, de vervanger van Azure \ac{AD}, wordt gezien in de vergelijking als het “nieuwe”. \\

De vergelijking tussen de twee bevat de eerste helft van het antwoord op de onderzoeksvraag. In deze vergelijking wordt er gekeken naar vier criteria of onderdelen. 

\begin{itemize}
    \item Werking van de \ac{API}: Wat zijn de mogelijke \Ac{HTTP}-verzoeken? Uit welke onderdelen bestaat het eindpunt? Zijn er verschillen aanwezig in requests en reponses tijdens het uitvoeren van een query? Welke PowerShell-versies zijn er?
    \item Aanspreekbare data-objecten en dependencies: Welke data-objecten kunnen worden aangesproken? Bevat de technologie bepaalde afhankelijkheden?
    \item Toegangsmogelijkheden: Hoe wordt er toegang verschaft tot de technologie? 
    \item Security: Zijn de toegangsmogelijkheden veilig? Wordt er gebruikgemaakt van rechten om misbruik tegen te gaan?
\end{itemize}

Met deze vier onderdelen wordt de vergelijking tussen het oude en het nieuwe uitgevoerd. De vergelijking tussen het oude en het nieuwe baseert zich op de verwerkte literatuur. \\

Om de tweede helft van het antwoord te geven op de onderzoeksvraag wordt de casus praktisch uitgewerkt in een Proof-of-Concept. Het verouderde auditing-script van Easi, dat geschreven is in PowerShell, maakt gebruik van de verouderde modules. Dit script wordt vernieuwd door middel van de modernere Microsoft Graph PowerShell-modules. Door dit script op een praktisch manier te herwerken, kan dit aantonen wat vandaag de dag wel en niet mogelijk is met Microsoft Graph. \\

De praktische uitwerking maakt gebruik van een vijfstappenplan. Als eerste, het bekijken en beschrijven van de startende dataset na het activeren van de licentie. Als tweede, het aanpassen van de dataset naar een nagebootste klantenomgeving. Deze nabootsing wordt op basis van een intern document binnen Easi uitgevoerd. Daarna volgt het aanmaken van een applicatie voor Microsoft Graph. Vervolgens, het maken van een PowerShell-script om de testomgeving aan te spreken. Als laatste, het omvormen van het verouderde PowerShell-script dat Easi gebruikt. \\

Om dit script zo efficiënt mogelijk te herwerken, wordt het script in twintig functies verdeeld. Deze twintig onderdelen of functies maken gebruik van de uitfaserende Azure \ac{AD} PowerShell-module. Hieronder volgt een opsomming van de twintig functies met een korte beschrijving. 

\begin{enumerate}
    \item Aantal aanwezige domeinen binnen de Office 365-omgeving.
    \item Aantal gebruikers binnen de omgeving, onderverdeeld in interne
    en extene gebruikers.
    \item Aantal gebruikers die geblokkeerd zijn.
    \item Aantal geblokkeerde gebruikers met actieve licenties.
    \item Overzicht en telling van administrators met hun account status.
    \item Overzicht en telling van de interne gebruikers met \ac{MFA}.
    \item Aantal gelicenseerde, actieve accounts waarbij \ac{MFA} niet aanwezig is.
    \item Aantal externe accounts waarbij \ac{MFA} uitstaat.
    \item Overzicht van accounts met administrator-priviliges,
    onderverdeeld in gebruik van \ac{MFA}.
    \item Groottes van elk type mailbox.
    \item Groottes van elke mailbox, gerangschikt van groot naar klein.
    \item Onderverdeling van gebruikersmailboxen op basis van grootte.
    \item Onderverdeling van gedeelde mailboxen op basis van grootte.
    \item Onderverdeling van hoeveelheid mailboxen per domein.
    \item Overzicht van accounts met als primair \Ac{SMTP}-adresdomein “.onmicrosoft.com”.
    \item Overzicht gedeelde mailboxen, onderverdeeld in aanwezigheid van een licentie.
    \item Overzicht van verborgen mailboxen.
    \item Overzicht van \ac{SMTP}-forwarding.
    \item Overzicht verschillende types van licenties en het aantal personen met dit licentietype.
    \item Overzicht methodes waarop \ac{MFA} wordt gebruikt met hoeveelheid gebruikers, onderverdeeld op basis van methode.
\end{enumerate}

Binnen deze twintig onderdelen zijn er vijf groepen van data. Dit zijn de vijf soorten data die worden nagekeken wanneer klanten een Office 365-audit uitvoeren door Easi. De vijf groepen worden hieronder weergegeven. 

\begin{itemize}
    \item Domeinen
    \item Gebruikers
    \item Licenties
    \item Mailboxen
    \item \ac{MFA}
\end{itemize}

Tijdens de praktische uitwerking wordt er gebruikgemaakt van de Microsoft Graph PowerShell-modules. In het bijzonder, de meest recente versie van de eerste stabiele uitrol dat beschikbaar is tijdens de uitvoering van dit onderzoek, namelijk versie 1.25 \autocite{Microsoft2023k}. De mogelijkheid is reëel dat er tijdens of na de uitwerking een recentere versie zal worden uitgegeven door Microsoft. Er wordt niet afgeweken van versie 1.25 met bijhorende modules in dit onderzoek. \\  

Wanneer het onderdeel kan worden uitgevoerd via de PowerShell-modules van Microsoft Graph, dan wordt er een PowerShell-script uitgewerkt voor dat onderdeel. Dit bewijst dat Microsoft Graph het onderdeel uit het audit-script kan vervangen. \\

Als een onderdeel niet kan worden uitgevoerd via de Microsoft Graph PowerShell-modules, dan wordt er dieper gekeken naar het bijhorende domein. Er wordt uitgezocht dat het domein wel of niet ondersteund wordt door de PowerShell-modules van Microsoft Graph. Een mogelijk alternatief wordt besproken, maar de uitwerking hiervan valt buiten de scope van dit onderzoek. \\

Uiteindelijk wordt een conclusie gevormd op basis van beide helften uit dit onderzoek. 

% === END ====