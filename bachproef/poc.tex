%%=============================================================================
%% Proof-of-Concept
%%=============================================================================

\chapter{Uitwerking audit-script met Microsoft Graph PowerShell-module}%
\label{ch:poc}

% TODO: Bron zoeken over Sandboxes en over De Azure AD tenant?

Het tweede deel van de conclusie bestaat uit een praktische uitwerking met Microsoft Graph. De praktische uitwerking wordt uitgevoerd binnen een testomgeving met fictieve data. Deze data simuleert een mogelijke omgeving van een klant waar Easi een audit voor uitvoert. De praktische uitwerking behandeld twintig onderdelen van het audit-script die met PowerShell worden uitgevoerd. % TODO: verder schrijven

\section{Bespreking testomgeving}

Voor de Proof-of-Concept wordt er gebruikgemaakt van een testomgeving. Deze testomgeving is een gratis Azure Active Directory tenant dat via het Microsoft 365 Developer Program wordt aangeboden. \\

Een Azure \ac{AD} developer tenant is een gratis Sandbox-omgeving dat voor 90 dagen beschikbaar wordt gesteld, met een mogelijkheid tot verlenging. Een Sandbox is in \ac{IT}-termen een omgeving waarin je kan spelen met een of meerdere technologieën zonder dat hier consequenties aan vast hangen. \\

Binnen deze Sandbox kan Microsoft Graph gebruikt worden wanneer dit wordt opgesteld. Zoals vermeld in de literatuurstudie kan Microsoft Graph op twee manieren worden toegepast. Voor de Proof-of-Concept wordt er gebruikgemaakt van de App-only access-methode. Enkele argumenten waarom deze authenticatiemethode wordt toegepast is hieronder te vinden. 

\begin{itemize}
    \item Er moet niet via een account worden ingelogd, dit bespaard moeite en tijd.
    \item Alleen de applicatie heeft nood aan de juiste rechten, dit is makkelijker te beheren.
    \item De applicatie wordt in dit gekoppeld aan een geheime code dat samenwerkt met \ac{MSAL}, dit zorgt voor extra veiligheid.
\end{itemize}

Een gedetailleerd stappenplan over de opzet van Microsoft Graph binnen de testomgeving wordt weergegeven in ... %TODO: uitschrijven! 

% TODO: Stappen oplijsten + figuren/images toevoegen van elke stap als BIJLAGE? => Zie andere BP's

\subsection{Bespreking Data-set}

De standaard data-set die werd verkregen na het activeren van de developer-licentie bestaat uit volgende informatie.

\begin{itemize}
    \item Naam: MSFT
    \item Tenant-id: ffa43659-6d7d-4f83-a517-838af35d1353
    \item Primair domein: 25ky3d.onmicrosoft.com
    \item Licentie Azure AD Premium P2
    \item Land: Ierland
    \item Gegevenslocatie: EU Model Clause compliant datacenters
    \item Gebruikers: zeventien
    \item Groepen: zes
    \item Toepassingen: nul
    \item Apparaten: nul
\end{itemize}

In de Proof-of-Concept is het cruciaal om de omgeving realistisch voor te stellen. In een productieomgeving van een klant of bedrijf is de data verschillend in vergelijking met de data die voorlopig wordt gebruikt in de test-omgeving. De test-tenant bevat volgende data per groep zonder aanpassingen.

\begin{itemize}
    \item Domeinen: Een domein genaamd 25ky3d.onmicrosoft.com.
    \item Gebruikers: Zeventien actieve gebruikers of “Members”.
    \begin{itemize}
        \item Administrator: Een gebruiker genaamd Dylan Pylyser.
        \item Standaard gebruikers: Zestien willekeurige gebruikers die door Microsoft worden voorzien.
    \end{itemize}
    \item Licenties: Alle zeventien gebruikers hebben alle licenties.
    \item Mailboxes: Alle zeventien gebruikers hebben een mailbox die leeg is.
    \item \Ac{MFA}: 
        \begin{itemize}
            \item \ac{MFA} is in gebruik: De administrator maakt gebruik van \ac{2FA} (wachtwoord en Microsoft Authenticator).
            \item \Ac{MFA} is niet in gebruik: De standaard gebruikers maken alleen gebruik van een User principal name en wachtwoord om in te loggen.
        \end{itemize}
\end{itemize}

Volgende aanpassingen worden uitgevoerd om de data van een productieomgeving te simuleren.

\begin{itemize}
    \item Gebruikers: 
    \begin{itemize}
        \item Een extra gebruiker genaamd Jarne Creten met Administrator-rechten.
        \item Een gebruiker genaamd Megan Bowen wordt gastgebruiker of “Guest”.
        \item Een gebruiker genaamd Lee Gu wordt geblokkeerd, “Accound status” staat gelijk aan disabled.
    \end{itemize}
    \item Licenties: De extra gebruiker Jarne Creten krijgt geen licenties toegewezen.
    \item Mailboxes: De gebruikers Adele, Diego, Johanna, Lynne, Megan, Miriam, Patti en Pradeep ontvangen drie mails waardoor de mailboxes meer gevuld zijn dan voordien.
    \item \ac{MFA}: De gebruikers Adele en Jarne maken gebruik van de Microsoft Authenticator applicatie om in te loggen waardoor \ac{MFA} actief is.
\end{itemize}

\section{Uitwerking}

