%%=============================================================================
%% Proof-of-Concept
%%=============================================================================

\chapter{Uitwerking audit-script met Microsoft Graph PowerShell-module}%
\label{ch:poc}

% TODO: Bron zoeken over Sandboxes en over De Azure AD tenant?

Het tweede deel van de conclusie bestaat uit een praktische uitwerking met Microsoft Graph. De praktische uitwerking wordt uitgevoerd binnen een testomgeving met fictieve data. Deze data simuleert een mogelijke omgeving van een klant waar Easi een audit voor uitvoert. De praktische uitwerking behandeld twintig onderdelen van het audit-script die met PowerShell worden uitgevoerd.

\section{Bespreking testomgeving}

Voor de Proof-of-Concept wordt er gebruikgemaakt van een testomgeving. Deze testomgeving is een gratis Azure Active Directory tenant dat via het Microsoft 365 Developer Program wordt aangeboden door \textcite{Microsoft2023q}. \\

Een Azure \ac{AD} developer tenant is een gratis Sandbox-omgeving die voor 90 dagen beschikbaar wordt gesteld, met een mogelijkheid tot verlenging. Een Sandbox is in \ac{IT}-termen een omgeving waarin je kan spelen met één of meerdere technologieën, zonder dat hier consequenties aan vast hangen. \\

Binnen deze Sandbox kan Microsoft Graph worden gebruikt wanneer dit  geconfigureerd is. Zoals vermeld in de literatuurstudie kan Microsoft Graph op twee manieren worden toegepast. Voor de Proof-of-Concept wordt er gebruikgemaakt van de App-only access-methode. Enkele argumenten waarom deze authenticatiemethode wordt toegepast zijn hieronder te vinden. 

\begin{itemize}
    \item Er moet niet via een account worden ingelogd, dit bespaard moeite en tijd.
    \item Enkel de applicatie heeft nood aan de juiste rechten, dit is makkelijker te beheren.
    \item De applicatie wordt aan Microsoft Graph gekoppeld via een geheime code dat samenwerkt met \ac{MSAL}, dit zorgt voor extra veiligheid.
\end{itemize}

\subsection{Bespreking Data-set}

De standaard data-set die werd verkregen na het activeren van de developer-licentie bestaat uit volgende informatie.

\begin{itemize}
    \item Naam: MSFT
    \item Tenant-id: ffa43659-6d7d-4f83-a517-838af35d1353
    \item Primair domein: 25ky3d.onmicrosoft.com
    \item Licentie Azure AD Premium P2
    \item Land: Ierland
    \item Gegevenslocatie: EU Model Clause compliant datacenters
    \item Gebruikers: zeventien
    \item Groepen: zes
    \item Toepassingen: nul
    \item Apparaten: nul
\end{itemize}

In de Proof-of-Concept is het cruciaal om de omgeving realistisch voor te stellen. In een productieomgeving van een klant of bedrijf is de data verschillend in vergelijking met de data die voorlopig wordt gebruikt in de test-omgeving. De test-tenant bevat volgende data per groep zonder aanpassingen.

\begin{itemize}
    \item Domeinen: Een domein genaamd 25ky3d.onmicrosoft.com.
    \item Gebruikers: Zeventien actieve gebruikers of “Members”.
    \begin{itemize}
        \item Administrator: Een gebruiker genaamd Dylan Pylyser.
        \item Standaard gebruikers: Zestien willekeurige gebruikers die door Microsoft worden voorzien.
    \end{itemize}
    \item Licenties: Alle zeventien gebruikers hebben alle licenties.
    \item Mailboxes: Alle zeventien gebruikers hebben een mailbox die leeg is.
    \item \Ac{MFA}: 
        \begin{itemize}
            \item \ac{MFA} is in gebruik: De administrator maakt gebruik van \ac{2FA} (wachtwoord en Microsoft Authenticator).
            \item \Ac{MFA} is niet in gebruik: De standaard gebruikers maken alleen gebruik van een User principal name en wachtwoord om in te loggen.
        \end{itemize}
\end{itemize}

Volgende aanpassingen worden uitgevoerd om de data van een productieomgeving te simuleren.

\begin{itemize}
    \item Gebruikers: 
    \begin{itemize}
        \item Een extra gebruiker genaamd Jarne Creten met Administrator-rechten.
        \item Een gebruiker genaamd Megan Bowen wordt gastgebruiker of “Guest”.
        \item Een gebruiker genaamd Lee Gu wordt geblokkeerd, “Accound status” staat gelijk aan disabled.
    \end{itemize}
    \item Licenties: De extra gebruiker Jarne Creten krijgt geen licenties toegewezen.
    \item Mailboxes: De gebruikers Adele, Diego, Johanna, Lynne, Megan, Miriam, Patti en Pradeep ontvangen drie mails waardoor de mailboxes meer gevuld zijn dan voordien.
    \item \ac{MFA}: De gebruikers Adele en Jarne maken gebruik van de Microsoft Authenticator applicatie om in te loggen waardoor \ac{MFA} actief is.
\end{itemize}

\section{Proof-of-Concept}

\subsection{Microsoft Graph activeren via een applicatie}

Voor het activeren van Microsoft Graph is er een stappenplan voorzien om dit uit te voeren. \\ % TODO: aanvullen met alle stappen in bijlagen! \\

Het opzetten van de effectieve connectie gebeurd via Listing \ref{PSQ0}. Dit PowerShell-script is uitgewerkt door \textcite{Terlisten2022}. Het script maakt gebruik van vier variabelen, namelijk “AppId”, “TenantId”, “ClientSecret” en “Token”. De eerste drie variabelen zijn nodig om de toepassing te kunnen aanspreken. De vierde variabele haalt een \ac{MSAL}-token op. Deze token wordt gebruikt om op een veilige manier in te loggen op de tenant met Microsoft Graph. In PowerShell wordt dit vertaald met het commando “Connect-Graph” met de optie “-AccessToken”. 

\subsection{Uitwerking onderdelen van het audit-script}

% TODO: Spreek alle onderdelen (functies aan en werk in 3 delen): Verwacht resultaat, script, output

De twintig functies waar het auditing-script van gebruikmaakt wordt in dit onderdeel besproken. Er wordt gebruikgemaakt van de testomgeving-data om de scripts te testen. \\

Alle functies in verband met \ac{MFA} maken gebruik van een reeds bestaand script van \textcite{Allen2023}. Dit script wordt omgevormd tijdens de uitwerking, maar dient als basis om deze functies uit te werken.

\subsubsection{Functie 1: Aantal aanwezige domeinen binnen de Office 365-omgeving}

De eerste functie is het opvragen van de domeinen binnen de Office 365-omgeving. Met het gebruik van de testomgeving-data wordt er verwacht dat er een domein genaamd “25ky3d.onmicrosoft.com” te zien is met een status in de aard van “Available” of “Working”. \\

Na het uitvoeren van Listing \ref{PSQ1} komt alleen het domein overeen met de verwachtingen. Het opvragen van de status is niet mogelijk en wordt aangeduid met “Not available” in het script, omdat er geen gegevens aanwezig zijn. Deze functie kan deels worden omgezet met Microsoft Graph.

\subsubsection{Functie 2: Aantal gebruikers binnen de omgeving, onderverdeeld in interne en extene gebruikers}

De tweede functie van het audit-script is het weergeven van alle gebruikers met hun account-type. Naast het weergeven wordt er ook een onderverdeling gemaakt tussen interne “Members” en externe “Guests” gebruikers. Er wordt verwacht dat er een externe gebruiker aanwezig is van de 18 gebruikers. Bovendien zijn alle gebruikers alleen beschikbaar in de cloud, en hebben ze geen connectie \ac{On-prem}. \\

Het script in Listing \ref{PSQ2} maakt gebruik van “get-MgUser”. Uit het commando “get-MgUser” worden de “UserPrincipalName”, “OnPremisesSyncenabled” en “UserType” gehaald.  Na het uitvoeren van het script komt de output overeen met de verwachtingen. Deze functie kan worden omgezet met Microsoft Graph. \\

\subsubsection{Functie 3: Aantal gebruikers die geblokkeerd zijn}

De derde functie van het audit-script is het aantal gebruikers die geblokkeerd zijn weergeven. De verwachting is dat er een account wordt weergegeven dat geblokkeerd is. \\

Dit script, te vinden in Listing \ref{PSQ3}, maakt gebruik van “get-MgUser”. Door het gebruik van “AccountEnabled” kan er worden achterhaald of het account geblokkeerd is. Na het uitvoeren van het script komt de output overeen met de verwachtingen. De derde functie kan worden omgezet met Microsoft Graph. 

\subsubsection{Functie 4: Aantal geblokkeerde gebruikers met actieve licenties}

De vierde functie gaat over het aantal geblokkeerde gebruikers met een actieve licentie. Er wordt verwacht dat de geblokkeerde gebruiker actieve licenties heeft. \\

Listing \ref{PSQ4} bevat het script voor dit onderdeel. Het script combineert het opvragen van gebruikers met het opvragen van de licenties. De output van het script bevestigt dat er 1 geblokkeerde gebruiker actieve licenties heeft. Het is met Microsoft Graph mogelijk om deze functie om te zetten.

\subsubsection{Functie 5: Overzicht en telling van administrators met hun account status}

Functie vijf bevat een overzicht en telling van de administrators binnen de omgeving met account status. De verwachtingen zijn dat de 2 administrators worden weergegeven met een actieve account status. \\

Na het uitvoeren van Listing \ref{PSQ5} zijn de twee beheerders te zien met een actieve account status. Het PowerShell-script filtert de administrators uit de lijst van gebruikers door middel van hun “UserId”. Om de juiste filtering toe te passen worden eerst de gebruikers met een administrator-rol opgevraagd, vooraleer de filtering op alle gebruikers plaatsvindt. Deze functie is succesvol omgezet met Microsoft Graph.

\subsubsection{Functie 6: Overzicht en telling van de interne gebruikers met MFA}

De zesde functie focust zich op het geven van een overzicht en telling van de interne gebruikers met \ac{MFA}. Met de data uit de testomgeving wordt er verwacht dat van de 17 interne gebruikers maar 3 gebruikers \ac{2FA} of \ac{MFA} toepassen. \\

Het script in Listing \ref{PSQ6} overloopt alle gebruikers met bijhorende \ac{MFA}-instellingen. Het verzamelt alle \ac{MFA}-gegevens in een object, dit object wordt daarna gebruikt om een telling uit te voeren omtrent het gebruik van\ac{MFA}. De output van het script komt overeen met de verwachtingen. Deze functie kan dus worden omgezet met Microsoft Graph. 

\subsubsection{Functie 7: Aantal gelicenseerde, actieve accounts waarbij MFA niet aanwezig is}

De zevende functie geeft alle actieve gebruikers weer die minstens een licentie hebben maar geen \ac{MFA} toepassen. Er wordt verwacht dat er 14 van de 17 actieve gelicenseerde gebruikers geen \ac{MFA} toepast. \\

Listing \ref{PSQ7} is gelijkaardig aan de vorige functie op vlak van PowerShell-code, maar filtert op de niet-geblokkeerde accounts met actieve licenties. Na het uitvoeren van dit script komen de resultaten overeen met de verwachtingen. De zevende functie kan worden omgezet met Microsoft Graph.

\subsection{Functie 8: Aantal externe accounts waarbij MFA uitstaat}

De achtste functie focust zich op een andere doelgroep, namelijk de “Guest”-accounts binnen de omgeving. Er is maar een externe gebruiker aanwezig in de testomgeving. Er verwacht dat het PowerShell-script aangeeft dat de enige externe gebruiker geen \ac{MFA} gebruikt. \\

Het script in Listing \ref{PSQ8} filtert op de externe gebruikers en haalt de \ac{MFA}-gegevens op. De output van het script geeft dezelfde resultaten weer zoals verwacht werd. De achteste functie kan met Microsoft Graph omgezet worden.

\subsection{Functie 9: Overzicht van accounts met administrator-priviliges, onderverdeeld in gebruik van MFA}

De negende functie heeft betrekking op de \ac{MFA}-instellingen van beheerders binnen de testomgeving. Er zijn 2 administrators aanwezig, waarbij beide gebruikers \ac{MFA} toepassen. Er wordt verwacht dat het PowerShell-script hetzelfde weergeeft. \\

Het negende PowerShell-script, dat in Listing \ref{PSQ9} te vinden is, filtert op de administrators. % TODO: verder schrijven!!!

