% !TeX spellcheck = nl_NL-Dutch
%%=============================================================================
%% Samenvatting
%%=============================================================================

% TODO: De "abstract" of samenvatting is een kernachtige (~ 1 blz. voor een thesis) synthese van het document.
%
% Een goede abstract biedt een kernachtig antwoord op volgende vragen:
%
% 1. Waarover gaat de bachelorproef?
% 2. Waarom heb je er over geschreven?
% 3. Hoe heb je het onderzoek uitgevoerd?
% 4. Wat waren de resultaten? Wat blijkt uit je onderzoek?
% 5. Wat betekenen je resultaten? Wat is de relevantie voor het werkveld?
%
% Daarom bestaat een abstract uit volgende componenten:
%
% - inleiding + kaderen thema
% - probleemstelling
% - (centrale) onderzoeksvraag
% - onderzoeksdoelstelling
% - methodologie
% - resultaten (beperk tot de belangrijkste, relevant voor de onderzoeksvraag)
% - conclusies, aanbevelingen, beperkingen
% === SAMENVATTING ===
%\IfLanguageName{english}{%
\selectlanguage{dutch}
\chapter*{Samenvatting}

% Taalcheck OK

Deze scriptie onderzoekt de evolutie van Azure Active Directory Graph naar Microsoft Graph binnen het domein van Microsoft administration. Het bedrijf Easi maakt gebruik van de uitfaserende Azure Active Directory Graph en wil weten of Microsoft Graph klaar is om deze uitfaserende technologie te vervangen. Het doel van het onderzoek is om te bepalen of Microsoft Graph op dit moment in staat is om de beheertaken die eerder mogelijk waren met Azure Active Directory Graph over te nemen. Naast het beantwoorden van deze vraag, zijn de onderzoeksdoelstellingen het herwerken van een PowerShell-script met Microsoft Graph en het vergroten van de kennis over Microsoft Graph voor Microsoft-systeembeheerders. De methodologie van het onderzoek bestaat uit twee delen. In eerste instantie wordt een vergelijkende studie uitgevoerd op basis van vier criteria om Azure Active Directory Graph en Microsoft Graph te analyseren. Vervolgens wordt een Proof-of-Concept gerealiseerd door middel van het herschrijven van een PowerShell-script met behulp van Microsoft Graph PowerShell-modules. De resultaten van het onderzoek tonen aan dat elf van de twintig functies volledig kunnen worden herwerkt, terwijl twee functies deels zijn herwerkt. De resterende zeven functies zijn omtrent niet-persoonlijke mailboxen en kunnen niet worden omgezet, wat leidt tot limitaties van de technologie vandaag de dag. Daarnaast concludeert de vergelijkende studie dat beide technologieën gelijkenissen vertonen, maar toch verschillen op eindpunt, aanspreekbare data-objecten en dependencies. De resultaten van beide delen tonen aan dat Microsoft Graph in staat is om de beheertaken van Azure Active Directory Graph over te nemen. Op basis van de resultaten wordt volgende conclusie getrokken. Het blijkt dat Microsoft Graph verdergaat dan enkel het vervangen van Azure Active Directory Graph. Azure Active Directory Graph is gefocust op het beheren van Azure Active Directory, terwijl Microsoft Graph vandaag de dag ook verschillende Microsoft-entiteiten zoals Office 365 en bijhorende applicaties kan beheren. 

%  en kan verschillende Microsoft-entiteiten zoals Office 365 en bijhorende applicaties
%\selectlanguage{english}

%%---------- Samenvatting -----------------------------------------------------
% De samenvatting in de hoofdtaal van het document

% Azure \ac{AD} is gefocust op het beheren van Azure \ac{AD}, terwijl Microsoft Graph vandaag de dag ook Microsoft Teams, Outlook, To Do en andere entiteiten kan aanspreken naast Azure \ac{AD}. Azure \ac{AD} kan minder Microsoft-entiteiten aanspreken in vergelijking met Microsoft Graph. 
