%%=============================================================================
%% Vergelijkende studie
%%=============================================================================

\chapter{Vergelijking oud en nieuw}%
\label{ch:vergelijking}

% TODO: Trek een duidelijke conclusie, in de vorm van een antwoord op de
% onderzoeksvra(a)g(en). Wat was jouw bijdrage aan het onderzoeksdomein en
% hoe biedt dit meerwaarde aan het vakgebied/doelgroep? 
% Reflecteer kritisch over het resultaat. In Engelse teksten wordt deze sectie
% ``Discussion'' genoemd. Had je deze uitkomst verwacht? Zijn er zaken die nog
% niet duidelijk zijn?
% Heeft het onderzoek geleid tot nieuwe vragen die uitnodigen tot verder 
%onderzoek?

\section{Achterliggende logica}

\subsection{Doorloop van data}

% TODO: 4. Idee nakijken
Idee: Vergelijking in hoe wordt de data verzonden, verwerkt en teruggestuurd?

Vb. Begint met ... en eindigt met JSON. Is dit hetzelfde of...

Note: ook nog mogelijk om meerdere subsecties of subsubsecties te maken om de fases goed in detail te bespreken en te vergelijken.

\section{Data-objecten}

\subsection{Werking van de API}

Idee: uitleggen hoe de API werkt en hoe het aan een data-object kan aanspreken

Vb. Waar zit het verschil? Is dit hetzelfde gebleven?

\subsection{Aantal aanspreekbare data-objecten}

Idee: een tabel of mooi overzicht over hoeveel data-objecten er kunnen worden aangesproken door beide.

Vb. Zijn er veel nieuwe bijgekomen? Valt er iets op? Is er iets weggegaan...

% Hier wordt het data-object 

% Dieper gaan op API = verbinding

\section{Gebruik}

\subsection{Authenticatiemethodes}

Idee: Via welke methodes kan men zich inloggen? 

Vb. Is er een verschil? Wordt er MFA ondersteund? ... (Gaat nog niet in detail over veiligheid -> zie Security)

\subsection{Omgeving}

Idee: Waar of hoe kan men de technologie aanspreken of gebruiken?

Vb. Moet die manueel gebeuren? Is dit ook mogelijk via scripts? Kan dit via PowerShell of moet je in Azure zitten...

% manueel werk, script werk... => verschil van werking, eventueel PowerShell aantoetsen

\section{Security}

\subsection{Veiligheid van authenticatiemethodes}

Idee: In dit onderdeel wordt er dieper in gegaan op vlak van de mogelijke authenticatiemethodes en hoe sterk ze zijn.

Vb. MFA is sterker dan alleen een wachtwoord...

\subsection{Gebruik van rechten}

Idee: Het beginsel van de minste voorrechten (PoLP) wordt toegepast. Is dit bij beide zo en hoe sterk is dit principe?

Vb. Leg PoLP dieper uit, wat kan het tegenhouden, waarom wordt het toegepast...

\subsection{Encryptie}

Idee: Wordt er een encryptie ondersteunt?
