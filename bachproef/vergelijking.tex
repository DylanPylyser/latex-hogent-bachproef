%%=============================================================================
%% Vergelijkende studie
%%=============================================================================

\chapter{Vergelijking oud en nieuw}%
\label{ch:vergelijking}

De eerste helft van de conclusie wordt gestaafd aan de hand van een vergelijkende studie. Azure \ac{AD} Graph en PowerShell wordt vergeleken met Microsoft Graph met bijhorende PowerShell-module. In de vergelijking worden beide technologieën aangesproken als het “oude” en het “nieuwe”. 

\section{Werking van de API}

% TO DO: 4. Ideeën nakijken => Normaal OK
% Idee: Vergelijking in hoe wordt de data verzonden, verwerkt en teruggestuurd?
% Vb. Begint met ... en eindigt met JSON. Is dit hetzelfde of...
% Note: ook nog mogelijk om meerdere subsecties of subsubsecties te maken om de fases goed in detail te bespreken en te vergelijken.

  

\subsection{Soorten HTTP-verzoek}

Zowel Azure \ac{AD} Graph als Microsoft Graph ondersteunen vijf soorten \ac{HTTP}-verzoeken. Deze vijf zijn GET, POST, PATCH, PUT en DELETE. 

\subsection{Eindpunt}

Op Figuur \ref{bfe} bij Azure \ac{AD} Graph, wordt er gebruikgemaakt van “windows” in de \ac{URL}. In vergelijking met Microsoft Graph, dat op Figuur \ref{RAM} te vinden is, komt het woord “microsoft” voor in de \ac{URL}. \\

Op vlak van onderdelen, wordt het volgende opgemerkt bij het vergelijken. Azure \ac{AD} Graph werkt met vier onderdelen: Tenant ID, Resource, (\ac{API}) Version en optionele OData query-parameters. Terwijl Microsoft Graph maar drie onderdelen gebruikt. de Tenant ID wordt uit het eindpunt geschrapt. Daarnaast worden de onderdelen “version” en “resource” van plaats gewisseld. De optionele query-parameters zijn in beide aanwezig. 


% TODO: Invullen wanneer er tijd over is
% !!! Idee: Alle HTTP-verzoeken uitvoeren, zien dat dit hetzelfde is en de resultaten opschrijven (zie hieronder voor voorbeeld)

% https://graphexplorer.azurewebsites.net/#
% GET: https://graph.windows.net/ffa43659-6d7d-4f83-a517-838af35d1353/domains
% Resultaat... Helemaal meegeven

% Dit hetzelfde voor MS Graph: https://developer.microsoft.com/en-us/graph/graph-explorer
% GET: https://graph.microsoft.com/v1.0/domains
% Resultaat... Helemaal meegeven

% VERGLIJKING !!!

% \subsection{Request en Response}

%Voor dit onderdeel wordt er gebruikgemaakt van de Graph Explorer die zowel het oude %als het nieuwe aanbiedt.



% \subsection{Versies}

% ---
% Section
% ---

\section{Aanspreekbare data-objecten en dependencies}

% Idee: uitleggen hoe de API werkt en hoe het aan een data-object kan aanspreken
% Vb. Waar zit het verschil? Is dit hetzelfde gebleven?

%Idee: een tabel of mooi overzicht over hoeveel data-objecten er kunnen worden aangesproken door beide.
%Vb. Zijn er veel nieuwe bijgekomen? Valt er iets op? Is er iets weggegaan...
% Hier wordt het data-object 


Met de migratie van Azure \ac{AD} Graph naar Microsoft Graph, wordt er verondersteld dat alle aanspreekbare data-objecten worden overgenomen. Wanneer het overzicht van data-objecten bij Azure \ac{AD} PowerShell naast dat van Microsoft Graph wordt gezet, ontstaat volgend overzicht dat te vinden is in Tabel \ref{AADMSG}. \\

\begin{table}
    \tiny
    \centering
    \begin{tabular}{ |c|c||c|c| } 
        \hline
        \textbf{Azure AD data-object} & \textbf{Aantal commando's} & \textbf{Microsoft Graph data-object} & \textbf{Aantal commando's} \\
        \hline
        Administrative Units & 9 & Administrative Units & 9 \\ 
        Application Proxy Application Management & 8 & / & / \\
        Application Proxy Connector Management & 9 & / & / \\
        Applications & 20 & Applications & 20 \\ 
        AzureAD & 49 & AzureAD & 48 \\ 
        Certificate Authorities & 4 & Certificate Authorities & 1 \\ 
        Connect to directory & 2 & Connect to directory & 2 \\ 
        Contacts & 8 & Contacts & 7 \\ 
        Contracts & 1 & Contracts & 1 \\ 
        Deleted Objects & 1 & Deleted Objects & 1 \\ 
        Devices & 11 & Devices & 9 \\    
        Directory & 3 & Directory & 3 \\
        Directory Objects & 1 & Directory Objects & 1 \\ 
        Directory Roles & 13 & Directory Roles & 13 \\ 
        Domains & 8 & Domains & 8 \\ 
        Extension Properties & 1 & Extension Properties & 1 \\ 
        Groups & 26 & Groups & 26 \\ 
        MSOnline & 97 & MSOnline & 76 \\
        OAuth2 & 2 & OAuth2 & 2 \\ 
        Policies & 2 & Policies & 2 \\ 
        Service Principals & 22 & Service Principals & 22 \\ 
        Users & 30 & Users & 30 \\ 
        \hline
    \end{tabular}
    \caption[Tabel migratie Azure AD data-objecten naar Microsoft Graph]{Tabel met overzicht van alle ondersteunende data-objecten bij de migratie van Azure \ac{AD} PowerShell naar Microsoft Graph uit documentatie van \textcite{Microsoft2023l}}
    \label{AADMSG}
\end{table}

Uit de blijkt dat Microsoft Graph de meerderheid van de Azure \ac{AD} data-objecten overneemt. Toch worden er enkele verschillen opgemerkt uit deze migratie. Zowel de Application Proxy Application Management en Application Proxy Connector Management vallen weg. Daarnaast is er een commando weggehaald uit het AzureAD en Contacts data-object. Vervolgens zijn er twee commando's minder bij het Devices data-object. Finaal zijn er 21 commando's weggehaald uit het MSOnline data-object in Microsoft Graph. \\

Naast de data-objecten bevat Microsoft Graph meer dependencies, dit wordt weergegeven in Tabel \ref{AADT}. Dit wilt zeggen dat Microsoft Graph meer Microsoft-entiteiten kan aanspreken dan de Azure \ac{AD} PowerShell-module. Azure \ac{AD} is dan ook gefocust op het beheren van Azure \ac{AD}, terwijl Microsoft Graph vandaag de dag ook Microsoft Teams, Outlook, To Do en andere entiteiten kan aanspreken naast Azure \ac{AD}.

% ---
% Section
% ---

\section{Gebruik}

\subsection{Authenticatiemethodes}

Idee: Via welke methodes kan men zich inloggen? 

Vb. Is er een verschil? Wordt er MFA ondersteund? ... (Gaat nog niet in detail over veiligheid -> zie Security)

\subsection{Omgeving}

Idee: Waar of hoe kan men de technologie aanspreken of gebruiken?

Vb. Moet die manueel gebeuren? Is dit ook mogelijk via scripts? Kan dit via PowerShell of moet je in Azure zitten...

% manueel werk, script werk... => verschil van werking, eventueel PowerShell aantoetsen, Java? ...




% ---
% Section
% ---

\section{Security}

\subsection{Veiligheid van authenticatiemethodes}

Idee: In dit onderdeel wordt er dieper in gegaan op vlak van de mogelijke authenticatiemethodes en hoe sterk ze zijn.

Vb. MFA is sterker dan alleen een wachtwoord...

\subsection{Gebruik van rechten}

% Idee: Het beginsel van de minste voorrechten (PoLP) wordt toegepast. Is dit bij beide zo en hoe sterk is dit principe?

% Vb. Leg PoLP dieper uit, wat kan het tegenhouden, waarom wordt het toegepast...


