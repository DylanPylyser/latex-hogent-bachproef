%%=============================================================================
%% Vergelijkende studie
%%=============================================================================

\chapter{Vergelijking oud en nieuw}%
\label{ch:vergelijking}

De eerste helft van de conclusie wordt gestaafd aan de hand van een vergelijkende studie. Azure \ac{AD} Graph met bijhorende PowerShell-modules wordt vergeleken met Microsoft Graph met bijhorende PowerShell-modules. In de vergelijking worden beide technologieën aangesproken als het “oude” en het “nieuwe”. 

\section{Werking van de API}

% TO DO: 4. Ideeën nakijken => Normaal OK
% Idee: Vergelijking in hoe wordt de data verzonden, verwerkt en teruggestuurd?
% Vb. Begint met ... en eindigt met JSON. Is dit hetzelfde of...
% Note: ook nog mogelijk om meerdere subsecties of subsubsecties te maken om de fases goed in detail te bespreken en te vergelijken.

  

\subsection{HTTP-verzoeken}

Zowel Azure \ac{AD} Graph als Microsoft Graph ondersteunen vijf soorten \ac{HTTP}-verzoeken. Deze vijf zijn GET, POST, PATCH, PUT en DELETE. 

\subsection{Eindpunt}

Op Figuur \ref{bfe} bij Azure \ac{AD} Graph, wordt er gebruikgemaakt van “windows” in de \ac{URL}. In vergelijking met Microsoft Graph, dat op Figuur \ref{RAM} te vinden is, komt het woord “microsoft” voor in de \ac{URL}. \\

Op vlak van onderdelen, wordt het volgende opgemerkt bij het vergelijken. Azure \ac{AD} Graph werkt met vier onderdelen: Tenant ID, Resource, (\ac{API}) Version en optionele OData query-parameters. Terwijl het nieuwe maar drie onderdelen gebruikt. De Tenant ID wordt uit het eindpunt geschrapt. Daarnaast worden de onderdelen “version” en “resource” van plaats gewisseld. De optionele query-parameters zijn in beide aanwezig. 




\subsection{Request en Response} % TODO: in bijlage !!!

Voor dit onderdeel wordt er gebruikgemaakt van de Graph Explorer. Beide technologieën bieden deze functie aan om de \ac{API} te testen. Voor Azure \ac{AD} Graph kan dit via “https://graphexplorer.azurewebsites.net/”. Voor Microsoft Graph kan dit via “https://developer.microsoft.com/en-us/graph/graph-explorer”. \\

In dit onderdeel worden er vier HTTP-verzoeken uitgevoerd op basis van de verschillende \ac{HTTP}-verzoeken die in beide technologieën aanwezig zijn. De verzoeken die worden gebruikt zijn GET, POST, PATCH en DELETE. PUT wordt niet ondersteund in de Azure \ac{AD} Graph Explorer, deze wordt dus niet in de vergelijking opgenomen. \\

Het verzoek en het antwoord worden op beide Graph Explorers uitgevoerd en vergeleken. De data uit de vergelijking wordt voorzien door de testomgeving van de Proof-of-Concept. De uitleg over de testomgeving met bijhorende data is te vinden in het volgende hoofdstuk. 

\subsubsection{Bespreking van de uitgevoerde queries}

De resultaten bij het uitvoeren van de vier queries wordt hieronder besproken met bijhorende gegevens uit Bijlage \ref{ch:HTTP-queries}. \\

Bij het opstellen van een verzoek voor beide \Ac{API}'s, is te merken dat de \Ac{URL} en \Ac{API}-version worden omgedraaid. Daarnaast is de \Ac{URL} bij Azure \ac{AD} Graph langer, door het gebruik van de tenant-id. Beide zaken werden benoemd in de literatuurstudie. \\

Bij het eerste verzoek zijn er lichte verschillen aan te merken. De response body, zie Listing \ref{Q1AADRB} en \ref{Q1MSGRB}, wordt anders weergegeven bij beide \ac{API}'s. Bovendien is de response header bij het nieuwe kleiner dan bij het oude. Dit wordt weergegeven op Listing \ref{Q1AADRH} en \ref{Q1MSGRH}. \\

Het tweede verzoek maakt gebruik van een request body waar een opvallend verschil is aan te merken tussen de twee. Het verschil is te zien bij Listing \ref{Q2AADRQB} en \ref{Q2MSGRQB}. Het oude maakt gebruik van de “forceChangePasswordNextLogin”, terwijl het nieuwe “forceChangePasswordNextSignin” gebruikt als parameter. Vervolgens is de response body bij Microsoft Graph veel kleiner. De response body van beide wordt weergegeven bij Listing \ref{Q2AADRB} en \ref{Q2MSGRB}. Ook is bij deze query de reponse header kleiner bij het nieuwe dan bij het oude. Dit wordt voorgesteld bij Listing \ref{Q2AADRH} en \ref{Q2MSGRH}. \\

Bij de derde query zijn beide request bodies hetzelfde. Beide request bodies worden voorgesteld in Listing \ref{Q3AADRQB} en \ref{Q3MSGRQB}. De resterende onderdelen bij de derde en vierde query zijn gelijkaardig aan elkaar.

\subsection{PowerShell-versies}

% TODO: Laten nakijken door Jarne

Azure \ac{AD} Graph en Microsoft Graph maken gebruik van PowerShell-modules. Deze PowerShell-modules worden vernieuwd aan de hand van versies die nieuwe of stabiele elementen toevoegen (of verwijderen) aan de PowerShell-modules. \\

Azure \ac{AD} Graph werkt met twee soorten PowerShell-modules, namelijk MSOnline en AzureAD. MSOnline is de eerste versie, waarbij AzureAD de verbeterde versie is van MSOnline voor het beheren van Azure \ac{AD}. \\

Bij Microsoft Graph zijn er op dit moment twee versies beschikbaar. Versie 1.0 die de stabiele versie voorstelt. Daarnaast bestaat ook de beta-versie, vandaag de dag voorgesteld als versie 2.0, die de nadruk legt op nieuwe elementen die nog in ontwikkeling zijn. \\ 

% TODO: Eventueel toelichten welke features in 2.0 toegevoegd (zullen) worden?
% ---
% Section
% ---

\section{Aanspreekbare data-objecten en dependencies}

% Idee: uitleggen hoe de API werkt en hoe het aan een data-object kan aanspreken
% Vb. Waar zit het verschil? Is dit hetzelfde gebleven?

%Idee: een tabel of mooi overzicht over hoeveel data-objecten er kunnen worden aangesproken door beide.
%Vb. Zijn er veel nieuwe bijgekomen? Valt er iets op? Is er iets weggegaan...
% Hier wordt het data-object 


Met de migratie van Azure \ac{AD} Graph naar Microsoft Graph, wordt er verondersteld dat alle aanspreekbare data-objecten worden overgenomen. Wanneer het overzicht van data-objecten bij Azure \ac{AD} PowerShell naast dat van Microsoft Graph wordt gezet, ontstaat volgend overzicht dat te vinden is in Tabel \ref{AADMSG}. \\

\begin{table}
    \centering
    \begin{adjustbox}{max width=\textwidth}
    \begin{tabular}{ |c|c||c|c| } 
        \hline
        \textbf{Azure AD Graph data-object} & \textbf{Aantal commando's} & \textbf{Microsoft Graph data-object} & \textbf{Aantal commando's} \\
        \hline
        Administrative Units & 9 & Administrative Units & 9 \\ 
        Application Proxy Application Management & 8 & / & / \\
        Application Proxy Connector Management & 9 & / & / \\
        Applications & 20 & Applications & 20 \\ 
        AzureAD & 49 & AzureAD & 48 \\ 
        Certificate Authorities & 4 & Certificate Authorities & 1 \\ 
        Connect to directory & 2 & Connect to directory & 2 \\ 
        Contacts & 8 & Contacts & 7 \\ 
        Contracts & 1 & Contracts & 1 \\ 
        Deleted Objects & 1 & Deleted Objects & 1 \\ 
        Devices & 11 & Devices & 9 \\    
        Directory & 3 & Directory & 3 \\
        Directory Objects & 1 & Directory Objects & 1 \\ 
        Directory Roles & 13 & Directory Roles & 13 \\ 
        Domains & 8 & Domains & 8 \\ 
        Extension Properties & 1 & Extension Properties & 1 \\ 
        Groups & 26 & Groups & 26 \\ 
        MSOnline & 97 & MSOnline & 76 \\
        OAuth2 & 2 & OAuth2 & 2 \\ 
        Policies & 2 & Policies & 2 \\ 
        Service Principals & 22 & Service Principals & 22 \\ 
        Users & 30 & Users & 30 \\ 
        \hline
    \end{tabular}
    \end{adjustbox}
    \caption[Tabel migratie Azure AD data-objecten naar Microsoft Graph]{Tabel met overzicht van alle ondersteunende data-objecten bij de migratie van Azure \ac{AD} PowerShell naar Microsoft Graph uit documentatie van \textcite{Microsoft2023l}}
    \label{AADMSG}
\end{table}

Uit Tabel \ref{AADMSG} blijkt dat Microsoft Graph 17 van de 22 Azure \ac{AD} data-objecten overneemt. Er zijn dus enkele verschillen aanwezig door deze migratie. Zowel de Application Proxy Application Management en Application Proxy Connector Management vallen weg. Daarnaast is er een commando weggehaald uit het AzureAD en Contacts data-object. Vervolgens zijn er twee commando's minder bij het Devices data-object. Finaal zijn er 21 commando's weggehaald uit het MSOnline data-object in Microsoft Graph. \\

In het specifiek gaat het om volgende commando's die vandaag de dag niet meer aanwezig zijn in Microsoft Graph.

\begin{itemize}
    \item Azure AD
    \begin{itemize}
        \item Get-AzureADApplicationProxyConnectorGroupMember
    \end{itemize}
    \item Certificate Authorities
    \begin{itemize}
        \item New-AzureADTrustedCertificateAuthority
        \item Remove-AzureADTrustedCertificateAuthority
        \item Set-AzureADTrustedCertificateAuthority
    \end{itemize}
    \item Contacts
    \begin{itemize}
        \item Remove-AzureADContactManager
    \end{itemize}
    \item Devices
    \begin{itemize}
        \item Add-AzureADDeviceRegisteredUser
        \item Remove-AzureADDeviceRegisteredUser
    \end{itemize}
    \item MSOnline
    \begin{itemize}
        \item Add-MsolForeignGroupToRole
        \item Confirm-MsolEmailVerifiedDomain
        \item Convert-MsolFederatedUser
        \item Get-MsolCompanyAllowedDataLocation
        \item Get-MsolDirSyncConfiguration
        \item Get-MsolDirSyncFeatures
        \item Get-MsolFederationProperty
        \item Get-MsolHasObjectsWithDirSyncProvisioningErrors
        \item Get-MsolPartnerInformation
        \item New-MsolFederatedDomain
        \item New-MsolWellKnownGroup
        \item Remove-MsolFederatedDomain
        \item Remove-MsolForeignGroupFromRole
        \item Reset-MsolStrongAuthenticationMethodByUpn
        \item Set-MsolADFSContext
        \item Set-MsolCompanyAllowedDataLocation
        \item Set-MsolCompanyMultiNationalEnabled
        \item Set-MsolDirSyncConfiguration
        \item Set-MsolDirSyncFeature
        \item Set-MsolPartnerInformation
        \item Update-MsolFederatedDomain    
    \end{itemize}
\end{itemize}

Naast de data-objecten bevat Microsoft Graph meer dependencies, dit wordt weergegeven in Tabel \ref{MSGDT}. Dit wilt zeggen dat Microsoft Graph meer Microsoft-entiteiten kan aanspreken dan de Azure \ac{AD} PowerShell-modules. 

% ---
% Section
% ---

\section{Toegangsmogelijkheden}

%Idee: Via welke methodes kan men zich inloggen? 

%Vb. Is er een verschil? Wordt er MFA ondersteund? ... (Gaat nog niet in detail over veiligheid -> zie Security)

Voor beide technologieën bestaan de twee dezelfde toegangsmogelijkheden, namelijk Delegated en App-only. Er zijn geen verschillen aan te merken voor de toegangsmogelijkheden. \\

De enige opmerking dat er kan gegeven worden, maar buiten de scope van dit onderdeel van de vergelijking valt, is dat \ac{ADAL} niet meer ondersteunt wordt op 30 juni 2023. Wanneer er wordt gebruikgemaakt van een library om in te loggen, bijvoorbeeld voor App-only access met secrets, dan kan dit via \ac{MSAL}.


% ---
% Section
% ---

\section{Security}

\subsection{Veiligheid van de toegangsmogelijkheden}

De eerste methode is de gedelegeerde of interactieve manier. Deze methode maakt gebruik van een interactie van een gebruiker om de data te kunnen raadplegen. Hier ligt dus de nadruk op de gebruiker die wilt inloggen. \\

De gebruiker die de actie uitvoert heeft bepaalde rechten verkregen. Dit kan gaan over leesrechten, maar ook over schrijfrechten. De veiligheid van de data wordt bepaald door twee factoren. \\

Ten eerste, de rechten van de gebruiker. Een beheerder heeft normalerwijze veel rechten om beheertaken te kunnen uitvoeren, zoals het lezen of schrijven van data. Wanneer er wordt ingelogd met een beheerder zijn de mogelijke consequenties ook groter, ten opzichte van een gebruiker die alleen maar data kan opvragen. \\ 

Ten tweede, de getroffen methodes om het aanmelden mogelijk te maken. Als de gebruiker alleen maar een naam en wachtwoord moet meegeven, is de kans op een inbreuk groter dan bij een gebruiker die gebruikmaakt van \ac{2FA} of \Ac{MFA}. \\

De tweede methode is de geprogrammeerde manier of niet-interactieve manier. Deze methode maakt geen of minder gebruik van interacties in tegenstelling tot de eerste manier. De nadruk ligt op een applicatie die gebruikt wordt om toegang te verschaffen. \\

De applicatie heeft net zoals een gebruiker rechten die kunnen worden ingesteld. Deze applicatie staat los van de gebruikers en kan alleen worden aangepast door een gebruiker die toegang heeft tot het domein. \\

Een applicatie kan niet gebruikmaken van \ac{MFA}, doordat er geen interacties plaatsvinden. In plaats daarvan is het mogelijk om gebruik te maken van secrets. Het principe van een secret is dat de waarde alleen bij het aanmaken wordt weergegeven, daarna kan het niet meer worden opgevraagd.

\subsection{Gebruik van rechten}

% Idee: Het beginsel van de minste voorrechten (PoLP) wordt toegepast. Is dit bij beide zo en hoe sterk is dit principe?

% Vb. Leg PoLP dieper uit, wat kan het tegenhouden, waarom wordt het toegepast...

Beide technologieën maken gebruik van rechten tot groepen van data. Standaard krijgt een gebruiker geen rechten en moeten deze worden verkregen via een gemachtigde gebruiker. Door deze aanpak wordt het “Principle of Least Privilege” toegepast. \\

De migratie van Azure \ac{AD} Graph naar Microsoft Graph op vlak van permissie scopes wordt weergegeven in Tabel \ref{AADMSGPS}. Deze tabel focust zich alleen op de overgang van Azure \ac{AD} scopes naar Microsoft Graph. Microsoft Graph heeft meer permissie scopes dan in de tabel wordt voorgesteld. \\

Het onderdeel “A/D” staat voor de beschikbaarheid van de permissie als Application of App-only en Delegated of interactief. \\

% TODO: Mini conclusie schrijven!

\begin{table}
    \centering
    \begin{adjustbox}{max width=\textwidth}
    \begin{tabular}{ |c|c||c|c|c| } 
        \hline
        \textbf{Azure AD scope} & \textbf{A/D} & \textbf{Microsoft Graph permissie domein} & \textbf{Microsoft Graph scope} & \textbf{A/D} \\
        \hline
        User.Read & D & User permissions & User.Read & D \\
        User.ReadBasic.All & D & User permissions & User.ReadBasic.All & D \\
        User.Read.All & D & User permissions & User.Read.All & A/D \\
        Group.Read.All & D & Group permissions & Group.Read.All & A/D \\
        Group.ReadWrite.All & D & Group permissions & Group.ReadWrite.All & A/D \\
        Device.ReadWrite.All & A & Device permissions & Device.ReadWrite.All & A \\
        Directory.Read.All & A/D & Directory permissions & Directory.Read.All & A/D \\
        Directory.ReadWrite.All & A/D & Directory permissions & Directory.ReadWrite.All & A/D \\
        Directory.AccessAsUser.All & D & Directory permissions & Directory.AccessAsUser.All & D \\ 
        \hline
    \end{tabular}
    \end{adjustbox}
    \caption[Tabel migratie Azure AD permissie scopes naar Microsoft Graph]{Tabel met overzicht van alle ondersteunende permissie scopes bij de migratie van Azure \ac{AD} PowerShell naar Microsoft Graph uit Tabel \ref{psaad} en documentatie van \textcite{Microsoft2023p}}
    \label{AADMSGPS}
\end{table}

Tabel \ref{AADMSGPS} geeft weer dat er geen veranderingen zijn gebeurd op vlak van de scopes. Microsoft Graph biedt de extra mogelijkheid door het op te splitsen per domein. Daarnaast heeft Microsoft Graph drie extra scopes Application en Delegated-access gemaakt. \\
