%==============================================================================
% Sjabloon onderzoeksvoorstel bachproef
%==============================================================================
% Gebaseerd op document class `hogent-article'
% zie <https://github.com/HoGentTIN/latex-hogent-article>

% Voor een voorstel in het Engels: voeg de documentclass-optie [english] toe.
% Let op: kan enkel na toestemming van de bachelorproefcoördinator!
\documentclass{hogent-article}
\usepackage{comment}

% Invoegen bibliografiebestand
\addbibresource{voorstel.bib}

% Informatie over de opleiding, het vak en soort opdracht
\studyprogramme{Professionele bachelor toegepaste informatica}
\course{Bachelorproef}
\assignmenttype{Onderzoeksvoorstel}
% Voor een voorstel in het Engels, haal de volgende 3 regels uit commentaar
% \studyprogramme{Bachelor of applied information technology}
% \course{Bachelor thesis}
% \assignmenttype{Research proposal}

\academicyear{2022-2023} % TODO: pas het academiejaar aan

% TODO: Werktitel
\title{Microsoft administration met Microsoft Graph: van module naar API}

% TODO: Studentnaam en emailadres invullen
\author{Dylan Pylyser}
\email{dylan.pylyser@student.hogent.be}

% TODO: Medestudent
% Gaat het om een bachelorproef in samenwerking met een student in een andere
% opleiding? Geef dan de naam en emailadres hier
% \author{Yasmine Alaoui (naam opleiding)}
% \email{yasmine.alaoui@student.hogent.be}

% TODO: Geef de co-promotor op
\supervisor[Co-promotor]{Jarne Creten (Easi, \href{mailto:j.creten@easi.net}{j.creten@easi.net})}

% Binnen welke specialisatierichting uit 3TI situeert dit onderzoek zich?
% Kies uit deze lijst:
%
% - Mobile \& Enterprise development
% - AI \& Data Engineering
% - Functional \& Business Analysis
% - System \& Network Administrator
% - Mainframe Expert
% - Als het onderzoek niet past binnen een van deze domeinen specifieer je deze zelf
%
\specialisation{System \& Network Administrator}
\keywords{Microsoft Graph, Microsoft 365, Future-proofing}

\begin{document}

\begin{abstract} % TODO: Samenvatting:
De visie van Microsoft naar een centraal ecosysteem bevat introduceren en uitfaseren van technologie. De bestaande Azure Active Directory Graph API moest plaatsmaken voor de opkomende Microsoft Graph. Deze aanpassing leidt tot verouderde scripts bij het IT-bedrijf Easi, die dagelijks met de uitfaserende Azure AD modules werkt. Microsoft Graph is een universele REST API die als eindpunt dient voor Microsoft 365 tenants. Voor productieomgevingen biedt Graph onzekerheid, waarbij dit onderzoek de mogelijkheden van Graph illustreert om zekerheid aan te tonen. Deze casus wordt uitgewerkt aan de hand van twee fases, een vergelijkende studie tussen de oude module en de nieuwe API met Proof-of-Concept. De Proof-of-Concept zal een audit-script met de verouderde modules vervangen door de opkomende API. Er wordt verwacht dat Microsoft Graph, vandaag de dag, niet stabiel is voor alle productieomgevingen met Microsoft 365 tenants. Er wordt wel verwacht dat Microsoft Graph verbeteringen en meerwaarde aantoont aan de hand van een verbeterd audit-script. Daarnaast heeft Graph meer voordelen op de vier onderdelen in vergelijking met de uitfaserende modules.
\end{abstract}

\tableofcontents

% De hoofdtekst van het voorstel zit in een apart bestand, zodat het makkelijk
% kan opgenomen worden in de bijlagen van de bachelorproef zelf.
%---------- Inleiding ---------------------------------------------------------

\section{Introductie}%
\label{sec:introductie}

\begin{comment}
Waarover zal je bachelorproef gaan? Introduceer het thema en zorg dat volgende zaken zeker duidelijk aanwezig zijn:

\begin{itemize}
  \item kaderen thema
  \item de doelgroep
  \item de probleemstelling en (centrale) onderzoeksvraag
  \item de onderzoeksdoelstelling
\end{itemize}

Denk er aan: een typische bachelorproef is \textit{toegepast onderzoek}, wat betekent dat je start vanuit een concrete probleemsituatie in bedrijfscontext, een \textbf{casus}. Het is belangrijk om je onderwerp goed af te bakenen: je gaat voor die \textit{ene specifieke probleemsituatie} op zoek naar een goede oplossing, op basis van de huidige kennis in het vakgebied.

De doelgroep moet ook concreet en duidelijk zijn, dus geen algemene of vaag gedefinieerde groepen zoals \emph{bedrijven}, \emph{developers}, \emph{Vlamingen}, enz. Je richt je in elk geval op it-professionals, een bachelorproef is geen populariserende tekst. Eén specifiek bedrijf (die te maken hebben met een concrete probleemsituatie) is dus beter dan \emph{bedrijven} in het algemeen.

Formuleer duidelijk de onderzoeksvraag! De begeleiders lezen nog steeds te veel voorstellen waarin we geen onderzoeksvraag terugvinden.

Schrijf ook iets over de doelstelling. Wat zie je als het concrete eindresultaat van je onderzoek, naast de uitgeschreven scriptie? Is het een proof-of-concept, een rapport met aanbevelingen, \ldots Met welk eindresultaat kan je je bachelorproef als een succes beschouwen?

\end{comment}

Tijdens en na de COVID-19-pandemie is het gebruik van cloud computing services aanzienlijk toegenomen, bevestigt de \textcite{EU2021}. Hierdoor stijgt de populariteit van Cloud-providers en de nood naar automatisatie hiervan. Bestaande onderzoeken naar Infrastructure Automation tools waaronder Ansible \autocite{RedHat2022}, Chef \autocite{PSC2022}, Packer \autocite{HashiCorp2022}, Puppet \autocite{Perforce2022} en Terraform \autocite{HashiCorp2022a} illustreren het gebruik en potentieel van deze tools. Daarnaast is er ook een mogelijkheid om infrastructuren te automatiseren en te configureren binnen Cloud-providers met deze tools. 

Het bedrijf Easi focust zich vooral rondom het Windows-framework. Hierbij gebruikt het Microsoft Azure \autocite{Microsoft2022} als Cloud-provider. Indien het bedrijf virtuele omgevingen moet opzetten, wordt dit gedaan via de GUI (Graphical User Interface). Deze methode is inefficiënt en kan geoptimaliseerd worden. Door deze eerste stap binnenin de Infrastructure Deployment and Management Lifecycle te automatiseren, versneld dit het Deployment proces en kan het dienen als Disaster Recovery. 

Dit onderzoek focust zich op de volgende onderzoeksvraag: “Wat zijn de mogelijkheden van niet Infrastructure Automation Tools tot het automatiseren van Azure configuraties in verband met een Linux-omgeving?”. Het doel wordt bereikt door eerst een vergelijking te maken van alle mogelijke tools. Nadien volgt een Proof-of-Concept die de werking van de tools in deze casus beschrijft. Vervolgens wordt er via Ansible een virtuele Linux-omgeving opgezet dat dient als nabootsing van een realistische omgeving. Als laatste volgt er een Disaster Recovery-scenario waarbij de omgeving met opzet wordt beschadigd en opnieuw wordt opgezet. Hierdoor wordt de impact geanalyseerd door het gebruikmaken van geautomatiseerde configuraties via Infrastructure Automation tools.



%---------- Stand van zaken ---------------------------------------------------

\section{State-of-the-art}%
\label{sec:state-of-the-art}

\begin{comment}

Hier beschrijf je de \emph{state-of-the-art} rondom je gekozen onderzoeksdomein, d.w.z.\ een inleidende, doorlopende tekst over het onderzoeksdomein van je bachelorproef. Je steunt daarbij heel sterk op de professionele \emph{vakliteratuur}, en niet zozeer op populariserende teksten voor een breed publiek. Wat is de huidige stand van zaken in dit domein, en wat zijn nog eventuele open vragen (die misschien de aanleiding waren tot je onderzoeksvraag!)?

Je mag de titel van deze sectie ook aanpassen (literatuurstudie, stand van zaken, enz.). Zijn er al gelijkaardige onderzoeken gevoerd? Wat concluderen ze? Wat is het verschil met jouw onderzoek?

Verwijs bij elke introductie van een term of bewering over het domein naar de vakliteratuur, bijvoorbeeld~\autocite{Hykes2013}! Denk zeker goed na welke werken je refereert en waarom.

Draag zorg voor correcte literatuurverwijzingen! Een bronvermelding hoort thuis \emph{binnen} de zin waar je je op die bron baseert, dus niet er buiten! Maak meteen een verwijzing als je gebruik maakt van een bron. Doe dit dus \emph{niet} aan het einde van een lange paragraaf. Baseer nooit teveel aansluitende tekst op eenzelfde bron.

Als je informatie over bronnen verzamelt in JabRef, zorg er dan voor dat alle nodige info aanwezig is om de bron terug te vinden (zoals uitvoerig besproken in de lessen Research Methods).

% Voor literatuurverwijzingen zijn er twee belangrijke commando's:
% \autocite{KEY} => (Auteur, jaartal) Gebruik dit als de naam van de auteur
%   geen onderdeel is van de zin.
% \textcite{KEY} => Auteur (jaartal)  Gebruik dit als de auteursnaam wel een
%   functie heeft in de zin (bv. ``Uit onderzoek door Doll & Hill (1954) bleek
%   ...'')

Je mag deze sectie nog verder onderverdelen in subsecties als dit de structuur van de tekst kan verduidelijken.

\end{comment}

AWS \autocite{AWS2022}, de grootste Cloud-provider volgens \textcite{Vailshery2022} en \textcite{SRG2022}, geeft de mogelijkheid tot automatisatie van Cloud Computing Services op hun platform. Hierdoor zijn er veel onderzoeken binnen AWS uitgevoerd omtrent automatisatie van virtuele-omgevingen. Dit onderzoek zal zich focussen op Azure, de grootste concurrent van AWS op dit moment.

Azure biedt ondersteuning aan voor verschillende niet Microsoft-gerelateerde infrastructure automation tools. Configuration management tool zoals Ansible, Chef en Puppet werden al onderzocht en onderling vergeleken in andere studies \autocite{Microsoft2022a}. 

Ansible biedt goede ondersteuning aan op \newline lange termijn en is simpel in gebruik met goede documentatie. Daarnaast is Ansible een krachtige tool met veel potentieel \autocite{Masek2018}. Het wordt eerder wel aangeraden voor kleinere projecten door zijn onvolledigheid. Waarbij Chef en Puppet meer compleet zijn en grote projecten beter kan verwerken \autocite{Bertram2016}. 

Naast bovenstaande tools ondersteunt Azure orchestration tools waaronder Terraform. Deze tool is onveranderlijk ten opzicht van configuration management tools. Deze wordt gebruikt om machines te orkestreren wat makkelijker is om meerdere instanties te voorzien. Terwijl configuration management tools zich focussen op het configureren van instanties \autocite{Brikman2016}.

Binnen Hogeschool Gent zijn er reeds een aantal studies uitgevoerd naar aspecten die voorkomen in dit onderwerp. Kelvin \textcite{Vermeulen2021} heeft een onderzoek uitgevoerd naar het automatiseren van een Public Cloud-omgeving binnen AWS. Daarnaast heeft Joachim \textcite{VandeKeere2021} een studie uitgevoerd naar het gebruik van Ansible binnen lokale omgevingen. 

In het eerste onderzoek van Vermeulen werd de configuratie van AWS manueel uitgevoerd. Vervolgens werd in het tweede onderzoek van Van der Keere Ansible gebruikt voor een lokale  \newline Windows- en Linux-omgeving. In deze studie worden de configuraties van Azure geautomatiseerd. Verder wordt er een Linux-testomgeving opgezet in de Cloud via Ansible.

%---------- Methodologie ------------------------------------------------------
\section{Methodologie}%
\label{sec:methodologie}

\begin{comment}

Hier beschrijf je hoe je van plan bent het onderzoek te voeren. Welke onderzoekstechniek ga je toepassen om elk van je onderzoeksvragen te beantwoorden? Gebruik je hiervoor literatuurstudie, interviews met belanghebbenden (bv.~voor requirements-analyse), experimenten, simulaties, vergelijkende studie, risico-analyse, PoC, \ldots?

Valt je onderwerp onder één van de typische soorten bachelorproeven die besproken zijn in de lessen Research Methods (bv.\ vergelijkende studie of risico-analyse)? Zorg er dan ook voor dat we duidelijk de verschillende stappen terug vinden die we verwachten in dit soort onderzoek!

Vermijd onderzoekstechnieken die geen objectieve, meetbare resultaten kunnen opleveren. Enquêtes, bijvoorbeeld, zijn voor een bachelorproef informatica meestal \textbf{niet geschikt}. De antwoorden zijn eerder meningen dan feiten en in de praktijk blijkt het ook bijzonder moeilijk om voldoende respondenten te vinden. Studenten die een enquête willen voeren, hebben meestal ook geen goede definitie van de populatie, waardoor ook niet kan aangetoond worden dat eventuele resultaten representatief zijn.

Uit dit onderdeel moet duidelijk naar voor komen dat je bachelorproef ook technisch voldoen\-de diepgang zal bevatten. Het zou niet kloppen als een bachelorproef informatica ook door bv.\ een student marketing zou kunnen uitgevoerd worden.

Je beschrijft ook al welke tools (hardware, software, diensten, \ldots) je denkt hiervoor te gebruiken of te ontwikkelen.

Probeer ook een tijdschatting te maken. Hoe lang zal je met elke fase van je onderzoek bezig zijn en wat zijn de concrete \emph{deliverables} in elke fase?

\end{comment}

Het onderzoek omvat vier fases. In de eerste fase wordt er een vergelijkende studie uitgevoerd. Deze tools worden onder de loep genomen en worden vergeleken op vlak van logica, leesbaarheid, verwerkingssnelheid, documentatie, community en eventuele limieten. Er wordt zowel gebruikgemaakt van eigen simulaties binnenin Azure als studies en vakliteratuur. Achteraf volgt er een verklarende tekst die alle resultaten en bevinden op papier zet. 

De tweede fase bestaat uit een Proof-of-Concept die de werking van de infrastructure automation tools illustreert. De tools worden gebruikt om de configuraties van Azure te automatiseren om vervolgens te kunnen gebruiken voor een virtuele omgeving. Hierbij worden er scripts uitgewerkt via de Azure CLI (Command Line Interface). Deze scripts worden vervolgens uitgewerkt in de scriptie met eventuele bevindingen. Deze fase is doorslaggevend voor de casus en de onderzoeksvraag.

In de derde fase volgt een nabootsing van een realistische Linux-omgeving. Aan de hand van de geautomatiseerde configuraties ontstaat de mogelijkheid om de omgeving op te zetten. Hierbij wordt Ansible als configuration management tool voor het opzetten van de infrastructuur. In de testomgeving voorzien we vier virtuele servers en een virtuele client, waarbij alle virtuele machines AlmaLinux (BRON) als distributie gebruiken. De eerste server maakt gebruik van BIND (BRON) om de DNS te voorzien. De tweede server voorziet DHCP voor dynamisch IP-adressering voor clients. De derde server fungeert als webserver en databank met een WordPress (BRON) installatie. De laatste server maakt gebruik van Prometheus (BRON) en Grafana (BRON) om de servers te kunnen monitoren. De client is een gebruiker van het domein en test de beschikbare functionaliteiten.

De laatste fase omvat een Disaster Recovery-scenario waarbij de omgeving wordt beschadigd. Door deze beschadiging wordt de omgeving van voren af opgezet aan de hand van de automatisatie uit de tweede fase. Hierbij ligt de focus op de impact dat het heeft via de geautomatiseerde manier ten opzichte van de manuele manier. Deze impact wordt geanalyseerd en getest, nadien worden alle bevinden verwerkt in de scriptie.


%---------- Verwachte resultaten ----------------------------------------------
\section{Verwacht resultaat, conclusie}%
\label{sec:verwachte_resultaten}

Hier beschrijf je welke resultaten je verwacht. Als je metingen en simulaties uitvoert, kan je hier al mock-ups maken van de grafieken samen met de verwachte conclusies. Benoem zeker al je assen en de onderdelen van de grafiek die je gaat gebruiken. Dit zorgt ervoor dat je concreet weet welk soort data je moet verzamelen en hoe je die moet meten.

Wat heeft de doelgroep van je onderzoek aan het resultaat? Op welke manier zorgt jouw bachelorproef voor een meerwaarde?

Hier beschrijf je wat je verwacht uit je onderzoek, met de motivatie waarom. Het is \textbf{niet} erg indien uit je onderzoek andere resultaten en conclusies vloeien dan dat je hier beschrijft: het is dan juist interessant om te onderzoeken waarom jouw hypothesen niet overeenkomen met de resultaten.



\emergencystretch=1em

\printbibliography[heading=bibintoc]

\end{document}