%---------- Inleiding ---------------------------------------------------------

\section{Introductie}%
\label{sec:introductie}

\begin{comment}
Waarover zal je bachelorproef gaan? Introduceer het thema en zorg dat volgende zaken zeker duidelijk aanwezig zijn:

\begin{itemize}
  \item kaderen thema
  \item de doelgroep
  \item de probleemstelling en (centrale) onderzoeksvraag
  \item de onderzoeksdoelstelling
\end{itemize}

Denk er aan: een typische bachelorproef is \textit{toegepast onderzoek}, wat betekent dat je start vanuit een concrete probleemsituatie in bedrijfscontext, een \textbf{casus}. Het is belangrijk om je onderwerp goed af te bakenen: je gaat voor die \textit{ene specifieke probleemsituatie} op zoek naar een goede oplossing, op basis van de huidige kennis in het vakgebied.

De doelgroep moet ook concreet en duidelijk zijn, dus geen algemene of vaag gedefinieerde groepen zoals \emph{bedrijven}, \emph{developers}, \emph{Vlamingen}, enz. Je richt je in elk geval op it-professionals, een bachelorproef is geen populariserende tekst. Eén specifiek bedrijf (die te maken hebben met een concrete probleemsituatie) is dus beter dan \emph{bedrijven} in het algemeen.

Formuleer duidelijk de onderzoeksvraag! De begeleiders lezen nog steeds te veel voorstellen waarin we geen onderzoeksvraag terugvinden.

Schrijf ook iets over de doelstelling. Wat zie je als het concrete eindresultaat van je onderzoek, naast de uitgeschreven scriptie? Is het een proof-of-concept, een rapport met aanbevelingen, \ldots Met welk eindresultaat kan je je bachelorproef als een succes beschouwen?

\end{comment}

Naar aanleiding van de snel evoluerende cloudtechnologie, focust Microsoft zich op het bouwen van een ecosysteem dat bestuurd kan worden vanuit zo veel mogelijk platformen \autocite{Parker2021}. \\

Deze visie van Microsoft brengt de volgende verschuiving met zich mee, namelijk het uitfaseren van de bestaande Azure AD (Active Directory) Graph API (Application Programming Interface) en ADAL (Azure AD Authentication Library) \autocite{Sahay2022}. Het uitrollen van Microsoft Graph bij Microsoft 365 tenants staat gepland op 30 juni 2023. \\

Bij het IT-bedrijf Easi worden courante taken, automatisaties en wederkerende acties door de vooraf vermelde Azure AD PowerShell modules. Door de uitfasering van de Azure AD modules is het bedrijf genoodzaakt om een oplossing te vinden voor dit probleem. \\

Een oplossing is het gebruik van MSAL (Microsoft Authentication Library) en Microsoft Graph API \autocite{Microsoft2023Va}. Microsoft Graph is nog volop in ontwikkeling, waardoor dit nog niet in productieomgevingen wordt ingezet. \\

Dit onderzoek zal zich focussen op wat er vandaag de dag mogelijk is binnen Microsoft Graph. Het doel is om een antwoord te geven op de vraag of Microsoft Graph klaar is om gebruikt te worden als vervanger van de Azure AD PowerShell modules. Het antwoord op de vraag wordt onderbouwd door een vergelijkende studie en Proof-of-Concept die de mogelijkheden van de technologie illustreert.


%---------- Stand van zaken ---------------------------------------------------

\section{State-of-the-art}%
\label{sec:state-of-the-art}

\begin{comment}

Hier beschrijf je de \emph{state-of-the-art} rondom je gekozen onderzoeksdomein, d.w.z.\ een inleidende, doorlopende tekst over het onderzoeksdomein van je bachelorproef. Je steunt daarbij heel sterk op de professionele \emph{vakliteratuur}, en niet zozeer op populariserende teksten voor een breed publiek. Wat is de huidige stand van zaken in dit domein, en wat zijn nog eventuele open vragen (die misschien de aanleiding waren tot je onderzoeksvraag!)?

Je mag de titel van deze sectie ook aanpassen (literatuurstudie, stand van zaken, enz.). Zijn er al gelijkaardige onderzoeken gevoerd? Wat concluderen ze? Wat is het verschil met jouw onderzoek?

Verwijs bij elke introductie van een term of bewering over het domein naar de vakliteratuur, bijvoorbeeld~\autocite{Hykes2013}! Denk zeker goed na welke werken je refereert en waarom.

Draag zorg voor correcte literatuurverwijzingen! Een bronvermelding hoort thuis \emph{binnen} de zin waar je je op die bron baseert, dus niet er buiten! Maak meteen een verwijzing als je gebruik maakt van een bron. Doe dit dus \emph{niet} aan het einde van een lange paragraaf. Baseer nooit teveel aansluitende tekst op eenzelfde bron.

Als je informatie over bronnen verzamelt in JabRef, zorg er dan voor dat alle nodige info aanwezig is om de bron terug te vinden (zoals uitvoerig besproken in de lessen Research Methods).

% Voor literatuurverwijzingen zijn er twee belangrijke commando's:
% \autocite{KEY} => (Auteur, jaartal) Gebruik dit als de naam van de auteur
%   geen onderdeel is van de zin.
% \textcite{KEY} => Auteur (jaartal)  Gebruik dit als de auteursnaam wel een
%   functie heeft in de zin (bv. ``Uit onderzoek door Doll & Hill (1954) bleek
%   ...'')

Je mag deze sectie nog verder onderverdelen in subsecties als dit de structuur van de tekst kan verduidelijken.

\end{comment}

Microsoft Azure is de voorbije jaren de grootste concurrent van AWS binnen Cloud Computing Services, blijkt uit onderzoek van \textcite{Vailshery2022} en \textcite{SRG2022}. Het platform biedt services in de Cloud aan waaronder Azure Active Directory. Azure AD kan beheerd worden vanuit bijhorende PowerShell modules, echter wordt de ondersteuning voor deze modules stopgezet, zoals vermeld in de introductie. \\

Microsoft 365-applicaties en -diensten hebben een eigen API om data te raadplegen. In projecten of applicaties waarbij  Microsoft 365-services moeten aangesproken worden, kan dit tot problemen leiden. Een voorbeeld hiervan is dat de API van Azure AD andere functies en werkwijzen heeft in vergelijking met een SharePoint API \autocite{VanRousselt2021}. Deze verschillen leiden tot inefficiëntie. Een oplossing is het gebruik van een algemene API die alle services kan aanspreken. \\

Microsoft Graph is een toegangspoort om data te kunnen raadplegen in Microsoft 365-applicaties en -diensten. Graph bevat een universele REST (Representational State Transfer) API dat toegepast kan worden voor deze problemen. De API heeft als doel een eindpunt aan te bieden voor de verschillende applicaties en services, waardoor deze een coherent geheel vormt. \\

Uit onderzoek blijkt dat Microsoft Graph stabiel genoeg is om specifieke taken binnen Microsoft 365-applicaties en -diensten uit te voeren, ondanks dat het nog in de ontwikkelfase zit. \textcite{Hoefling2022} maakt gebruik van Graph om OneDrive data te kunnen raadplegen. \textcite{Jenkins2021} illustreert het gebruik van Graph in combinatie met Teams. Daarnaast bevestigen \textcite{Parsa2019} de eenvoud van de Graph API om Outlook data te verwerken in een kamerbeheersysteem. \\

Deze studie legt de focus op Azure Active Directory, een Microsoft service dat nog niet in onderzoek werd verwerkt met Microsoft Graph. 

%---------- Methodologie ------------------------------------------------------
\section{Methodologie}%
\label{sec:methodologie}

\begin{comment}

Hier beschrijf je hoe je van plan bent het onderzoek te voeren. Welke onderzoekstechniek ga je toepassen om elk van je onderzoeksvragen te beantwoorden? Gebruik je hiervoor literatuurstudie, interviews met belanghebbenden (bv.~voor requirements-analyse), experimenten, simulaties, vergelijkende studie, risico-analyse, PoC, \ldots?

Valt je onderwerp onder één van de typische soorten bachelorproeven die besproken zijn in de lessen Research Methods (bv.\ vergelijkende studie of risico-analyse)? Zorg er dan ook voor dat we duidelijk de verschillende stappen terug vinden die we verwachten in dit soort onderzoek!

Vermijd onderzoekstechnieken die geen objectieve, meetbare resultaten kunnen opleveren. Enquêtes, bijvoorbeeld, zijn voor een bachelorproef informatica meestal \textbf{niet geschikt}. De antwoorden zijn eerder meningen dan feiten en in de praktijk blijkt het ook bijzonder moeilijk om voldoende respondenten te vinden. Studenten die een enquête willen voeren, hebben meestal ook geen goede definitie van de populatie, waardoor ook niet kan aangetoond worden dat eventuele resultaten representatief zijn.

Uit dit onderdeel moet duidelijk naar voor komen dat je bachelorproef ook technisch voldoen\-de diepgang zal bevatten. Het zou niet kloppen als een bachelorproef informatica ook door bv.\ een student marketing zou kunnen uitgevoerd worden.

Je beschrijft ook al welke tools (hardware, software, diensten, \ldots) je denkt hiervoor te gebruiken of te ontwikkelen.

Probeer ook een tijdschatting te maken. Hoe lang zal je met elke fase van je onderzoek bezig zijn en wat zijn de concrete \emph{deliverables} in elke fase?

\end{comment}

Dit onderzoek bestaat uit twee fases. \\
 
In de eerste fase wordt er een vergelijkende studie uitgevoerd tussen de uitfaserende Azure AD PowerShell modules en Microsoft Graph. Deze twee technologieën worden vergeleken op vlak van achterliggende logica, aanspreekbare data-objecten, gebruik en security. Deze vergelijking verduidelijkt de evolutie van Azure AD naar Graph. \\

Eerst, worden de achterliggende logica van de twee technologieën geanalyseerd. De focus ligt op de werking en communicatie met Microsoft 365-tenants. \\

Ten tweede, worden Microsoft 365 data-objecten onder de loep genomen. Bij dit onderdeel ligt de nadruk op welke Microsoft 365-applicaties en -services aanspreekbaar zijn door beide technologieën \\

Vervolgens, wordt het gebruik besproken. Dit wilt zeggen, hoe zien beide technologieën eruit op vlak van code en notatie. Een voorbeeld hiervan is hoe een functie gedeclareerd wordt bij beiden. Dit leidt tot meer begrip over de automatisatie en complexiteit van een script. \\

Als laatste, worden de PowerShell modules en Graph vergeleken op security. De focus wordt gelegd op welke data beide technologieën verwerken en nodig hebben. Bovendien wordt er gekeken naar mogelijke veiligheidsrisico's. Dit zal bepalen hoe veilig het gebruik van beide technologieën zijn. \\

De tweede fase is het opstellen van een Proof-of-Concept die de werking van Microsoft Graph illustreert. Voor deze uitwerking wordt er gebruikgemaakt van een bestaand PowerShell script. Dit script omvat het genereren van Office 365 Security Audit documenten voor klanten van Easi. Het genereren van deze data wordt via Azure AD PowerShell modules voorzien, waardoor deze kwetsbaar is voor de uitfasering. \\

Als Proof-of-Concept wordt bovenstaand script omgevormd en uitgewerkt met Microsoft Graph. Het doel van deze uitwerking is om te bewijzen wat Microsoft Graph vandaag de dag al kan en wat nog niet. Bovendien verduidelijkt deze fase de evolutie van Azure AD Graph modules naar de Graph API op een praktische manier. Daarnaast kan deze uitwerking dienen als basis voor andere toepassingen met Graph zoals Microsoft Purview \autocite{Microsoft2023V}. 

%---------- Verwachte resultaten ----------------------------------------------
\section{Verwacht resultaat, conclusie}%
\label{sec:verwachte_resultaten}

\begin{comment}

Hier beschrijf je welke resultaten je verwacht. Als je metingen en simulaties uitvoert, kan je hier al mock-ups maken van de grafieken samen met de verwachte conclusies. Benoem zeker al je assen en de onderdelen van de grafiek die je gaat gebruiken. Dit zorgt ervoor dat je concreet weet welk soort data je moet verzamelen en hoe je die moet meten.

Wat heeft de doelgroep van je onderzoek aan het resultaat? Op welke manier zorgt jouw bachelorproef voor een meerwaarde?

Hier beschrijf je wat je verwacht uit je onderzoek, met de motivatie waarom. Het is \textbf{niet} erg indien uit je onderzoek andere resultaten en conclusies vloeien dan dat je hier beschrijft: het is dan juist interessant om te onderzoeken waarom jouw hypothesen niet overeenkomen met de resultaten.

\end{comment}

Er wordt verwacht dat Microsoft Graph in vergelijking met Azure AD Graph verbeterd is op de vier onderdelen. De logica van Microsoft Graph is anders, echter is de ondersteuning langer door het gebruik van REST. Microsoft Graph ondersteunt meer data-objecten, dus meer Microsoft 365-applicaties en -services. Het gebruik van Microsoft Graph is efficiënter en biedt meer schaalbaarheid, door middel van universele functies en minder soorten API’s. De security van Microsoft Graph is veiliger door een Zero Trust-beleid dat wordt toegepast bij het gebruiken van de Microsoft Graph API. \\

Vervolgens wordt er verwacht dat Microsoft Graph, op dit moment, niet stabiel genoeg is om alle Azure AD PowerShell modules te vervangen tegen de deadline van 30 juni 2023. Bovendien is Microsoft Graph, op dit moment, niet aan te raden in cruciale productieomgevingen van klanten. Echter wordt er wel verwacht dat Microsoft Graph de Proof-of-Concept in verband met auditing kan uitwerken. Hierdoor is het stabiel genoeg om niet-bedrijfskritische toepassingen uit te werken.
